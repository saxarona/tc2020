\documentclass[]{book}

%These tell TeX which packages to use.
\usepackage{array,epsfig}
\usepackage{amsmath}
\usepackage{amsfonts}
\usepackage{amssymb}
\usepackage{amsxtra}
\usepackage{amsthm}
\usepackage{mathrsfs}
\usepackage{color}
\usepackage[spanish, mexico]{babel}
\usepackage[utf8]{inputenc}

%Here I define some theorem styles and shortcut commands for symbols I use often
\theoremstyle{definition}
\newtheorem{defn}{Definition}
\newtheorem{thm}{Theorem}
\newtheorem{cor}{Corollary}
\newtheorem*{rmk}{Remark}
\newtheorem{lem}{Lemma}
\newtheorem*{joke}{Joke}
\newtheorem{ex}{Example}
\newtheorem*{sol}{Solución}
\newtheorem{prop}{Proposition}

\newcommand{\lra}{\longrightarrow}
\newcommand{\ra}{\rightarrow}
\newcommand{\surj}{\twoheadrightarrow}
\newcommand{\graph}{\mathrm{graph}}
\newcommand{\bb}[1]{\mathbb{#1}}
\newcommand{\Z}{\bb{Z}}
\newcommand{\Q}{\bb{Q}}
\newcommand{\R}{\bb{R}}
\newcommand{\C}{\bb{C}}
\newcommand{\N}{\bb{N}}
\newcommand{\M}{\mathbf{M}}
\newcommand{\m}{\mathbf{m}}
\newcommand{\MM}{\mathscr{M}}
\newcommand{\HH}{\mathscr{H}}
\newcommand{\Om}{\Omega}
\newcommand{\Ho}{\in\HH(\Om)}
\newcommand{\bd}{\partial}
\newcommand{\del}{\partial}
\newcommand{\bardel}{\overline\partial}
\newcommand{\textdf}[1]{\textbf{\textsf{#1}}\index{#1}}
\newcommand{\img}{\mathrm{img}}
\newcommand{\ip}[2]{\left\langle{#1},{#2}\right\rangle}
\newcommand{\inter}[1]{\mathrm{int}{#1}}
\newcommand{\exter}[1]{\mathrm{ext}{#1}}
\newcommand{\cl}[1]{\mathrm{cl}{#1}}
\newcommand{\ds}{\displaystyle}
\newcommand{\vol}{\mathrm{vol}}
\newcommand{\cnt}{\mathrm{ct}}
\newcommand{\osc}{\mathrm{osc}}
\newcommand{\LL}{\mathbf{L}}
\newcommand{\UU}{\mathbf{U}}
\newcommand{\support}{\mathrm{support}}
\newcommand{\AND}{\;\wedge\;}
\newcommand{\OR}{\;\vee\;}
\newcommand{\Oset}{\varnothing}
\newcommand{\st}{\ni}
\newcommand{\wh}{\widehat}

%Pagination stuff.
\setlength{\topmargin}{-.3 in}
\setlength{\oddsidemargin}{0in}
\setlength{\evensidemargin}{0in}
\setlength{\textheight}{9.in}
\setlength{\textwidth}{6.5in}
\setlength{\itemsep}{0.45in}
\pagestyle{empty}



\begin{document}

\begin{center}
{\huge Matemáticas Computacionales TC2020 -- N}\\[1.5ex]
{\large \textbf{Tarea 01 -- Soluciones}\\[1.5ex] %You should put your name here
} %You should write the date here.
\end{center}

\vspace{0.2 cm}

\subsection*{Preliminares: Lógica proposicional}

\begin{enumerate}
	\itemsep0.35in

	\item Convierte los siguientes enunciados en otros que digan lo mismo, pero sin implicaciones ni negaciones.
	Primero define el vocabulario proposicional que utilizarás. Después, usando equivalencias lógicas ve eliminando primero implicaciones y luego negaciones.
	Finalmente, escribe la nueva proposición equivalente.
	\begin{enumerate}
		\item Conjunto de números naturales que si contienen al menos un cero, entonces terminan en 1
		\begin{sol}
            Sea $\bb{N}$ el conjunto de números naturales, $P$ el conjunto de números que contienen un cero y $Q$ el conjunto de números que terminan en 1.
            Entonces, podemos definir el `conjunto $C$ de números naturales que si contienen un 0 entonces terminan en 1' como $n \in C$, donde
            $$C = \{n \mid n \in \bb{N}, n \in P \implies n \in Q \}$$
            Específicamente, la condición $P \implies Q$ puede transformarse:
            \begin{align*}
                P \implies Q \equiv \neg P \vee Q & \tag{\text{equivalencia de implicación material}}
            \end{align*}
            de tal manera que el conjunto $C = \{n \mid n \in \bb{N}, n \not \in P \vee n \in Q\}$
        \end{sol}
		\item Conjunto de palabras formadas con $a$s y $b$s que contienen $a$s si y sólo si contienen $b$s.
		\begin{sol}
            Sean $A$ y $B$ los enunciados `contiene $a$s' y `contiene $b$s' respectivamente. Entonces, el conjunto $C$ de `palabras formadas con $a$s y $b$s que contienen $a$s si y sólo si contienen $b$s' puede expresarse como
            $$C = \{c \mid A \iff B\}$$
            La doble implicación puede ser descompuesta fácilmente:
            \begin{align*}
                A \iff B & \equiv \\
                & \equiv A \implies B \wedge B \implies A & \tag{\text{definición de doble implicación}}\\
                & \equiv (\neg A \vee B) \wedge (\neg B \vee A) & \tag{\text{implicación material}}
            \end{align*}
            de tal modo que el conjunto $C = \{c \mid (\neg A \vee B) \wedge (\neg B \vee A)\}$
            que se leería como `cualquier palabra que no tenga $a$ o tenga $b$, y que además no tenga $b$ o tenga $a$'.
        \end{sol}
	\end{enumerate}
	\item Considera el conjunto de números naturales tales que si son mayores que 5 o terminan en 5, entonces contienen algún 1 o 2.
	\begin{enumerate}
		\item Especifica tres números que cumplan la condición y tres que no la cumplan.
		\begin{sol}
            Los números $3, 14, 25$ y $200$ cumplen la condición, mientras que $5, 33, 49$ y $88$ no la cumplen.
        \end{sol}
		\item Expresa el enunciado como una fórmula proposicional donde $M$ significa \textit{mayor que 5}, $T$ significa que \textit{terminan en 5}, $U$ significa que \textit{contienen algún 1} y $D$ significa que \textit{contienen algún 2}.
		\begin{sol}
            $(M \vee T) \implies (U \vee D)$
        \end{sol}
		\item Transforma la fórmula del inciso anterior de forma que se eliminen implicaciones y negaciones, y posteriormente escríbela con palabras.
		\begin{sol}
            La fórmula $(M \vee T) \implies (U \vee D)$ puede reescribirse de la siguiente manera:w

            \begin{align*}
                (M \vee T) \implies (U \vee D) & \equiv \\
                & \equiv \neg (M \vee T) \vee (U \vee D) & \tag{\text{implicación material}} \\
                & \equiv (\neg M \wedge \neg T) \vee (U \vee D) & \tag{\text{de Morgan}}
            \end{align*}
            lo cual se lee como `números que no sean mayores a 5 ni que terminen en 5, o bien que contengan un 1 o un 2'.
            Podemos ver que $3$ está permitido porque no es mayor que 5 ni tampoco termina en cinco, aunque no contiene un $1$ o un $2$.
            $25$, en cambio, contiene un 2 y con eso es suficiente; incluso si termina en 5.
        \end{sol}
	\end{enumerate}
    \end{enumerate}
    
    \vspace{0.5in}
    \noindent Sube un PDF \textit{typeset} (nativamente digital en Word/\LaTeX) con tus respuestas al espacio designado en Canvas.
\end{document}


