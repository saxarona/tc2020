\documentclass[]{book}

%These tell TeX which packages to use.
\usepackage{array,epsfig}
\usepackage{amsmath}
\usepackage{amsfonts}
\usepackage{amssymb}
\usepackage{amsxtra}
\usepackage{amsthm}
\usepackage{mathrsfs}
\usepackage{color}
\usepackage[spanish, mexico]{babel}
\usepackage[utf8]{inputenc}
\usepackage{hyperref}

%Here I define some theorem styles and shortcut commands for symbols I use often
\theoremstyle{definition}
\newtheorem{defn}{Definition}
\newtheorem{thm}{Theorem}
\newtheorem{cor}{Corollary}
\newtheorem*{rmk}{Remark}
\newtheorem{lem}{Lemma}
\newtheorem*{joke}{Joke}
\newtheorem{ex}{Example}
\newtheorem*{sol}{Solution}
\newtheorem{prop}{Proposition}

\newcommand{\lra}{\longrightarrow}
\newcommand{\ra}{\rightarrow}
\newcommand{\surj}{\twoheadrightarrow}
\newcommand{\graph}{\mathrm{graph}}
\newcommand{\bb}[1]{\mathbb{#1}}
\newcommand{\Z}{\bb{Z}}
\newcommand{\Q}{\bb{Q}}
\newcommand{\R}{\bb{R}}
\newcommand{\C}{\bb{C}}
\newcommand{\N}{\bb{N}}
\newcommand{\M}{\mathbf{M}}
\newcommand{\m}{\mathbf{m}}
\newcommand{\MM}{\mathscr{M}}
\newcommand{\HH}{\mathscr{H}}
\newcommand{\Om}{\Omega}
\newcommand{\Ho}{\in\HH(\Om)}
\newcommand{\bd}{\partial}
\newcommand{\del}{\partial}
\newcommand{\bardel}{\overline\partial}
\newcommand{\textdf}[1]{\textbf{\textsf{#1}}\index{#1}}
\newcommand{\img}{\mathrm{img}}
\newcommand{\ip}[2]{\left\langle{#1},{#2}\right\rangle}
\newcommand{\inter}[1]{\mathrm{int}{#1}}
\newcommand{\exter}[1]{\mathrm{ext}{#1}}
\newcommand{\cl}[1]{\mathrm{cl}{#1}}
\newcommand{\ds}{\displaystyle}
\newcommand{\vol}{\mathrm{vol}}
\newcommand{\cnt}{\mathrm{ct}}
\newcommand{\osc}{\mathrm{osc}}
\newcommand{\LL}{\mathbf{L}}
\newcommand{\UU}{\mathbf{U}}
\newcommand{\support}{\mathrm{support}}
\newcommand{\AND}{\;\wedge\;}
\newcommand{\OR}{\;\vee\;}
\newcommand{\Oset}{\varnothing}
\newcommand{\st}{\ni}
\newcommand{\wh}{\widehat}

%Pagination stuff.
\setlength{\topmargin}{-.3 in}
\setlength{\oddsidemargin}{0in}
\setlength{\evensidemargin}{0in}
\setlength{\textheight}{9.in}
\setlength{\textwidth}{6.5in}
\setlength{\itemsep}{0.45in}
\pagestyle{empty}



\begin{document}

\begin{center}
{\huge Matemáticas Computacionales TC2020}\\[1.5ex]
{\large \textbf{Tarea 05 -- Soluciones}\\[1.5ex] %You should put your name here
21.09.18} %You should write the date here.
\end{center}

\vspace{0.2 cm}

\subsection*{Expresiones Regulares}

\textbf{Esta tarea es una extensión de la Tarea 4. Es decir, se realizará con el mismo equipo pero en un entregable distinto}.
\vspace{2ex}

\subsubsection*{ERs canónicas}

\begin{enumerate}
    \item Este enunciado fue ambiguo, por lo que o $11(1010)*00$ o $11((1+0)(1+0))*00$ son ERs válidas.
    \item $a* + a*ba* + a*bb*$
    \item $11(0+1)*00 + 00(0+1)*11$
\end{enumerate}
\subsubsection*{ERs en formato PCRE}

\begin{enumerate}
    \item Este enunciado fue ambiguo, por lo que o $\hat{}11(1010)*00\$$ o $\hat{}11((1|0)(1|0))*00\$$ son ERs válidas. Sus equivalentes son $\hat{}11((10)\{2\})*00\$$ y $\hat{}11([01]\{2\})*00\$$
    \item $\hat{}a* | a*ba* | a*bb*\$$, $\hat{}a\{0,\} | a\{0,\}ba\{0,\} | a\{0,\}bb\{0,\}\$$
    \item $\hat{}11(0|1)*00 | 00(0|1)*11\$$, $\hat{}11[01]*00\$$
\end{enumerate}

\end{document}