\documentclass[]{book}

%These tell TeX which packages to use.
\usepackage{array,epsfig}
\usepackage{amsmath}
\usepackage{amsfonts}
\usepackage{amssymb}
\usepackage{amsxtra}
\usepackage{amsthm}
\usepackage{mathrsfs}
\usepackage{color}
\usepackage[spanish, mexico]{babel}
\usepackage[utf8]{inputenc}

%Here I define some theorem styles and shortcut commands for symbols I use often
\theoremstyle{definition}
\newtheorem{defn}{Definition}
\newtheorem{thm}{Theorem}
\newtheorem{cor}{Corollary}
\newtheorem*{rmk}{Remark}
\newtheorem{lem}{Lemma}
\newtheorem*{joke}{Joke}
\newtheorem{ex}{Example}
\newtheorem*{sol}{Solution}
\newtheorem{prop}{Proposition}

\newcommand{\lra}{\longrightarrow}
\newcommand{\ra}{\rightarrow}
\newcommand{\surj}{\twoheadrightarrow}
\newcommand{\graph}{\mathrm{graph}}
\newcommand{\bb}[1]{\mathbb{#1}}
\newcommand{\Z}{\bb{Z}}
\newcommand{\Q}{\bb{Q}}
\newcommand{\R}{\bb{R}}
\newcommand{\C}{\bb{C}}
\newcommand{\N}{\bb{N}}
\newcommand{\M}{\mathbf{M}}
\newcommand{\m}{\mathbf{m}}
\newcommand{\MM}{\mathscr{M}}
\newcommand{\HH}{\mathscr{H}}
\newcommand{\Om}{\Omega}
\newcommand{\Ho}{\in\HH(\Om)}
\newcommand{\bd}{\partial}
\newcommand{\del}{\partial}
\newcommand{\bardel}{\overline\partial}
\newcommand{\textdf}[1]{\textbf{\textsf{#1}}\index{#1}}
\newcommand{\img}{\mathrm{img}}
\newcommand{\ip}[2]{\left\langle{#1},{#2}\right\rangle}
\newcommand{\inter}[1]{\mathrm{int}{#1}}
\newcommand{\exter}[1]{\mathrm{ext}{#1}}
\newcommand{\cl}[1]{\mathrm{cl}{#1}}
\newcommand{\ds}{\displaystyle}
\newcommand{\vol}{\mathrm{vol}}
\newcommand{\cnt}{\mathrm{ct}}
\newcommand{\osc}{\mathrm{osc}}
\newcommand{\LL}{\mathbf{L}}
\newcommand{\UU}{\mathbf{U}}
\newcommand{\support}{\mathrm{support}}
\newcommand{\AND}{\;\wedge\;}
\newcommand{\OR}{\;\vee\;}
\newcommand{\Oset}{\varnothing}
\newcommand{\st}{\ni}
\newcommand{\wh}{\widehat}

%Pagination stuff.
\setlength{\topmargin}{-.3 in}
\setlength{\oddsidemargin}{0in}
\setlength{\evensidemargin}{0in}
\setlength{\textheight}{9.in}
\setlength{\textwidth}{6.5in}
\setlength{\itemsep}{0.45in}
\pagestyle{empty}



\begin{document}

\begin{center}
{\huge Matemáticas Computacionales TC2020 -- 01}\\[1.5ex]
{\large \textbf{Tarea 02 -- Soluciones}\\[1.5ex] %You should put your name here
20.08.18} %You should write the date here.
\end{center}

\vspace{0.2 cm}

\subsection*{Preliminares: Lógica proposicional}

\begin{enumerate}
	\itemsep0.35in

	\item Convierte los siguientes enunciados en otros que digan lo mismo, pero sin implicaciones ni negaciones.
	Primero define el vocabulario proposicional que utilizarás. Después, usando equivalencias lógicas ve eliminando primero implicaciones y luego negaciones.
	Finalmente, escribe la nueva proposición equivalente.
	\begin{enumerate}
        \item Conjunto de números naturales que si contienen al menos un cero, entonces terminan en 1
        \begin{align*}
            n \in \mathbb{N} & = n \text{ es un número natural}\\
            P & = \text{Contienen al menos un 0}\\
            Q & = \text{Terminan en 1}\\[2ex]
            & P \implies Q\\
            & \neg P \vee Q\\[2ex]
            R & = \neg P = \text{No contienen al menos un 0}\\
            A & = \{n : n \in \mathbb{N}, R \vee Q\} \quad \blacksquare
        \end{align*}
        
        \item Conjunto de palabras formadas con $a$s y $b$s que contienen $a$s si y sólo si contienen $b$s.
        \begin{align*}
            a^+ b^+ & = \text{palabras formadas con } a \text{s y } b \text{s}\\
            P & = \text{Contienen } a \text{s}\\
            Q & = \text{Contienen } b \text{s}\\[2ex]
            & P \iff Q\\
            & P \implies Q \wedge Q \implies P\\
            & \neg P \vee Q \wedge \neg Q \vee P\\[2ex]
            R & = \neg P\\
            S & = \neg Q\\[2ex]
            L & = \{a^+ b^+ : R \vee Q, S \vee P\} \quad \blacksquare
        \end{align*}
	\end{enumerate}
	\item Considera el conjunto de números naturales tales que si son mayores que 5 o terminan en 5, entonces contienen algún 1 o 2.
	\begin{enumerate}
        \item Especifica tres números que cumplan la condición y tres que no la cumplan.
        
        Cumplen el 1, 2 y 3. El 30, 40 y 60 no cumplen.

        \item Expresa el enunciado como una fórmula proposicional donde $M$ significa \textit{mayor que 5}, $T$ significa que \textit{terminan en 5}, $U$ significa que \textit{contienen algún 1} y $D$ significa que \textit{contienen algún 2}.

        \[ M \vee T \implies U \vee D\]

        \item Transforma la fórmula del inciso anterior de forma que se eliminen implicaciones y negaciones, y posteriormente escríbela con palabras.

        \begin{align*}
            & M \vee T \implies U \vee D\\
            & \neg (M \vee T) \vee U \vee D\\
            & \neg M \wedge \neg T \vee U \vee D\\[2ex]
            & P = \neg M\\
            & Q = \neg T\\[2ex]
            & A = \{n : n \in \mathbb{N}, P \wedge Q \vee U \vee D\}
        \end{align*}

        Conjunto de números naturales tal que o contienen un uno, o contienen un dos, o son menores que cinco y no terminan en cinco. {\tiny Aunque todos los números naturales menores que cinco no terminan en cinco, así que...} $\blacksquare$
	\end{enumerate}
	\end{enumerate}
\end{document}


