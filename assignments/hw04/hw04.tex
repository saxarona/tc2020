\documentclass[]{book}

%These tell TeX which packages to use.
\usepackage{array,epsfig}
\usepackage{amsmath}
\usepackage{amsfonts}
\usepackage{amssymb}
\usepackage{amsxtra}
\usepackage{amsthm}
\usepackage{mathrsfs}
\usepackage{color}
\usepackage[spanish, mexico]{babel}
\usepackage[utf8]{inputenc}
\usepackage{hyperref}

%Here I define some theorem styles and shortcut commands for symbols I use often
\theoremstyle{definition}
\newtheorem{defn}{Definition}
\newtheorem{thm}{Theorem}
\newtheorem{cor}{Corollary}
\newtheorem*{rmk}{Remark}
\newtheorem{lem}{Lemma}
\newtheorem*{joke}{Joke}
\newtheorem{ex}{Example}
\newtheorem*{sol}{Solution}
\newtheorem{prop}{Proposition}

\newcommand{\lra}{\longrightarrow}
\newcommand{\ra}{\rightarrow}
\newcommand{\surj}{\twoheadrightarrow}
\newcommand{\graph}{\mathrm{graph}}
\newcommand{\bb}[1]{\mathbb{#1}}
\newcommand{\Z}{\bb{Z}}
\newcommand{\Q}{\bb{Q}}
\newcommand{\R}{\bb{R}}
\newcommand{\C}{\bb{C}}
\newcommand{\N}{\bb{N}}
\newcommand{\M}{\mathbf{M}}
\newcommand{\m}{\mathbf{m}}
\newcommand{\MM}{\mathscr{M}}
\newcommand{\HH}{\mathscr{H}}
\newcommand{\Om}{\Omega}
\newcommand{\Ho}{\in\HH(\Om)}
\newcommand{\bd}{\partial}
\newcommand{\del}{\partial}
\newcommand{\bardel}{\overline\partial}
\newcommand{\textdf}[1]{\textbf{\textsf{#1}}\index{#1}}
\newcommand{\img}{\mathrm{img}}
\newcommand{\ip}[2]{\left\langle{#1},{#2}\right\rangle}
\newcommand{\inter}[1]{\mathrm{int}{#1}}
\newcommand{\exter}[1]{\mathrm{ext}{#1}}
\newcommand{\cl}[1]{\mathrm{cl}{#1}}
\newcommand{\ds}{\displaystyle}
\newcommand{\vol}{\mathrm{vol}}
\newcommand{\cnt}{\mathrm{ct}}
\newcommand{\osc}{\mathrm{osc}}
\newcommand{\LL}{\mathbf{L}}
\newcommand{\UU}{\mathbf{U}}
\newcommand{\support}{\mathrm{support}}
\newcommand{\AND}{\;\wedge\;}
\newcommand{\OR}{\;\vee\;}
\newcommand{\Oset}{\varnothing}
\newcommand{\st}{\ni}
\newcommand{\wh}{\widehat}

%Pagination stuff.
\setlength{\topmargin}{-.3 in}
\setlength{\oddsidemargin}{0in}
\setlength{\evensidemargin}{0in}
\setlength{\textheight}{9.in}
\setlength{\textwidth}{6.5in}
\setlength{\itemsep}{0.45in}
\pagestyle{empty}



\begin{document}

\begin{center}
{\huge Matemáticas Computacionales TC2020}\\[1.5ex]
{\large \textbf{Tarea 04}\\[1.5ex] %You should put your name here
21.09.18} %You should write the date here.
\end{center}

\vspace{0.2 cm}

\subsection*{Autómatas Finitos No Deterministas}

\textbf{Esta tarea es en parejas}.
\vspace{2ex}

Con ayuda de JFLAP (\url{http://www.jflap.org/}), generen \textdf{un AFN correcto y completo} para cada uno de los lenguajes especificados.

\begin{enumerate}
	\itemsep3.5ex
    \item El lenguaje de las palabras en $\{0,1\}^*$ que inician exactamente con $11$, terminan exactamente con $00$ y en medio tienen una subcadena de unos y ceros de longitud par. (Tip: El 0 es un número par.)
        \begin{itemize}
            \item Se aceptan: $1100, 110100, 1101010100, \dots$
            \item No se aceptan: $\varepsilon, 1, 0, 11, 00, 1101, \dots$
        \end{itemize}
    
    \item El lenguaje de las palabras en $\{a,b\}^*$ que si contienen dos o más $b$s, entonces todas las $b$s de la palabra aparecen al final.
        \begin{itemize}
            \item Se aceptan: $\varepsilon, a,b, aa, bb, aaa, baa, aaabbb, \dots$
            \item No se aceptan: $abba, abbabaa, aaabbab, \dots$
        \end{itemize}
    
    \item El lenguaje de las palabras en $\{0,1\}^*$ que inician con $11$ y terminan en $00$, o inician en $00$ y terminan en $11$.
        \begin{itemize}
            \item Se aceptan: $1100, 0011, 11000, 001011, \dots$
            \item No se aceptan: $\varepsilon, 0, 1, 01, 10, 1110, 0001, \dots$
        \end{itemize}
\end{enumerate}

Al finalizar, enumeren y guarden cada uno de los AFNs generados (\texttt{.jff}), y súbanlos en una carpeta comprimida (\texttt{.zip}) con el nombre $M_1$-$M_2$\texttt{-tarea03.zip}, donde $M_i$ es su número de matrícula.
Con que uno de los integrantes suba el archivo es suficiente, no hay necesidad de que cada uno suba una copia.

Es muy importante que incluyan el \texttt{.jff}, o de lo contrario los autómatas no serán calificados.

\end{document}