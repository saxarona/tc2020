\documentclass[]{book}

%These tell TeX which packages to use.
\usepackage{array,epsfig}
\usepackage{amsmath}
\usepackage{amsfonts}
\usepackage{amssymb}
\usepackage{amsxtra}
\usepackage{amsthm}
\usepackage{mathrsfs}
\usepackage{color}
\usepackage[spanish, mexico]{babel}
\usepackage[utf8]{inputenc}
\usepackage{hyperref}

%Here I define some theorem styles and shortcut commands for symbols I use often
\theoremstyle{definition}
\newtheorem{defn}{Definition}
\newtheorem{thm}{Theorem}
\newtheorem{cor}{Corollary}
\newtheorem*{rmk}{Remark}
\newtheorem{lem}{Lemma}
\newtheorem*{joke}{Joke}
\newtheorem{ex}{Example}
\newtheorem*{sol}{Solution}
\newtheorem{prop}{Proposition}

\newcommand{\lra}{\longrightarrow}
\newcommand{\ra}{\rightarrow}
\newcommand{\surj}{\twoheadrightarrow}
\newcommand{\graph}{\mathrm{graph}}
\newcommand{\bb}[1]{\mathbb{#1}}
\newcommand{\Z}{\bb{Z}}
\newcommand{\Q}{\bb{Q}}
\newcommand{\R}{\bb{R}}
\newcommand{\C}{\bb{C}}
\newcommand{\N}{\bb{N}}
\newcommand{\M}{\mathbf{M}}
\newcommand{\m}{\mathbf{m}}
\newcommand{\MM}{\mathscr{M}}
\newcommand{\HH}{\mathscr{H}}
\newcommand{\Om}{\Omega}
\newcommand{\Ho}{\in\HH(\Om)}
\newcommand{\bd}{\partial}
\newcommand{\del}{\partial}
\newcommand{\bardel}{\overline\partial}
\newcommand{\textdf}[1]{\textbf{\textsf{#1}}\index{#1}}
\newcommand{\img}{\mathrm{img}}
\newcommand{\ip}[2]{\left\langle{#1},{#2}\right\rangle}
\newcommand{\inter}[1]{\mathrm{int}{#1}}
\newcommand{\exter}[1]{\mathrm{ext}{#1}}
\newcommand{\cl}[1]{\mathrm{cl}{#1}}
\newcommand{\ds}{\displaystyle}
\newcommand{\vol}{\mathrm{vol}}
\newcommand{\cnt}{\mathrm{ct}}
\newcommand{\osc}{\mathrm{osc}}
\newcommand{\LL}{\mathbf{L}}
\newcommand{\UU}{\mathbf{U}}
\newcommand{\support}{\mathrm{support}}
\newcommand{\AND}{\;\wedge\;}
\newcommand{\OR}{\;\vee\;}
\newcommand{\Oset}{\varnothing}
\newcommand{\st}{\ni}
\newcommand{\wh}{\widehat}

%Pagination stuff.
\setlength{\topmargin}{-.3 in}
\setlength{\oddsidemargin}{0in}
\setlength{\evensidemargin}{0in}
\setlength{\textheight}{9.in}
\setlength{\textwidth}{6.5in}
\setlength{\itemsep}{0.45in}
\pagestyle{empty}



\begin{document}

\begin{center}
{\huge Matemáticas Computacionales TC2020}\\[1.5ex]
{\large \textbf{Tarea 03}\\[1.5ex] %You should put your name here
31.08.18} %You should write the date here.
\end{center}

\vspace{0.2 cm}

\subsection*{Autómatas Finitos Deterministas}

\textbf{Esta tarea es en parejas}.
\vspace{2ex}

Con ayuda de JFLAP (\url{http://www.jflap.org/}), generen \textdf{un AFD correcto y completo} para cada uno de los lenguajes especificados.

\begin{enumerate}
	\itemsep3.5ex
	\item El lenguaje de las palabras en $\{a,b\}^*$ que contienen exactamente 2 $a$s.
	\begin{itemize}
		\item Se aceptan: $aa, baa, aba, bababb, \dots$
		\item No se aceptan: $\varepsilon , a, b, aaa, bbbb, \dots$
	\end{itemize}
	
	\item El lenguaje de los números binarios que si tienen un solo dígito $0$, debe ser el último dígito del número. (La palabra vacía no es un número binario).
	\begin{itemize}
		\item Se aceptan: $0, 10, 11, 001, 0101, 1110, 1000, \dots$
		\item No se aceptan: $\varepsilon , 01, 1101, 110111, 011, \dots$
	\end{itemize}

	\item El lenguaje de las palabras en $\{a, b\}^*$ que contienen la subcadena $aa$ y contienen la subcadena $bb$.
	\begin{itemize}
		\item Se aceptan: $aabb, bbbaa, aababb, abaababba, \dots$
		\item No se aceptan: $\varepsilon , a, b, abb, aab, ababa, \dots$
	\end{itemize}

	\item El lenguaje de las palabras en $\{0,1 \}^*$ que contienen un número par de $0$s o un número impar de $1$s.
	\begin{itemize}
		\item Se aceptan: $\varepsilon , 1, 11, 00, 1011, 10101, 11100, \dots$
		\item No se aceptan: $0, 011, 11011, 01010, \dots$
	\end{itemize}

	\item El lenguaje de las palabras en $\{a, b\}^*$ que si terminan con $b$ entonces tienen una sola $b$, y si empieza con $a$ entonces tiene al menos otra $a$.
	\begin{itemize}
		\item Se aceptan: $\varepsilon , b, aab, aa, ba, bba, ababa, \dots$
		\item No se aceptan: $a, ab, bb, abb, baaab, \dots$
	\end{itemize}
\end{enumerate}

Al finalizar, enumeren y guarden cada uno de los AFDs generados (\texttt{.jff}), y súbanlos en una carpeta comprimida (\texttt{.zip}) con el nombre $M_1$-$M_2$\texttt{-tarea03.zip}, donde $M_i$ es su número de matrícula.
Con que uno de los integrantes suba el archivo es suficiente, no hay necesidad de que cada uno suba una copia.

Es muy importante que incluyan el \texttt{.jff}, o de lo contrario los autómatas no serán calificados.

\end{document}