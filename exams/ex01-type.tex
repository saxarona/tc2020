\documentclass[8pt, onside]{article}
%\usepackage[a4paper]{geometry}
\usepackage{fullpage}
\usepackage[utf8]{inputenc}
\usepackage[spanish, mexico]{babel}
\usepackage{lipsum}
\usepackage{bm}
\usepackage{upgreek}
\usepackage{enumitem}
\usepackage{mathrsfs}
\usepackage{amsmath}
\usepackage{amssymb}
\usepackage{tikz}
\usetikzlibrary{arrows, automata}

\tikzset{
    ->, %arrow type
    >=stealth', %arrow head type (bold)
    shorten >=1pt, 
    auto,
    semithick,
    initial text=$ $, %no start text
    }


% mathtools for: Aboxed (put box on last equation in align envirenment)
\usepackage{microtype} %improves the spacing between words and letters

%% COLOR DEFINITIONS

\usepackage{xcolor} % Enabling mixing colors and color's call by 'svgnames'

\definecolor{MyColor1}{rgb}{0.2,0.4,0.6} %mix personal color
\newcommand{\textb}{\color{Black} \usefont{OT1}{lmss}{m}{n}}
\newcommand{\blue}{\color{MyColor1} \usefont{OT1}{lmss}{m}{n}}
\newcommand{\blueb}{\color{MyColor1} \usefont{OT1}{lmss}{b}{n}}
\newcommand{\red}{\color{LightCoral} \usefont{OT1}{lmss}{m}{n}}
\newcommand{\green}{\color{Turquoise} \usefont{OT1}{lmss}{m}{n}}

\DeclareMathOperator{\trace}{trace}
\DeclareMathOperator{\diag}{diag}

%% FONTS AND COLORS

%    SECTIONS

\usepackage{titlesec}
\usepackage{sectsty}
%%%%%%%%%%%%%%%%%%%%%%%%
%set section/subsections HEADINGS font and color
%\sectionfont{\color{Black}}  % sets colour of sections
%\subsectionfont{\color{Black}}  % sets colour of sections

%set section enumerator to arabic number (see footnotes markings alternatives)
\renewcommand\thesection{\arabic{section}} %define sections numbering
\renewcommand\thesubsection{\thesection\arabic{subsection}} %subsec.num.

%define new section style
\newcommand{\mysection}{
\titleformat{\section} [runin] {\usefont{OT1}{lmss}{b}{n}\color{MyColor1}} 
{\thesection} {3pt} {} } 


% %	CAPTIONS
% \usepackage{caption}
% \usepackage{subcaption}
% %%%%%%%%%%%%%%%%%%%%%%%%
% \captionsetup[figure]{labelfont={color=Turquoise}}


%		!!!EQUATION (ARRAY) --> USING ALIGN INSTEAD
%using amsmath package to redefine eq. numeration (1.1, 1.2, ...) 
\renewcommand{\theequation}{\thesection\arabic{equation}}

\setlength\parindent{0pt}




\makeatletter
\let\reftagform@=\tagform@
\def\tagform@#1{\maketag@@@{(\ignorespaces\textcolor{red}{#1}\unskip\@@italiccorr)}}
\renewcommand{\eqref}[1]{\textup{\reftagform@{\ref{#1}}}}
\makeatother
\usepackage{hyperref}
\hypersetup{colorlinks=true}

% For labeling top of page on every page but first one:
%\usepackage{fancyhdr}

\newcommand{\myclass}{TC2020 -- Matemáticas Computacionales} % Class name?
\newcommand{\mytitle}{Ejercicios de Repaso} % Title of document?
\newcommand{\mydate}{10.09.18} % The date?
\newcommand{\myheader}{
    \begin{flushleft}
        \large
        Nombre: \rule{13 cm}{0.4mm} \\
        Matrícula: \rule{5 cm}{0.4mm} \hfill Fecha: \mydate
    \end{flushleft}
}

\title{
    \myclass \\
    \textbf{\mytitle} \\
    \myheader
    \date{}
}

% You can set the date automatically by replacing "date goes here" with "\today"

% \renewcommand{\rmdefault}{phv} % Arial Font
% \renewcommand{\sfdefault}{phv} % Arial Font

% \pagestyle{fancy}
% \fancyhead{}
% \fancyhead[CO,CE]{{\small{{\bf{\mytitle}} -- \myclass}}}

\begin{document}
\maketitle

\vspace{-1.5cm}

Estos ejercicios de repaso sirvieron alguna vez como problemas de examen.
Lee cuidadosamente y contesta lo que se te pide.
El examen fue pensado para resolverse de manera individual y en poco menos de 90 minutos.

\section{Relaciones y funciones (15\%)}

Sean $A = \{1,2,3,4,5\}$ y $B = \{1,2,3,4,5\}$.
Para las siguientes relaciones, indica si son \textbf{reflexivas}, \textbf{simétricas} o \textbf{transitivas}.
Menciona también si son \textbf{funciones}. En caso de que lo sean, indica si son \textbf{totales} o \textbf{parciales}, y si son \textbf{inyecciones}, \textbf{sobreyecciones} o \textbf{biyecciones}.

\begin{enumerate}[label=\tt \alph*)]
    \itemsep0em
    \item $\{(1,1), (2,2), (3,3), (4,4)\}$
    \item $\{(2,2), (1,1), (3,3), (5,5), (4,4)\}$
    \item $\{(1,2), (2,1), (3,4), (4,3), (3,5), (5,3)\}$
    \item $\{(1,5), (2,3), (3,2), (4,4), (5, 4)\}$
    \item $A \times B$
\end{enumerate}


\section{Operaciones con conjuntos (15 \%)}

Calcula el resultado de las siguientes operaciones.

\begin{enumerate}[label=\tt \alph*)]
    \itemsep0em
    \item $\{1, 2, 3\} \cup \{z : z \in \mathbb{Z}, 4 \leq z < 10 \}$
    \item $\{1\} \times \{2,3,4\}$
    \item $|\mathscr{P}(\{n : n \in \mathbb{N} \cup \{0\}, n < 15\})|$
    \item $\{10, 20, 30\} \cap \{r : r \in \mathbb{N}\}$
    \item $\{1, 2, 3\} \cap \{4, 5, 6\}$
\end{enumerate}


\section{Diseño de AFDs (15 \%)}

Diseña un AFD correcto y completo para el lenguaje de las palabras en $\{a,b\}$ que contienen la subcadena \texttt{abba} (10 \%) y escribe su definición formal completa (2 \%).
Después, da tres ejemplos de palabras que sean aceptadas y otros tres de palabras no aceptadas (3 \%).


\section{Minimización de AFDs (25 \%)}

A continuación encontrarás el diagrama de un AFD.
Escribe su definición formal completa (5 \%) y simplifícalo utilizando el método que prefieras (20 \%).
Intenta seguir un proceso ordenado y claro.
Si algún movimiento te parece muy ``lógico'', puedes explicarlo a grandes rasgos.

\begin{center}
    \begin{tikzpicture}[node distance=1.8cm]
        \node[initial, state] (s0) {$A$};
        \node[state, right of=s0] (s1) {$B$};
        \node[state, below of=s0] (s4) {$C$};
        \node[state, right of=s4] (s5) {$D$};
        \node[state, accepting, right of=s1] (s2) {$E$};
        \node[state, right of=s2] (s3) {$F$};
        \node[state, below of=s2] (s6) {$G$};
        \node[state, accepting, right of=s6] (s7) {$H$};
    
        \path
        (s0) edge node {$0$} (s1)
        (s0) edge node [left] {$1$} (s4)
        (s1) edge [bend left] node {$0$} (s5)
        (s1) edge node {$1$} (s2)
        (s2) edge node {$0$} (s3)
        (s2) edge node {$1$} (s6)
        (s3) edge [loop above] node {$0,1$} (s3)
        (s4) edge node {$0$} (s1)
        (s4) edge [loop below] node {$1$} (s4)
        (s5) edge [bend left] node [right] {$0$} (s1)
        (s5) edge node {$1$} (s4)
        (s6) edge node {$0$} (s3)
        (s6) edge [bend left] node [below] {$1$} (s7)
        (s7) edge node [right] {$0$} (s3)
        (s7) edge [bend left] node {$1$} (s6);
    \end{tikzpicture}
\end{center}

\section{Diseño de Autómatas Finitos (30 \%)}

Genera un autómata finito por \textbf{unión} que acepte el lenguaje de las palabras tengan 2 $a$s seguidas o que si empiezan con $b$ terminen con $a$ (10 \%).
Posteriormente, minimízalo y preséntalo como un autómata finito determinista (10 \%) con su descripción formal (10 \%).


% \section{Diseño de Autómatas Finitos (30 \% + 7 \%)}

% El \textit{Biologic Space Lab} que orbita \textsc{SR388} sufrió una falla eléctrica dejando fuertes daños en algunos de sus sectores. Uno de los más afectados fue el que alberga muestras de la vida marina del planeta, el Sector 4 (AQA). Por ello, el equipo de mantenimiento te ha asignado la tarea de implementar un componente de control para el nivel del agua de los 32 tanques en el sector mientras se llevan a cabo las reparaciones.

% Cada tanque tiene tres sensores: uno para la detección de movimiento (y así saber si en el tanque se encuentra algún espécimen o si el tanque se encuentra vacío), otro para medir la toxicidad del agua, y uno más para medir la cantidad de carga eléctrica, la cual es indicio de alguna falla. En el caso de los últimos dos, se marca como `peligroso' cuando se sobrepasa un umbral.

% Los tanques envían actualizaciones al mecanismo de control una vez cada dos horas por medio de un packet de 8 bits, donde los primeros 3 corresponden a la lectura de los sensores (movimiento, toxicidad y carga, en ese orden), y los últimos 5 al ID del tanque (en binario).

% HQ te ha notificado que la compuerta que alimenta a cada tanque debe cerrarse en cualquiera de las siguientes situaciones:

% \begin{itemize}
%     \itemsep0em
%     \item Si las especies del tanque están a salvo (baja toxicidad y baja carga eléctrica).
%     \item Si el tanque está deshabitado, pero es altamente tóxico, sin importar lo demás.
%     \item Si el agua del tanque presenta una cantidad peligrosa de carga eléctrica, independientemente de que esté habitado o de su toxicidad.
% \end{itemize}

% Diseña un autómata que funcione como mecanismo de control para cerrar la compuerta de cada tanque si el packet completo de información que se recibe es aceptado. Sugerencias:

% \begin{enumerate}[label=\tt \alph*)]
%     \itemsep0em
%     \item Genera el alfabeto de tu autómata (2 \%)
%     \item Convierte las condiciones de aceptación a palabras de entrada específicas (3 \%)
%     \item Diseña cuantos componentes sean necesarios (15 \%)
%     \item Presenta el mecanismo de control completo (5 \%)
%     \item Describe formalmente tu autómata (5 \%)
%     \item Ponle un nombre a tu mecanismo de control (+2 \%)
%     \item Si te es posible, minimízalo (+5 \%)
% \end{enumerate}

% \section*{Reto: Conversión de AFN a AFDs (+15 \%)}

% Diseña un AFN por \textbf{concatenación} para el lenguaje de las palabras $\{a^n b^m\}$ donde $n$ y $m$ describen las veces que se repite cada carácter.
% Las palabras tienen $n \,$ $a$s, seguidas por $m \,$ $b$s para cualquier valor de $n$ y $m$.

% \begin{itemize}
%     \item Palabras aceptadas: $\varepsilon, a, b, ab, aab, aabb, \dots$
%     \item Palabras no aceptadas: $ba, bba, bbbba, abba, aaba, \dots$
% \end{itemize}

% Sugerencia: minimiza lo más que puedas tu AFN para que no batalles al convertirlo.

\vfill

\textbf{De acuerdo con el Código de Ética del Tecnológico de Monterrey, mi desempeño en esta actividad estará guiado por la integridad académica.}
\end{document}