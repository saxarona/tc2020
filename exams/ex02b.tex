\documentclass[8pt, onside]{article}
%\usepackage[a4paper]{geometry}
\usepackage{fullpage}
\usepackage[utf8]{inputenc}
\usepackage[spanish, mexico]{babel}
\usepackage{lipsum}
\usepackage{bm}
\usepackage{upgreek}
\usepackage{enumitem}
\usepackage{mathrsfs}
\usepackage{amsmath}
\usepackage{amssymb}
\usepackage{tikz}
\usepackage{tcolorbox}
\usepackage{csquotes}
\usetikzlibrary{arrows, automata}

% mathtools for: Aboxed (put box on last equation in align envirenment)
\usepackage{microtype} %improves the spacing between words and letters

%% COLOR DEFINITIONS

\usepackage{xcolor} % Enabling mixing colors and color's call by 'svgnames'

\definecolor{MyColor1}{rgb}{0.2,0.4,0.6} %mix personal color
\newcommand{\textb}{\color{Black} \usefont{OT1}{lmss}{m}{n}}
\newcommand{\blue}{\color{MyColor1} \usefont{OT1}{lmss}{m}{n}}
\newcommand{\blueb}{\color{MyColor1} \usefont{OT1}{lmss}{b}{n}}
\newcommand{\red}{\color{LightCoral} \usefont{OT1}{lmss}{m}{n}}
\newcommand{\green}{\color{Turquoise} \usefont{OT1}{lmss}{m}{n}}

\DeclareMathOperator{\trace}{trace}
\DeclareMathOperator{\diag}{diag}

%% FONTS AND COLORS

%    SECTIONS

\usepackage{titlesec}
\usepackage{sectsty}
%%%%%%%%%%%%%%%%%%%%%%%%
%set section/subsections HEADINGS font and color
%\sectionfont{\color{Black}}  % sets colour of sections
%\subsectionfont{\color{Black}}  % sets colour of sections

%set section enumerator to arabic number (see footnotes markings alternatives)
\renewcommand\thesection{\arabic{section}} %define sections numbering
\renewcommand\thesubsection{\thesection\arabic{subsection}} %subsec.num.

%define new section style
\newcommand{\mysection}{
\titleformat{\section} [runin] {\usefont{OT1}{lmss}{b}{n}\color{MyColor1}} 
{\thesection} {3pt} {} } 


% %	CAPTIONS
% \usepackage{caption}
% \usepackage{subcaption}
% %%%%%%%%%%%%%%%%%%%%%%%%
% \captionsetup[figure]{labelfont={color=Turquoise}}


%		!!!EQUATION (ARRAY) --> USING ALIGN INSTEAD
%using amsmath package to redefine eq. numeration (1.1, 1.2, ...) 
\renewcommand{\theequation}{\thesection\arabic{equation}}

\setlength\parindent{0pt}




\makeatletter
\let\reftagform@=\tagform@
\def\tagform@#1{\maketag@@@{(\ignorespaces\textcolor{red}{#1}\unskip\@@italiccorr)}}
\renewcommand{\eqref}[1]{\textup{\reftagform@{\ref{#1}}}}
\makeatother
\usepackage{hyperref}
\hypersetup{colorlinks=true}

% For labeling top of page on every page but first one:
%\usepackage{fancyhdr}

\newcommand{\myclass}{TC2020 -- Matemáticas Computacionales} % Class name?
\newcommand{\mytitle}{Examen 2} % Title of document?
\newcommand{\mydate}{22.10.18} % The date?
\newcommand{\myheader}{
    \begin{flushleft}
        \large
        Nombre: \rule{10 cm}{0.4mm} \quad \rule{2.5 cm}{0.4mm} \\
        Nombre: \rule{10 cm}{0.4mm} \quad \rule{2.5 cm}{0.4mm} \\
        Nombre: \rule{10 cm}{0.4mm} \quad \rule{2.5 cm}{0.4mm} \\
    \end{flushleft}
}

\title{
    \myclass \\
    \textbf{\mytitle} \\
    \myheader
    \date{}
}

% You can set the date automatically by replacing "date goes here" with "\today"

% \renewcommand{\rmdefault}{phv} % Arial Font
\renewcommand{\familydefault}{phv} % Arial Font

% \pagestyle{fancy}
% \fancyhead{}
% \fancyhead[CO,CE]{{\small{{\bf{\mytitle}} -- \myclass}}}

\begin{document}
\maketitle

\vspace{-1.5cm}

Este examen está pensado para resolverse con el nuevo equipo (\textit{Season 2}).
Revisen con calma lo que se pide. Es bastante trabajo, por lo que se sugiere que consideren bien sus tiempos.
Al momento de contestar, intenten ser lo más explícitos posible: se calificará con base en lo que esté escrito, y se considerará el proceso aún cuando la respuesta final esté errada.
Recuerden que pueden revisar material de la clase, el libro de texto o sus notas.
Buena suerte.

\section{Expresiones Regulares (36\%)}

Describan cada uno de los siguientes lenguajes utilizando \textbf{expresiones regulares} genéricas o PCRE; \textit{your choice}. En dado caso de que el lenguaje no se pueda describir con una ER, demuestren el porqué. Posteriormente, e independientemente de si tiene ER o no, escriban dos ejemplos de palabras aceptadas para cada lenguaje.

\begin{enumerate}[label=\tt \alph*)]
    \itemsep0em
    \item El lenguaje de las palabras en $\{a,b\}$ que empiezan con $a$ y terminan con $b$.
    \item El lenguaje de las palabras en $\{0,1\}$ que son múltiplos de $100$.
    \item El lenguaje de las palabras en $\{x, y\}$ que son de longitud impar o que terminan en $xxx$.
    \item El lenguaje de las palabras en $\{0,1,2\}$ que tengan un $1$ en el centro, y sean simétricas.
    \item El lenguaje de las palabras en $\{z : z \in \mathbb{Z}, z \geq 0\}$ que son también denominaciones de billetes o monedas expedidas por el Banco de México y que están en circulación. Omitan centavos y series especiales (Onzas de plata, Centenarios, etc.).
    \item El lenguaje de las palabras en $\{a,b\}$ que tengan un número par de $b$s. (\textit{Hint: consideren hacer un autómata y luego convertir si les parece muy complicado.})
\end{enumerate}


\section{Gramáticas Regulares (14 \% + 4 \%)}

En \texttt{HTML}, le llamamos ``etiquetas'' a las \textit{keywords} para abrir o cerrar elementos de la página web.
Una etiqueta siempre se escribe entre \textit{brackets} para iniciar el elemento, y la misma etiqueta pero con una diagonal al principio para finalizar el elemento.
Las etiquetas \verb|<html>...</html>| sirven para delimitar el contenido de la página web.

\begin{enumerate}[label=\tt \alph*)]
    \itemsep0em
    \item Escriban una expresión regular para seleccionar (o sea, que acepte) las \textit{keywords} de inicio y fin de las listas ordenadas (\texttt{ol}) y no ordenadas (\texttt{ul}), incluyendo los brackets como parte de la selección (4 \%)
    \item Conviertan a AF y escriban su definición formal (5 \%)
    \item Conviertan a Gramática \textbf{Regular} (no es GLC) y escriban su definición formal (5 \%)
    \item Escriban el árbol de derivación para dos palabras válidas (+ 4\%)
\end{enumerate}


\section{Ambigüedad en GLCs (15\%)}

Sea $G=(\{S\},\{\mathtt{if}, c, \mathtt{then}, \mathtt{else}, x\}, \{S \to x, S \to \mathtt{if} \, c \, \mathtt{then} \, S, S \to \mathtt{if} \, c \, \mathtt{then} \, S \, \mathtt{else} \, S\}, S)$ una gramática libre de contexto.

\begin{enumerate}[label=\tt \alph*)]
    \item Demuestren que $G$ es una gramática \textbf{ambigua} utilizando dos \textbf{árboles de derivación diferentes} para una misma \textit{string} (10 \%)
    \item Expresen cada árbol de derivación en forma secuencial (usando $\implies$) (5 \%)
\end{enumerate}


\section{Lenguajes Libres de Contexto (35 \% +2 \%)}

\subsection*{Identificando anomalías}

El \textit{Biologic Space Lab} (BSL) está en problemas nuevamente.
Debido a una brecha en un ducto de ventilación, ciertos parásitos lograron entrar a uno de los laboratorios de acceso restringido (Sector 7), en el cual se almacenaban muestras de ADN de especies inteligentes para fines taxonómicos.
Algunos de los bancos de datos están corruptos debido a la infección, pero otros aún están a salvo.
HQ les ha asignado la tarea de diseñar un sistema de control para identificar aquellas muestras de ADN que aún pueden rescatarse, y les ha enviado la siguiente información:

\begin{itemize}
    \itemsep0em
    \item Los parásitos tienen la capacidad de alterar el genoma de las especies, copiando y replicando ciertas subcadenas de ADN mediante mutación.
    \item El parásito replica los genes pertenecientes al número de extremidades inferiores y de globos oculares de manera arbitraria.
    \item Gracias al análisis de los datos, se determinó que los parásitos sólo están afectando a ciertas especies que consideran `aptas y balanceadas`, que son aquellas que tienen el mismo número de ojos que de piernas.
    \item Sin embargo, se ha confirmado que el número de genes replicados pertenecientes a los ojos nunca es igual que el de las extremidades inferiores en el mismo individuo---en un individuo mutado, el número final de ojos nunca es igual al número final de piernas.
\end{itemize}

Además, dada la gran bio-diversidad del BSL, HQ les sugiere lo siguiente:

\begin{tcolorbox}
{\small Tenemos 902,451 muestras en el databank del laboratorio, de las cuáles sólo 487,350 podrían estar infectadas.
Nos interesa saber cuáles de esas muestras posiblemente infectadas están sanas para separarlas cuanto antes.
Como son tantas especies distintas, creemos que es mejor enfocarse en encontrar diferencias en el número de genes en lugar de buscar una secuencia válida.}
\end{tcolorbox}

Diseñen un sistema de reglas para generar subcadenas de genes de ojos y piernas de individuos sanos.

\begin{enumerate}[label=\tt \alph*)]
    \item Generen el alfabeto de terminales (2 \%)
    \item Describan (con notación de conjuntos o con palabras) el lenguaje aceptado por su gramática (5 \%)
    \item Generen las reglas de su gramática (5 \%)
    \item Describan formalmente su gramática (5 \%)
    \item Usando su gramática, sus símbolos y reglas, escriban el árbol de derivación de un \textit{Dachora} sano, que es una especie de ave que tiene dos piernas y dos ojos (8 \%)
    \item Hagan un Autómata \textit{adecuado} para su GLC y descríbanlo formalmente (5 \%)
    \item Dibujen un \textit{Dachora} (+2 \%)
    
\end{enumerate}

\subsection*{Uniendo mecanismos}

HQ quiere tener acceso a las $k$ especies sanas más rápidamente.
La sugerencia de uno de los \textit{interns} de algoritmia del BSL es agregar un prefijo a cada cadena sana y ordenar toda la base de datos para conseguir todas las cadenas de especies sanas juntas y cortar más fácilmente.

Generen alguna mecanismo (autómata o gramática; \textit{your choice}) que incluya el sistema completo de tal manera que logre \textit{reconocer} cadenas sanas con todo y el prefijo de ordenamiento. Sugerencias:

\begin{enumerate}[label=\tt \alph*)]
    \item Propongan un prefijo para ordenamiento (1 \%)
    \item Describan con notación de conjuntos o palabras el lenguaje aceptado por el mecanismo completo (1 \%)
    \item Generen las reglas, variables, estados o transiciones necesarias según sea el caso (1 \%)
    \item Describan el mecanismo completo formalmente (1 \%)
    \item Demuestren que su mecanismo es regular o libre de contexto, según consideren (1 \%)
\end{enumerate}

\section*{Extra Challenge (+8 \%)}

Determinen si llevar a cabo la sugerencia del \textit{intern} es viable o no y expliquen por qué. Sugerencias:

\begin{enumerate}[label=\tt \alph*)]
    \item ¿Cuáles son los valores máximos y mínimos posibles de $k$? (+2 \%)
    \item Asuman que se utilizará un algoritmo \textbf{eficiente} de ordenamiento sobre las $n$ muestras del laboratorio.
    \item ¿Cuál es el valor de $n$? (+1 \%)
    \item ¿Cuántas operaciones tomaría encontrar todas las $k$ especies sanas si las buscas entre las $n$ posibles de una por una? (+2 \%)
    \item ¿Cuántas operaciones tomaría ordenar las $n$ especies? (+2 \%)
    \item ¿Qué es más rápido? ¿La opción \texttt{d} o la opción \texttt{e}? (+1 \%)
\end{enumerate}

\vfill

\textbf{De acuerdo con el Código de Ética del Tecnológico de Monterrey, nuestro desempeño en esta actividad estará guiado por la integridad académica.}
\end{document}