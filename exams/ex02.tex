\documentclass[8pt, onside]{article}
%\usepackage[a4paper]{geometry}
\usepackage{fullpage}
\usepackage[utf8]{inputenc}
\usepackage[spanish, mexico]{babel}
\usepackage{lipsum}
\usepackage{bm}
\usepackage{upgreek}
\usepackage{enumitem}
\usepackage{mathrsfs}
\usepackage{amsmath}
\usepackage{amssymb}
\usepackage{tikz}
\usetikzlibrary{arrows, automata}

\tikzset{
    automaton/.style={
        ->, %arrow type
        >=stealth', %arrow head type (bold)
        shorten >=1pt, 
        auto,
        %semithick,
        initial text=$ $, %no start text
    }
}


% mathtools for: Aboxed (put box on last equation in align envirenment)
\usepackage{microtype} %improves the spacing between words and letters

%% COLOR DEFINITIONS

\usepackage{xcolor} % Enabling mixing colors and color's call by 'svgnames'

\definecolor{MyColor1}{rgb}{0.2,0.4,0.6} %mix personal color
\newcommand{\textb}{\color{Black} \usefont{OT1}{lmss}{m}{n}}
\newcommand{\blue}{\color{MyColor1} \usefont{OT1}{lmss}{m}{n}}
\newcommand{\blueb}{\color{MyColor1} \usefont{OT1}{lmss}{b}{n}}
\newcommand{\red}{\color{LightCoral} \usefont{OT1}{lmss}{m}{n}}
\newcommand{\green}{\color{Turquoise} \usefont{OT1}{lmss}{m}{n}}

\DeclareMathOperator{\trace}{trace}
\DeclareMathOperator{\diag}{diag}

%% FONTS AND COLORS

%    SECTIONS

\usepackage{titlesec}
\usepackage{sectsty}
%%%%%%%%%%%%%%%%%%%%%%%%
%set section/subsections HEADINGS font and color
%\sectionfont{\color{Black}}  % sets colour of sections
%\subsectionfont{\color{Black}}  % sets colour of sections

%set section enumerator to arabic number (see footnotes markings alternatives)
\renewcommand\thesection{\arabic{section}} %define sections numbering
\renewcommand\thesubsection{\thesection\arabic{subsection}} %subsec.num.

%define new section style
\newcommand{\mysection}{
\titleformat{\section} [runin] {\usefont{OT1}{lmss}{b}{n}\color{MyColor1}} 
{\thesection} {3pt} {} } 


% %	CAPTIONS
% \usepackage{caption}
% \usepackage{subcaption}
% %%%%%%%%%%%%%%%%%%%%%%%%
% \captionsetup[figure]{labelfont={color=Turquoise}}


%		!!!EQUATION (ARRAY) --> USING ALIGN INSTEAD
%using amsmath package to redefine eq. numeration (1.1, 1.2, ...) 
\renewcommand{\theequation}{\thesection\arabic{equation}}

\setlength\parindent{0pt}




\makeatletter
\let\reftagform@=\tagform@
\def\tagform@#1{\maketag@@@{(\ignorespaces\textcolor{red}{#1}\unskip\@@italiccorr)}}
\renewcommand{\eqref}[1]{\textup{\reftagform@{\ref{#1}}}}
\makeatother
\usepackage{hyperref}
\hypersetup{colorlinks=true}

% For labeling top of page on every page but first one:
%\usepackage{fancyhdr}

\newcommand{\myclass}{TC2020 -- Matemáticas Computacionales} % Class name?
\newcommand{\mytitle}{Examen 2} % Title of document?
\newcommand{\mydate}{22.10.18} % The date?
\newcommand{\myheader}{
    \begin{flushleft}
        \large
        Nombre: \rule{13 cm}{0.4mm} \\
        Matrícula: \rule{5 cm}{0.4mm} \hfill Fecha: \mydate
    \end{flushleft}
}

\title{
    \myclass \\
    \textbf{\mytitle} \\
    \myheader
    \date{}
}

% You can set the date automatically by replacing "date goes here" with "\today"

% \renewcommand{\rmdefault}{phv} % Arial Font
% \renewcommand{\sfdefault}{phv} % Arial Font

% \pagestyle{fancy}
% \fancyhead{}
% \fancyhead[CO,CE]{{\small{{\bf{\mytitle}} -- \myclass}}}

\begin{document}
\maketitle

\vspace{-1.5cm}

Lee cuidadosamente y contesta lo que se te pide.
Este examen está pensado para resolverse de manera individual, y en poco menos de 90 minutos.
Se sugiere que administres bien tu tiempo.

Al momento de contestar, intenta ser lo más explícito posible: se calificará con base en lo que esté escrito, y se considerará el proceso aún cuando la respuesta final esté errada.
Recuerda que puedes revisar material de la clase, el libro de texto o tus notas.
Buena suerte.

\section{Expresiones Regulares (36\%)}

Describe cada uno de los siguientes lenguajes utilizando expresiones regulares genéricas. En dado caso de que el lenguaje no se pueda describir con una ER, menciona por qué. Posteriormente e independientemente de si tienes ER o no, escribe dos ejemplos de palabras aceptadas para cada lenguaje.

\begin{enumerate}[label=\tt \alph*)]
    \itemsep0em
    \item El lenguaje de las palabras en $\{a,b\}^*$ que empiezan con $a$ y terminan con $b$.
    \item El lenguaje de las palabras en $\{0,1\}^*$ que son múltiplos de $100$.
    \item El lenguaje de las palabras en $\{x, y\}^*$ que son de longitud impar o que terminan en $xxx$.
    \item El lenguaje de las palabras en $\{0,1,2\}^*$ que tengan un $1$ en el centro, y sean simétricas.
    \item El lenguaje de las palabras en $\{z : z \in \mathbb{N}, 0 \leq z \leq 9\}^*$ que son también denominaciones de billetes o monedas expedidas por el Banco de México y que están en circulación. Omite centavos y series especiales (Onzas de plata, Centenarios, etc.).
    \item El lenguaje de las palabras en $\{a,b\}^*$ que tengan un número par de $b$s. (\textit{Hint: considera hacer un autómata si te parece muy complicado})
\end{enumerate}


\section{Gramáticas Regulares (14 \% + 6 \%)}

En \texttt{HTML}, le llamamos ``etiquetas'' a las \textit{keywords} para abrir o cerrar elementos de la página web.
Una etiqueta siempre se escribe entre \textit{brackets} para iniciar el elemento, y la misma etiqueta pero con una diagonal al principio para finalizar el elemento.
La etiqueta \verb|<html>...</html>| sirve para delimitar el contenido de la página web.

\begin{enumerate}[label=\tt \alph*)]
    \itemsep0em
    \item Escribe una expresión regular para seleccionar las \textit{keywords} de inicio y fin de las listas ordenadas (\texttt{ol}) y no ordenadas (\texttt{ul}), incluyendo los brackets como parte de la selección (4 \%)
    \item Convierte a AF (5 \%)
    \item Convierte a Gramática \textbf{Regular} (no GLC) (5 \%)
    \item Escribe el árbol de derivación para dos palabras válidas (+ 6\%)
\end{enumerate}

\section{Gramáticas Libres de Contexto (30 \% + 7 \%)}

BSL goes here!

El \textit{Biologic Space Lab} que orbita \textsc{SR388} sufrió una falla eléctrica dejando fuertes daños en algunos de sus sectores. Uno de los más afectados fue el que alberga muestras de la vida marina del planeta, el Sector 4 (AQA). Por ello, el equipo de mantenimiento te ha asignado la tarea de implementar un componente de control para el nivel del agua de los 32 tanques en el sector mientras se llevan a cabo las reparaciones.

Cada tanque tiene tres sensores: uno para la detección de movimiento (y así saber si en el tanque se encuentra algún espécimen o si el tanque se encuentra vacío), otro para medir la toxicidad del agua, y uno más para medir la cantidad de carga eléctrica, la cual es indicio de alguna falla. En el caso de los últimos dos, se marca como `peligroso' cuando se sobrepasa un umbral.

Los tanques envían actualizaciones al mecanismo de control una vez cada dos horas por medio de un packet de 8 bits, donde los primeros 3 corresponden a la lectura de los sensores (movimiento, toxicidad y carga, en ese orden), y los últimos 5 al ID del tanque (en binario).

HQ te ha notificado que la compuerta que alimenta a cada tanque debe cerrarse en cualquiera de las siguientes situaciones:

\begin{itemize}
    \itemsep0em
    \item Si las especies del tanque están a salvo (baja toxicidad y baja carga eléctrica).
    \item Si el tanque está deshabitado, pero es altamente tóxico, sin importar lo demás.
    \item Si el agua del tanque presenta una cantidad peligrosa de carga eléctrica, independientemente de que esté habitado o de su toxicidad.
\end{itemize}

Diseña un autómata que funcione como mecanismo de control para cerrar la compuerta de cada tanque si el packet completo de información que se recibe es aceptado. Sugerencias:

\begin{enumerate}[label=\tt \alph*)]
    \itemsep0em
    \item Genera el alfabeto de tu autómata (2 \%)
    \item Convierte las condiciones de aceptación a palabras de entrada específicas (3 \%)
    \item Diseña cuantos componentes sean necesarios (15 \%)
    \item Presenta el mecanismo de control completo (5 \%)
    \item Describe formalmente tu autómata (5 \%)
    \item Ponle un nombre a tu mecanismo de control (+2 \%)
    \item Si te es posible, minimízalo (+5 \%)
\end{enumerate}

\section*{Reto: Conversión de AFN a AFDs (+15 \%)}

Diseña un AFN por \textbf{concatenación} para el lenguaje de las palabras $\{a^n b^m\}$ donde $n$ y $m$ describen las veces que se repite cada carácter.
Las palabras tienen $n \,$ $a$s, seguidas por $m \,$ $b$s para cualquier valor de $n$ y $m$.

\begin{itemize}
    \item Palabras aceptadas: $\varepsilon, a, b, ab, aab, aabb, \dots$
    \item Palabras no aceptadas: $ba, bba, bbbba, abba, aaba, \dots$
\end{itemize}

Sugerencia: minimiza lo más que puedas tu AFN para que no batalles al convertirlo.

\vfill

\textbf{De acuerdo con el Código de Ética del Tecnológico de Monterrey, mi desempeño en esta actividad estará guiado por la integridad académica.}
\end{document}