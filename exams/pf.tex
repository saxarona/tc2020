\documentclass[8pt, onside]{article}
%\usepackage[a4paper]{geometry}
\usepackage{fullpage}
\usepackage[utf8]{inputenc}
\usepackage[spanish, mexico]{babel}
\usepackage{lipsum}
\usepackage{bm}
\usepackage{upgreek}
\usepackage{enumitem}
\usepackage{mathrsfs}
\usepackage{amsmath}
\usepackage{amssymb}
\usepackage{tikz}
\usepackage{tcolorbox}
\usepackage{csquotes}
\usetikzlibrary{arrows, automata}

\tikzset{
    automaton/.style={
        ->, %arrow type
        >=stealth', %arrow head type (bold)
        shorten >=1pt, 
        auto,
        %semithick,
        initial text=$ $, %no start text
    }
}


% mathtools for: Aboxed (put box on last equation in align envirenment)
\usepackage{microtype} %improves the spacing between words and letters

%% COLOR DEFINITIONS

\usepackage{xcolor} % Enabling mixing colors and color's call by 'svgnames'

\definecolor{MyColor1}{rgb}{0.2,0.4,0.6} %mix personal color
\newcommand{\textb}{\color{Black} \usefont{OT1}{lmss}{m}{n}}
\newcommand{\blue}{\color{MyColor1} \usefont{OT1}{lmss}{m}{n}}
\newcommand{\blueb}{\color{MyColor1} \usefont{OT1}{lmss}{b}{n}}
\newcommand{\red}{\color{LightCoral} \usefont{OT1}{lmss}{m}{n}}
\newcommand{\green}{\color{Turquoise} \usefont{OT1}{lmss}{m}{n}}

\DeclareMathOperator{\trace}{trace}
\DeclareMathOperator{\diag}{diag}

%% FONTS AND COLORS

%    SECTIONS

\usepackage{titlesec}
\usepackage{sectsty}
%%%%%%%%%%%%%%%%%%%%%%%%
%set section/subsections HEADINGS font and color
%\sectionfont{\color{Black}}  % sets colour of sections
%\subsectionfont{\color{Black}}  % sets colour of sections

%set section enumerator to arabic number (see footnotes markings alternatives)
\renewcommand\thesection{\arabic{section}} %define sections numbering
\renewcommand\thesubsection{\thesection\arabic{subsection}} %subsec.num.

%define new section style
\newcommand{\mysection}{
\titleformat{\section} [runin] {\usefont{OT1}{lmss}{b}{n}\color{MyColor1}} 
{\thesection} {3pt} {} } 


% %	CAPTIONS
% \usepackage{caption}
% \usepackage{subcaption}
% %%%%%%%%%%%%%%%%%%%%%%%%
% \captionsetup[figure]{labelfont={color=Turquoise}}


%		!!!EQUATION (ARRAY) --> USING ALIGN INSTEAD
%using amsmath package to redefine eq. numeration (1.1, 1.2, ...) 
\renewcommand{\theequation}{\thesection\arabic{equation}}

\setlength\parindent{0pt}




\makeatletter
\let\reftagform@=\tagform@
\def\tagform@#1{\maketag@@@{(\ignorespaces\textcolor{red}{#1}\unskip\@@italiccorr)}}
\renewcommand{\eqref}[1]{\textup{\reftagform@{\ref{#1}}}}
\makeatother
\usepackage{hyperref}
\hypersetup{colorlinks=true}

% For labeling top of page on every page but first one:
%\usepackage{fancyhdr}

\newcommand{\myclass}{TC2020 -- Matemáticas Computacionales} % Class name?
\newcommand{\mytitle}{Proyecto Final} % Title of document?
\newcommand{\mydate}{} % The date?
\newcommand{\myheader}{
    \begin{flushleft}
        \large
        Nombre: \rule{10 cm}{0.4mm} \quad \rule{2.5 cm}{0.4mm} \\
        Nombre: \rule{10 cm}{0.4mm} \quad \rule{2.5 cm}{0.4mm} \\
        Nombre: \rule{10 cm}{0.4mm} \quad \rule{2.5 cm}{0.4mm} \\
    \end{flushleft}
}

\title{
    \myclass \\
    \textbf{\mytitle} \\
    \myheader
    \date{}
}

% You can set the date automatically by replacing "date goes here" with "\today"

% \renewcommand{\rmdefault}{phv} % Arial Font
% \renewcommand{\sfdefault}{phv} % Arial Font

% \pagestyle{fancy}
% \fancyhead{}
% \fancyhead[CO,CE]{{\small{{\bf{\mytitle}} -- \myclass}}}

\begin{document}
\maketitle

\vspace{-1.5cm}

Este proyecto debe realizarse o bien de manera \textbf{individual} o \textbf{en equipos}.
Revisen con calma lo que se pide. Es bastante trabajo, por lo que se sugiere que consideren bien sus tiempos.
Al momento de contestar, intenten ser lo más explícitos posible: se calificará con base en lo que esté escrito, y se considerará el proceso aun cuando la respuesta final esté errada.

\vspace{1.5ex}

Todos los procedimientos deben venir en \textit{typesetting} (Word o \LaTeX), y serán entregados en un PDF.
Si necesitan buscar referencias de algo que no está en el material de la clase, se invita a que lo citen apropiadamente.
La inclusión de bibliografía adecuadamente citada (IEEE, ACM, MLA {\tiny (APA no tiene lugar en ciencias)}) da \textbf{1 punto extra} sobre este trabajo.

\vspace{1.5ex}

\textbf{IMPORTANTE}: Deberán contestar su proyecto usando el \textit{alfabeto adecuado}.

\vspace{1.5ex}

Sea $\Sigma$ el \textit{alfabeto adecuado} para su proyecto, y \texttt{mod} la operación módulo (residuo después de la división), entonces:

$$ \Sigma = 
\begin{cases}
   \{0,1\}, & \text{ si el \textbf{número maestro} } \mathtt{mod} \, 3 = 0 \\
   \{a,b\}, & \text{ si el \textbf{número maestro} } \mathtt{mod} \, 3 = 1 \\
   \{x,y\}, & \text{ si el \textbf{número maestro} } \mathtt{mod} \, 3 = 2 \\
\end{cases}
$$

donde \textbf{el número maestro} es la suma del último dígito de cada una de sus matrículas:

\begin{align*}
    A0117006\mathbf{5} + A0097344\mathbf{1} & =  6, \\
    6 \, \mathtt{mod} \, 3 & = 0 \\
    \therefore \quad \Sigma & = \{0,1\}
\end{align*}

Además, consideren que si su equipo tiene $x$ integrantes y la ponderación de una pregunta es de $\frac{20}{x}$, en los equipos de más de un integrante hay que hacer $x$ versiones del ejercicio, una para cada alumno;
de tal manera que $\frac{20}{2} + \frac{20}{2} = 20 \%$ si $x=2$, por ejemplo.

\rule{\textwidth}{0.1mm}

\section{Máquinas de Turing (40 \% + 2 \%)}

Generen una máquina de Turing $\mathcal{M}_i$ que realice las siguientes operaciones.
Para cada una, incluyan su \textbf{definición formal} (usando $M = (Q,\Sigma,\Gamma,\delta,q,f)$ y su \textbf{diagrama de estados} (`autómata'):

\begin{enumerate}[label=\tt \alph*)]
    \itemsep0em
    \item $\mathcal{M}_1 =$ \textit{Char}-wise \texttt{AND} (10 \%)
    \begin{itemize}
        \item \textbf{Input}: dos cadenas de caracteres usando el $\Sigma$ adecuado
        \item \textbf{Output}: una cadena binaria que sirva como \textit{Character}-wise \texttt{AND} de los inputs, usando 1 si son iguales y 0 si no
        \item \textbf{Ejemplo}: $\mathcal{M}_1(aba,abb) = 110$
    \end{itemize}
    \pagebreak
    \item $\mathcal{M}_2 = $ Inversor de palabras (10 \%)
    \begin{itemize}
        \item \textbf{Input}: Una palabra $w$ usando el $\Sigma$ adecuado
        \item \textbf{Output}: La inversión $w^R$ de la palabra $w$
        \item \textbf{Ejemplo}: $\mathcal{M}_2(0111000) = 0001110$
    \end{itemize}
    \item $\mathcal{M}_3 = $ Verificador de CURP ($\frac{20}{x}$ \%)
    \begin{itemize}
        \item \textbf{Input}: Un CURP válido
        \item \textbf{Output}: Una \textit{string} de 3 caracteres en donde el primero es la inicial del sexo ($H, M$) y los últimos dos son la clave de la entidad federativa.
        \item \textbf{Ejemplo}: Son muchas posibilidades, así que considera que los inputs serán personas de cualquier sexo pero de tu mismo estado de procedencia.
        \item \textbf{Bonus}: Escribe una expresión regular en PCRE para identificar un CURP bien formado (+ $\frac{1}{x}$ \%)
    \end{itemize}
\end{enumerate}

\section{Turing \textit{Encoding} (50 \% + 2 \%)}

Lean \textit{On Computable Numbers with an Application to the Entscheidungsproblem}, de Alan Turing, sobre todo las páginas 230--241 (1--12 del \texttt{PDF}). Presten especial atención al \textit{encoding} para la máquina universal. Con esta información y considerando que los tres alfabetos adecuados son equivalentes (es decir $0=a=x$ y $1=b=y$), entonces:

\begin{enumerate}[label=\tt \alph*)]
    \item Generen una \textit{skeleton table} para los ejercicios $a)$ y $b)$ de la sección anterior (30 \%)
    \item Expresen cada una de las \textit{skeleton tables} como una \textit{Standard Description} (S.D) (10 \%)
    \item Expresen cada una de las \textit{standard descriptions} como un \textit{Description Number} (D.N) (10 \%)
    \item \textbf{Bonus}: generen una expresión regular para cualquier S.D y otra para cualquier D.N en PCRE (+ 2 \%)
\end{enumerate}

Consideren que tanto la tabla como las descripciones deben coincidir con los estados que hayan utilizado en la Sección 1.

\section{Lectura y opinión ($\frac{10}{x}$ \%)}

{\footnotesize \it Esto es prácticamente pregunta de rescate...}

\begin{enumerate}[label=\tt \alph*)]
    \item Escribe una opinión breve (entre 150-200 palabras) del artículo de Turing ($\frac{8}{x}$ \%)
    \item Escribe una opinión breve (entre 150-200 palabras) de la clase, en donde describas en qué campo de tu área o puesto en el futuro podrías usar lo aprendido en el curso. ($\frac{2}{x}$ \%)
\end{enumerate}

\vfill

\textbf{De acuerdo con el Código de Ética del Tecnológico de Monterrey, mi desempeño en esta actividad estará guiado por la integridad académica.}
\end{document}