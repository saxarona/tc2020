\documentclass[8pt, onside]{article}
%\usepackage[a4paper]{geometry}
\usepackage{fullpage}
\usepackage[utf8]{inputenc}
\usepackage[spanish, mexico]{babel}
\usepackage{lipsum}
\usepackage{bm}
\usepackage{upgreek}
\usepackage{enumitem}
\usepackage{mathrsfs}
\usepackage{amsmath}
\usepackage{amssymb}
\usepackage{tikz}
\usepackage{tcolorbox}
\usepackage{csquotes}
\usetikzlibrary{arrows, automata}

\tikzset{
    automaton/.style={
        ->, %arrow type
        >=stealth', %arrow head type (bold)
        shorten >=1pt, 
        auto,
        %semithick,
        initial text=$ $, %no start text
    }
}


% mathtools for: Aboxed (put box on last equation in align envirenment)
\usepackage{microtype} %improves the spacing between words and letters

%% COLOR DEFINITIONS

\usepackage{xcolor} % Enabling mixing colors and color's call by 'svgnames'

\definecolor{MyColor1}{rgb}{0.2,0.4,0.6} %mix personal color
\newcommand{\textb}{\color{Black} \usefont{OT1}{lmss}{m}{n}}
\newcommand{\blue}{\color{MyColor1} \usefont{OT1}{lmss}{m}{n}}
\newcommand{\blueb}{\color{MyColor1} \usefont{OT1}{lmss}{b}{n}}
\newcommand{\red}{\color{LightCoral} \usefont{OT1}{lmss}{m}{n}}
\newcommand{\green}{\color{Turquoise} \usefont{OT1}{lmss}{m}{n}}

\DeclareMathOperator{\trace}{trace}
\DeclareMathOperator{\diag}{diag}

%% FONTS AND COLORS

%    SECTIONS

\usepackage{titlesec}
\usepackage{sectsty}
%%%%%%%%%%%%%%%%%%%%%%%%
%set section/subsections HEADINGS font and color
%\sectionfont{\color{Black}}  % sets colour of sections
%\subsectionfont{\color{Black}}  % sets colour of sections

%set section enumerator to arabic number (see footnotes markings alternatives)
\renewcommand\thesection{\arabic{section}} %define sections numbering
\renewcommand\thesubsection{\thesection\arabic{subsection}} %subsec.num.

%define new section style
\newcommand{\mysection}{
\titleformat{\section} [runin] {\usefont{OT1}{lmss}{b}{n}\color{MyColor1}} 
{\thesection} {3pt} {} } 


% %	CAPTIONS
% \usepackage{caption}
% \usepackage{subcaption}
% %%%%%%%%%%%%%%%%%%%%%%%%
% \captionsetup[figure]{labelfont={color=Turquoise}}


%		!!!EQUATION (ARRAY) --> USING ALIGN INSTEAD
%using amsmath package to redefine eq. numeration (1.1, 1.2, ...) 
\renewcommand{\theequation}{\thesection\arabic{equation}}

\setlength\parindent{0pt}




\makeatletter
\let\reftagform@=\tagform@
\def\tagform@#1{\maketag@@@{(\ignorespaces\textcolor{red}{#1}\unskip\@@italiccorr)}}
\renewcommand{\eqref}[1]{\textup{\reftagform@{\ref{#1}}}}
\makeatother
\usepackage{hyperref}
\hypersetup{colorlinks=true}

% For labeling top of page on every page but first one:
%\usepackage{fancyhdr}

\newcommand{\myclass}{TC2020 -- Matemáticas Computacionales} % Class name?
\newcommand{\mytitle}{Examen 2} % Title of document?
\newcommand{\mydate}{25.10.18} % The date?
\newcommand{\myheader}{
    \begin{flushleft}
        \large
        Nombre: \rule{13 cm}{0.4mm} \\
        Matrícula: \rule{5 cm}{0.4mm} \hfill Fecha: \mydate
    \end{flushleft}
}

\title{
    \myclass \\
    \textbf{\mytitle} \\
    \myheader
    \date{}
}

% You can set the date automatically by replacing "date goes here" with "\today"

% \renewcommand{\rmdefault}{phv} % Arial Font
% \renewcommand{\sfdefault}{phv} % Arial Font

% \pagestyle{fancy}
% \fancyhead{}
% \fancyhead[CO,CE]{{\small{{\bf{\mytitle}} -- \myclass}}}

\begin{document}
\maketitle

\vspace{-1.5cm}

Lee cuidadosamente y contesta lo que se te pide.
Este examen es individual.

Al momento de contestar, intenta ser lo más explícito posible: se calificará con base en lo que esté escrito, y se considerará el proceso aún cuando la respuesta final esté errada.
Recuerda que puedes revisar material de la clase, el libro de texto o tus notas.
Administra bien tu tiempo.
Buena suerte.

\section{Expresiones Regulares (36\%)}

Describe cada uno de los siguientes lenguajes utilizando expresiones regulares genéricas (las que vimos en clase).
En dado caso de que el lenguaje no se pueda describir con una ER, menciona por qué.
Posteriormente e independientemente de si tienes ER o no, escribe dos ejemplos de palabras aceptadas para cada lenguaje.

\begin{enumerate}[label=\tt \alph*)]
    \itemsep0em
    \item El lenguaje de las palabras en $\{a,b\}$ que empiezan con $a$ y terminan con $b$.
    \item El lenguaje de las palabras en $\{0,1\}$ que son múltiplos de $100$.
    \item El lenguaje de las palabras en $\{x, y\}$ que son de longitud impar, o que terminan en $xxx$.
    \item El lenguaje de las palabras en $\{0,1,2\}$ que tengan un $1$ en el centro, y se lean al derecho y al revés.
    \item El lenguaje de las palabras en $\{z : z \in \mathbb{N}, 0 \leq z \leq 9\}$ que son también denominaciones de billetes o monedas expedidas por el Banco de México y que están en circulación. Omite centavos y series especiales (Onzas de plata, Centenarios, etc.).
    \item El lenguaje de las palabras en $\{a,b\}$ que tengan un número par de $b$s. (\textit{Hint: considera hacer un autómata y luego convertir a ER si te parece muy complicado})
\end{enumerate}


\section{Gramáticas Regulares (34 \% + 6 \%)}

En \texttt{HTML}, le llamamos ``etiquetas'' a las \textit{keywords} para abrir o cerrar elementos de la página web.
Una etiqueta siempre se escribe entre \textit{brackets} para iniciar el elemento, y la misma etiqueta pero con una diagonal al principio para finalizar el elemento.
La etiqueta \verb|<html>...</html>| sirve para delimitar el contenido de la página web.

\begin{enumerate}[label=\tt \alph*)]
    \itemsep0em
    \item Escribe una expresión regular para seleccionar \textbf{sólo} las \textit{etiquetas} de inicio y fin de \texttt{ol} y \texttt{ul}, incluyendo los brackets como parte de la selección. \texttt{</ul>} es una palabra válida (12 \%)
    \item Convierte la Expresión Regular a Autómata Finito (12 \%)
    \item Expresa el mismo lenguaje como una Gramática \textbf{Regular}, (no GLC). Recuerda que las GRs crecen hacia un solo lado (10 \%)
    \item Escribe el árbol de derivación para dos palabras válidas (+ 6\%)
\end{enumerate}


\begin{comment}
\section{Ambigüedad en GLCs (20\%)}

Sea $G=(\{S\},\{\mathtt{if}, c, \mathtt{then}, \mathtt{else}, x\}, \{S \to x, S \to \mathtt{if} \, c \, \mathtt{then} \, S, S \to \mathtt{if} \, c \, \mathtt{then} \, S \, \mathtt{else} \, S\}, S)$ una gramática libre de contexto. Recuerda que las gramáticas vienen de la forma $G = (V, \Sigma, R, S)$.

\begin{enumerate}[label=\tt \alph*)]
    \item Demuestra que $G$ es ambigua. Para eso, encuentra una palabra a la cual puedas llegar utilizando dos árboles de derivación distintos. $\mathtt{if} \, c \, \mathtt{then} \, x$ es una palabra válida (10 \%)
    \item Expresa cada árbol de derivación en forma lineal (la que usa $\implies$) (10 \%)
\end{enumerate}
\end{comment}


\section{Gramáticas Libres de Contexto (30 \% +18 \%)}

{\small \it
Este problema es un poco más complejo de lo que hemos resuelto hasta el momento.
Se recomienda que avances lo más que puedas en el resto del examen antes de comenzarlo.
}

\bigskip

El \textit{Biologic Space Lab} está en problemas nuevamente.
Debido a una brecha en un ducto de ventilación, ciertos parásitos lograron entrar a uno de los laboratorios de acceso restringido, en el cual se almacenaban muestras de ADN de especies extraterrestres inteligentes para fines taxonómicos.
Algunos de los bancos de datos están corruptos debido a la infección, pero otros aún están a salvo.
HQ te ha asignado la tarea de diseñar un sistema de control para identificar aquellas muestras de ADN que aún pueden rescatarse, y te ha enviado la siguiente información:

\begin{itemize}
    \itemsep0em
    \item Los parásitos tienen la capacidad de alterar el genoma de las especies, copiando y replicando ciertos genes mediante mutación.
    \item El parásito replica los genes pertenecientes al número de extremidades inferiores y de globos oculares de manera arbitraria.
    \item Gracias al análisis de los datos, se determinó que los parásitos sólo están afectando a ciertas especies que consideran `aptas y balanceadas', que son aquellas que tienen el mismo número de ojos que de piernas.
    \item Sin embargo, se ha confirmado que el número de genes replicados pertenecientes a los ojos nunca es el igual que el de las extremidades inferiores en el mismo individuo---en un individuo mutado, el número final de ojos nunca es igual al número final de piernas.
\end{itemize}

Además, dado a la gran bio-diversidad del BSL, HQ te sugiere lo siguiente:

\begin{tcolorbox}
{\small Tenemos cerca de 900,000 muestras en el databank, de las cuáles sólo 487,350 podrían estar infectadas.
Nos interesa saber cuáles de esas muestras posiblemente infectadas están sanas para separarlas cuanto antes.
Como son tantas especies distintas, creemos que es mejor enfocarse en encontrar diferencias en el número de genes en lugar de buscar una secuencia válida.}
\end{tcolorbox}

Diseña un sistema de reglas (Gramática) para generar subcadenas de genes de ojos y piernas de individuos sanos.

Para probar tu sistema, HQ te envió el ADN de un individuo de la especie \textit{Etecoon}, que es un primate con antena y que por naturaleza debería tener dos ojos y dos piernas.
Utilizando las reglas de tu gramática, el sistema intentará generar una palabra idéntica al genoma del \textit{Etecoon}.
Si lo logra, el \textit{Etecoon} está sano.

Piensa de manera abstracta.
Imagina que el \textit{input} de tu sistema son las subcadenas de ADN específicas de ojos y piernas. Sugerencias:

\begin{enumerate}[label=\tt \alph*)]
    \item Genera el alfabeto de terminales. ¿Qué símbolos usarás para representar ojos y cuál para representar piernas? (2 \%)
    \item Describe con un enunciado (como los de la Sección 1) el lenguaje que debería aceptar tu gramática (5 \%)
    \item Genera las reglas de tu GLC (10 \%)
    \item Describe formalmente tu GLC usando $G = (V, \Sigma, R, S)$ (5 \%)
    \item Usando tu gramática, escribe el árbol de derivación de un \textit{Dachora} sano, una especie de ave que tiene dos piernas y dos ojos (8 \%)
    \item Dibuja un \textit{Etecoon} y un \textit{Dachora} (+8 \%)
    \item Haz un Autómata de Pila que acepte el mismo lenguaje que tu GLC (+10 \%)
\end{enumerate}

\vfill

\textbf{De acuerdo con el Código de Ética del Tecnológico de Monterrey, mi desempeño en esta actividad estará guiado por la integridad académica.}
\end{document}