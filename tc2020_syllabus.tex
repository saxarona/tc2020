\documentclass[12pt, letterpaper, oneside]{article}
%\usepackage{geometry}
\usepackage[spanish, mexico]{babel}
\usepackage[utf8]{inputenc}
\usepackage{amssymb}
\usepackage[inner=1.5cm,outer=1.5cm,top=2.5cm,bottom=2.5cm]{geometry}
\pagestyle{empty}
\usepackage{graphicx}
\usepackage{fancyhdr, lastpage, bbding, pmboxdraw}
\usepackage[usenames,dvipsnames]{color}
\definecolor{darkblue}{rgb}{0,0,.6}
\definecolor{darkred}{rgb}{.7,0,0}
\definecolor{darkgreen}{rgb}{0,.6,0}
\definecolor{red}{rgb}{.98,0,0}
\usepackage[colorlinks,pagebackref,pdfusetitle,urlcolor=darkblue,citecolor=darkblue,linkcolor=darkred,bookmarksnumbered,plainpages=false]{hyperref}
\renewcommand{\thefootnote}{\fnsymbol{footnote}}

\newcommand{\thecourse}{Matemáticas Computacionales (TC2020)}
\newcommand{\thesemester}{Agosto--Diciembre, 2018}
\newcommand{\theinstructor}{Xavier Sánchez Díaz}
\newcommand{\themail}{sax@itesm.mx}
\newcommand{\thetime}{LuJu 11:30 -- 13:00}
\newcommand{\theplace}{TBD}

\newcommand{\topic}{{\color{darkgreen}{\Rectangle}}}
\newcommand{\subtopic}{{\enskip \color{darkblue}{\Rectangle}}}

\pagestyle{fancyplain}
\fancyhf{}
\lhead{ \fancyplain{}{\thecourse} }
%\chead{ \fancyplain{}{} }
\rhead{ \fancyplain{}{\thesemester} }
%\rfoot{\fancyplain{}{page \thepage\ of \pageref{LastPage}}}
\fancyfoot[RO] {Página \thepage\ de \pageref{LastPage}}
\thispagestyle{plain}

%%%%%%%%%%%% LISTING %%%
\usepackage{listings}
\usepackage{caption}
\DeclareCaptionFont{white}{\color{white}}
\DeclareCaptionFormat{listing}{\colorbox{gray}{\parbox{\textwidth}{#1#2#3}}}
\captionsetup[lstlisting]{format=listing,labelfont=white,textfont=white}
\usepackage{verbatim} % used to display code
\usepackage{fancyvrb}
\usepackage{acronym}
\usepackage{amsthm}
\VerbatimFootnotes % Required, otherwise verbatim does not work in footnotes!



\definecolor{OliveGreen}{cmyk}{0.64,0,0.95,0.40}
\definecolor{CadetBlue}{cmyk}{0.62,0.57,0.23,0}
\definecolor{lightlightgray}{gray}{0.93}



\lstset{
  %language=bash,                          % Code langugage
  basicstyle=\ttfamily,                   % Code font, Examples: \footnotesize, \ttfamily
  keywordstyle=\color{OliveGreen},        % Keywords font ('*' = uppercase)
  commentstyle=\color{gray},              % Comments font
  numbers=left,                           % Line nums position
  numberstyle=\tiny,                      % Line-numbers fonts
  stepnumber=1,                           % Step between two line-numbers
  numbersep=5pt,                          % How far are line-numbers from code
  backgroundcolor=\color{lightlightgray}, % Choose background color
  frame=none,                             % A frame around the code
  tabsize=2,                              % Default tab size
  captionpos=t,                           % Caption-position = bottom
  breaklines=true,                        % Automatic line breaking?
  breakatwhitespace=false,                % Automatic breaks only at whitespace?
  showspaces=false,                       % Dont make spaces visible
  showtabs=false,                         % Dont make tabls visible
  columns=flexible,                       % Column format
  morekeywords={__global__, __device__},  % CUDA specific keywords
}

%%%%%%%%%%%%%%%%%%%%%%%%%%%%%%%%%%%%
\begin{document}
  \begin{center}
  {\Large \textsc{\thecourse}}
  \end{center}
  \begin{center}
  \thesemester
  \end{center}

  \begin{center}
  \rule{6in}{0.4pt}
  \begin{minipage}[t]{.75\textwidth}
  \begin{tabular}{llcccll}
  \textbf{Instructor:} & \theinstructor & & &  & \textbf{Hora:} & \thetime \\
  \textbf{Email:} &  \href{mailto:sax@itesm.mx}{\themail} & & & & \textbf{Lugar:} & \theplace
  \end{tabular}
  \end{minipage}
  \rule{6in}{0.4pt}
  \end{center}
  \vspace{.5cm}
  \setlength{\unitlength}{1in}
  \renewcommand{\arraystretch}{2}

  \noindent\textbf{Página del curso:}
  
  \begin{enumerate}
  \item \url{https://saxarona.gitlab.io/teaching/tc2020}
  \end{enumerate}

  \vskip.15in

  \noindent\textbf{Horario de oficina:} TBD.

  \vskip.15in

  \noindent\textbf{Material recomendado:} % \footnotemark
  Ésta es una lista de recursos que pueden serte de utilidad durante el curso.
  No es mala idea que los consultes ocasionalmente.

  \begin{itemize}
    \item E. Rich, \textit{Automata, Computatbility and Complexity: Theory and Applications}. Austin, TX: Prentice Hall, 2008.
    \item M. Sipser, \textit{Introduction to the Theory of Computation}, 3rd Ed. Boston, MA: Cengage Learning, 2012.
  \end{itemize} 

  % \footnotetext{Downloadable ebook versions are available on AeLP.}

  \vskip.15in

  \noindent\textbf{Objetivos:}
  Al final del curso, el alumno:

  \begin{itemize}
    \item será capaz de \textbf{abstraer modelos discretos relevantes y adecuados, a partir de situaciones que observa en el mundo real}, en términos de conceptos de estados, transiciones, autómatas, expresiones regulares y gramáticas;
    \item aplicará \textbf{transformaciones} a los modelos antes mencionados, tales como la simplificación o conversión a formas más convenientes que permitan llegar a la solución abstracta del problema a resolver;
    \item distinguirá entre aquellos problemas \textbf{resolubles}  y aquellos que son \textbf{imposibles} dentro del campo, para evitar perder el tiempo tratando de resolver problemas que se sabe que no tienen solución;
    \item aplicará \textbf{soluciones abstractas en el mundo real} aportadas por los métodos de autómatas y lenguajes, considerando que los modelos abstractos son una \textit{simplificación útil} y no una \textit{verdad inmutable}.
  \end{itemize}

  \vskip.15in
  \noindent\textbf{Requisitos:}
  Haber cursado Matemáticas Discretas (TC1003) y Estructura de Datos (TC1018).

  % \vspace*{.15in}
  \pagebreak

  \noindent \textbf{Índice analítico del curso:}
  El curso está dividido en tres módulos---Lenguajes regulares, lenguajes libres de contexto y lenguajes recursivamente numerables.

  \begin{center} 
  \begin{minipage}{5in}
  \begin{flushleft}
  {\large I. Lenguajes regulares} \\[2ex]
  \topic ~Conceptos preliminares \\
  \subtopic ~Conceptos matemáticos \\
  \subtopic ~Lenguajes formales \\
  \topic ~Teoría de los lenguajes \\
  \subtopic ~Máquinas de estados finitos \\
  \subtopic ~Autómatas finitos determinísticos \\
  \subtopic ~Autómatas finitos no determinísticos \\
  \topic ~Lenguajes regulares \\
  \subtopic ~Expresiones regulares \\
  \subtopic ~Gramáticas regulares \\
  \subtopic ~Análisis Léxico \\[2.5ex]
  {\large II. Lenguajes libres de contexto }\\[2ex]
  \topic ~Lenguajes libres de contexto \\
  \subtopic ~Jerarquía de Chomsky \\
  \subtopic ~Gramáticas libres de contexto \\
  \subtopic ~Propiedades de las gramáticas \\
  \topic ~Análisis sintáctico \\
  \subtopic ~Análisis sintáctico descendente \\
  \subtopic ~Análisis sintáctico ascendent e\\[2.5ex]
  {\large III. Lenguajes recursivamente numerables} \\[2ex]
  \topic ~Tópicos avanzados \\
  \subtopic ~Máquinas de Turing \\
  \subtopic ~Decidibilidad \\
  \subtopic ~Computabilidad
  \end{flushleft}
  \end{minipage}
  \end{center}

  \vspace*{.15in}
  \noindent\textbf{Política de evaluación:} Tareas y Quizzes (50\%), Examen de medio término (25\%), Examen Final (25\%).

  \vskip.15in
  \noindent\textbf{Fechas importantes:}
  \begin{center} \begin{minipage}{3.8in}
  \begin{flushleft}
  Examen de medio término \dotfill ~TBD \\
  Semana i \dotfill ~TBD \\
  Examen final \dotfill ~TBD
  \end{flushleft}
  \end{minipage}
  \end{center}

  \vskip.15in
  \noindent\textbf{Políticas del curso:}  
  \begin{itemize}
  \item Please sign up for AeLP. I will confirm your enrollment for the course, then you will be able to see the course page.
  \item We have weekly homework and quiz. You will be given a quick quiz (based on the given homework)
  on the day that the homework is due. You are allowed to use your homework solutions to help you on the quiz, but not anything else.
  \item Late homework will never be accepted. Homework not submitted online before the deadline and/or not turned in with the quiz will be considered late.
  \item Homework solutions must be typeset (preferably using \LaTeX~), and all programming codes should be well documented.
  \item Nearly perfect solutions may be considered as an official solution of that homework and will be uploaded to the course web site, and the student gets a bonus mark.
  \item All homework solutions, programming codes, etc., must be submitted both electronically (through AeLP) and in class (along with the quiz). For electronic submission, create a folder (directory) on your computer, put your files all in there, zip the package, and submit it once you get them done. You can submit your files only once, and you are NOT allowed to edit the files after submission, so read/edit your files carefully before submission. If there is something that you would like me to know while grading your assignment, please write it in the comment box above the submit button or create a file called \texttt{README} in that directory and write your message there. So, please do not mail your homework solutions, codes, etc to me.
  \item You may discuss homework problems with other students, but you must write up your homework independently in your own words. You are not allowed to search the Web for solutions,  as AeLP is equipped with a built-in plagiarism detector.
  \item Your lowest homework-quiz score will be dropped when calculating your final homework-quiz grade.
  \item The exams may or may not be take-home. If not, by default, all exams (midterms and final) are closed book, and you are not allowed to use any electronic devices such as mobiles and tablets. 
  \end{itemize}

  \vskip.15in
  \noindent\textbf{Políticas de clase:}  
  \begin{itemize}
  \item Regular attendance is essential and expected. A student who incurs an excessive number of absences may be withdrawn from the class at the instructor's discretion.
  \item Be courteous when using mobile devices. Make sure your cell phone is turned fully off, or silent. No texting, reading emails, playing games, or whatever else it is that people do with those wretched gizmos.
  \item If you must use a laptop in class, then turn off the sound and do not type on laptop keyboards which is really distracting.
  \item Missing one class could easily lead to a disastrous domino effect. If you have to miss a lecture, then I strongly recommend you study the material you missed before you return to class. I require that you know all material covered in class. You are responsible for making up anything that was covered in lectures you missed. If you miss a lecture, I recommend doing the following:
  \begin{itemize}
  \item Photocopy, and read notes from someone who was in class,
  \item Reading the relevant sections from the lecture note, texts, Wikipedia, etc.
  %\item Look at the lecture schedule posted online.
  \end{itemize}
  After you have done this, you may contact me if you need clarification on any materials.
  \end{itemize}

  \vskip.15in
  \noindent\textbf{Honestidad académica:}   Lack of knowledge of the academic honesty policy is not a reasonable explanation for a violation. Questions related to course assignments and the
  academic honesty policy should be directed to the instructor. \\
  {\color{darkred}{\Large \HandRight}} ~I certainly impose a sanction on the student committed to any academic fraud. It varies depending upon the instructor's evaluation of the nature and gravity of the offense. Possible sanctions include but are not limited to, the following: (1) Require the student to redo the assignment; (2) Require the student to complete another assignment; (3) Assign a grade of \textbf{zero} to the assignment; (4) Assign  a final grade of \textbf{zero} for the whole course.

  %%%%%% END 
\end{document} 