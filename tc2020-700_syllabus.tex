\documentclass[12pt, letterpaper, oneside]{article}
%\usepackage{geometry}
\usepackage[spanish, mexico]{babel}
\usepackage[utf8]{inputenc}
\usepackage{amssymb}
\usepackage[inner=1.5cm,outer=1.5cm,top=2.5cm,bottom=2.5cm]{geometry}
\pagestyle{empty}
\usepackage{graphicx}
\usepackage{fancyhdr, lastpage, bbding, pmboxdraw}
\usepackage[usenames,dvipsnames]{color}
\definecolor{darkblue}{rgb}{0,0,.6}
\definecolor{darkred}{rgb}{.7,0,0}
\definecolor{darkgreen}{rgb}{0,.6,0}
\definecolor{red}{rgb}{.98,0,0}
\usepackage[colorlinks,pagebackref,pdfusetitle,urlcolor=darkblue,citecolor=darkblue,linkcolor=darkred,bookmarksnumbered,plainpages=false]{hyperref}
\renewcommand{\thefootnote}{\fnsymbol{footnote}}

\newcommand{\thecourse}{Matemáticas Computacionales (TC2020--700)}
\newcommand{\thesemester}{Agosto--Diciembre 2018}
\newcommand{\theinstructor}{Xavier Sánchez Díaz}
\newcommand{\themail}{sax@itesm.mx}
\newcommand{\thetime}{LuJu 11:30--13:00}
\newcommand{\theplace}{Sesión vía Zoom}

\newcommand{\topic}{{\color{darkgreen}{\Rectangle}}}
\newcommand{\subtopic}{{\enskip \color{darkblue}{\Rectangle}}}

\pagestyle{fancyplain}
\fancyhf{}
\lhead{ \fancyplain{}{\thecourse} }
%\chead{ \fancyplain{}{} }
\rhead{ \fancyplain{}{\thesemester} }
%\rfoot{\fancyplain{}{page \thepage\ of \pageref{LastPage}}}
\fancyfoot[RO] {Página \thepage\ de \pageref{LastPage}}
\thispagestyle{plain}

%%%%%%%%%%%% LISTING %%%
\usepackage{listings}
\usepackage{caption}
\DeclareCaptionFont{white}{\color{white}}
\DeclareCaptionFormat{listing}{\colorbox{gray}{\parbox{\textwidth}{#1#2#3}}}
% \captionsetup[lstlisting]{format=listing,labelfont=white,textfont=white}
\usepackage{verbatim} % used to display code
\usepackage{fancyvrb}
\usepackage{acronym}
\usepackage{amsthm}
\VerbatimFootnotes % Required, otherwise verbatim does not work in footnotes!



\definecolor{OliveGreen}{cmyk}{0.64,0,0.95,0.40}
\definecolor{CadetBlue}{cmyk}{0.62,0.57,0.23,0}
\definecolor{lightlightgray}{gray}{0.93}

\lstset{
  %language=bash,                          % Code langugage
  basicstyle=\ttfamily,                   % Code font, Examples: \footnotesize, \ttfamily
  keywordstyle=\color{OliveGreen},        % Keywords font ('*' = uppercase)
  commentstyle=\color{gray},              % Comments font
  numbers=left,                           % Line nums position
  numberstyle=\tiny,                      % Line-numbers fonts
  stepnumber=1,                           % Step between two line-numbers
  numbersep=5pt,                          % How far are line-numbers from code
  backgroundcolor=\color{lightlightgray}, % Choose background color
  frame=none,                             % A frame around the code
  tabsize=2,                              % Default tab size
  captionpos=t,                           % Caption-position = bottom
  breaklines=true,                        % Automatic line breaking?
  breakatwhitespace=false,                % Automatic breaks only at whitespace?
  showspaces=false,                       % Dont make spaces visible
  showtabs=false,                         % Dont make tabls visible
  columns=flexible,                       % Column format
  morekeywords={__global__, __device__},  % CUDA specific keywords
}

%%%%%%%%%%%%%%%%%%%%%%%%%%%%%%%%%%%%
\begin{document}
  \begin{center}
  {\Large \textsc{\thecourse}}
  \end{center}
  \begin{center}
  \thesemester
  \end{center}

  \begin{center}
  \rule{6in}{0.4pt}
  \begin{minipage}[t]{.75\textwidth}
  \begin{tabular}{llcccll}
  \textbf{Instructor:} & \theinstructor & & &  & \textbf{Hora:} & \thetime \\
  \textbf{Email:} &  \href{mailto:sax@itesm.mx}{\themail} & & & & \textbf{Lugar:} & \theplace
  \end{tabular}
  \end{minipage}
  \rule{6in}{0.4pt}
  \end{center}
  \vspace{.5cm}
  \setlength{\unitlength}{1in}
  \renewcommand{\arraystretch}{2}

  \noindent\textbf{Página del curso:}
  
  \begin{enumerate}
  \item \url{https://saxarona.gitlab.io/teaching/tc2020}
  \end{enumerate}

  \vskip.15in

  \noindent\textbf{Horario de oficina:}
  Usualmente puedes encontrarme los días lunes de 16:00 a 18:00 hrs. y en ocasiones los miércoles de 11:00 a 13:00 hrs. en la oficina (CT-536, CETEC Sur).
  Sin embargo, es más fácil que envíes un correo para poder agendar una cita.

  \vskip.15in

  \noindent\textbf{Material recomendado:} % \footnotemark
  Ésta es una lista de recursos que pueden serte de utilidad durante el curso.

  \begin{itemize}
    \item A. Maheshwari y M. Smid, Introduction to Theory of Computation. Canada. 2017. \footnotemark 
    \item R. Brena, \textit{Autómatas y Lenguajes}, México: McGraw Hill, 2010.
    \item E. Rich, \textit{Automata, Computatbility and Complexity: Theory and Applications}. Austin, TX: Prentice Hall, 2008.
    \item M. Sipser, \textit{Introduction to the Theory of Computation}, 3rd Ed. Boston, MA: Cengage Learning, 2012.
    \item J. Hopcroft y J. Ullman, \textit{Introduction to automata theory, languages and computation}, USA: Addison Wesley, 2001.
  \end{itemize} 

  \footnotetext{Puedes obtener este libro de manera gratuita en \url{http://cglab.ca/~michiel/TheoryOfComputation/TheoryOfComputation.pdf}}

  \vskip.15in

  \noindent\textbf{Objetivos:}
  Al final del curso, el alumno:

  \begin{itemize}
    \item será capaz de \textbf{abstraer modelos discretos relevantes y adecuados, a partir de situaciones que observa en el mundo real}, en términos de conceptos de estados, transiciones, autómatas, expresiones regulares y gramáticas;
    \item aplicará \textbf{transformaciones} a los modelos antes mencionados, tales como la simplificación o conversión a formas más convenientes que permitan llegar a la solución abstracta del problema a resolver;
    \item distinguirá entre aquellos problemas \textbf{resolubles}  y aquellos que son \textbf{imposibles} dentro del campo, para evitar perder el tiempo tratando de resolver problemas que se sabe que no tienen solución;
    \item aplicará \textbf{soluciones abstractas en el mundo real} aportadas por los métodos de autómatas y lenguajes, considerando que los modelos abstractos son una \textit{simplificación útil} y no una \textit{verdad inmutable}.
  \end{itemize}

  \vskip.15in
  \noindent\textbf{Requisitos:}
  Haber cursado Matemáticas Discretas (TC1003) y Estructura de Datos (TC1018).

  \vspace*{.15in}

  \noindent \textbf{Índice analítico del curso:}
  El curso está dividido en tres módulos---Lenguajes regulares, lenguajes libres de contexto y lenguajes recursivamente numerables.

  \begin{center} 
  \begin{minipage}{5in}
  \begin{flushleft}
  {\large I. Lenguajes regulares} \\[2ex]
  \topic ~Conceptos preliminares \\
  \subtopic ~Conceptos matemáticos \\
  \subtopic ~Lenguajes formales \\
  \topic ~Teoría de los lenguajes \\
  \subtopic ~Máquinas de estados finitos \\
  \subtopic ~Autómatas finitos determinísticos \\
  \subtopic ~Autómatas finitos no determinísticos \\
  \topic ~Lenguajes regulares \\
  \subtopic ~Expresiones regulares \\
  \subtopic ~Gramáticas regulares \\
  \subtopic ~Análisis Léxico \\[2.5ex]
  {\large II. Lenguajes libres de contexto }\\[2ex]
  \topic ~Lenguajes libres de contexto \\
  \subtopic ~Jerarquía de Chomsky \\
  \subtopic ~Gramáticas libres de contexto \\
  \subtopic ~Propiedades de las gramáticas \\
  \topic ~Análisis sintáctico \\
  \subtopic ~Análisis sintáctico descendente \\
  \subtopic ~Análisis sintáctico ascendent e\\[2.5ex]
  {\large III. Lenguajes recursivamente numerables} \\[2ex]
  \topic ~Tópicos avanzados \\
  \subtopic ~Máquinas de Turing \\
  \subtopic ~Decidibilidad \\
  \subtopic ~Computabilidad
  \end{flushleft}
  \end{minipage}
  \end{center}

  \vspace*{.15in}
  \noindent\textbf{Política de evaluación:}
  Tareas y Quizzes (40\%), Examen 1 (20\%), Examen 2 (20\%), Examen Final (20\%).
  La suma será posteriormente multiplicada por 95\% debido a que el 5\% restante corresponde a la Semana i.

  \textbf{Recuerda que lo que se evalúa es tu desempeño, no tu persona}.
  En los exámenes, evaluamos lo que escribes, no lo que piensas ni lo que sabes.
  Las evaluaciones---a pesar de sus limitaciones---son un elemento básico para que la institución pueda certificar, al final de tu carrera, que asististe a los cursos y que posees los conocimientos, habilidades, actitudes y valores de un profesionista.

  \vskip.15in
  \noindent\textbf{Fechas importantes:}
  Dependiendo del departamento, las fechas de los exámenes podrían cambiar.
  Si ése es el caso, serás notificado con tiempo.
  Mientras tanto, éstas son las fechas tentativas:

  \begin{center} \begin{minipage}{3.8in}
    \begin{flushleft}
    Examen 1 \dotfill ~10 de septiembre \\
    Semana \textit{i} \dotfill ~18 al 24 de septiembre\\
    Examen 2 \dotfill ~22 de octubre \\
    Examen final \dotfill ~3 de diciembre
    \end{flushleft}
  \end{minipage}
  \end{center}

  \vskip.15in
  \noindent\textbf{Políticas del curso:}  
  \begin{itemize}
  \item Se sugiere que al inicio del semestre navegues por la página del curso, el curso en Blackboard y que revises los contenidos, su forma de evaluación y las reglas. \textbf{El desconocimiento de una regla que fue dada a conocer no justifica su omisión}.
  \item Verifica que tu correo del Tec esté funcionando, ya que será utilizado como medio oficial de comunicación. \textbf{El hecho de que no tengas acceso a tu correo no es justificación para no llevar a cabo una entrega}.
  \item Las tareas serán entregadas por el medio especificado y antes de la fecha límite. En caso de que no puedas entregar una tarea a tiempo, es probable que puedas entregarla de otro modo aunque con una penalización. Acércate al profesor.
  \item Las soluciones a las tareas deberán ser entregadas en limpio y en formato digital (\textit{typeset}), en un archivo PDF y subidas a la plataforma. Evita subir fotos o escaneos de trabajos a mano.
  \item Para tareas en las que la solución sea de más de un archivo, sube una carpeta comprimida en formato ZIP. 
  \item Las soluciones a las tareas con un puntaje casi perfecto podrían ser consideradas como soluciones oficiales de dicha tarea y subidas a la plataforma. En caso de ser así, el estudiante ganará puntos extras.
  \item Si hay algo que crees necesario que deba tomar en cuenta al momento de calificar tu tarea, escríbelo en los comentarios de la plataforma, o bien crea un archivo de texto con el nombre \texttt{README} y escribe ahí tu mensaje e inclúyelo en el archivo comprimido. No envíes estos mensajes por correo.
  \item Puedes discutir los problemas de la tarea con otros estudiantes, pero recuerda que debes subir un archivo escrito por ti (y los miembros de tu equipo, según sea el caso). En trabajos colaborativos, un solo entregable basta, pero asegúrate de incluir a todos los integrantes.
  \item Las aclaraciones de los alumnos respecto a calificaciones de actividades y exámenes sólo podrán hacerse dentro de las dos semanas siguientes a la publicación de las calificaciones respectivas.
  \item Los comentarios o aclaraciones que haga el profesor durante la aplicación de un examen son usados por el alumno bajo su propia responsabilidad, si considera que le son de utilidad, y en ningún momento podrán usarse como argumento para discutir la calificación de algún problema del examen.
  \item En caso de que un alumno no pueda presentar un examen por causas de fuerza mayor, deberá conseguir un visto bueno de la dirección de carrera, quien mandará un correo u otro documento equivalente al profesor. El profesor no revisará directamente comprobantes médicos o documentos de esa índole.
  \end{itemize}

  \vskip.15in
  \noindent\textbf{Políticas de clase:}  
  \begin{itemize}
  \item La entrada al salón de clases es a la hora especificada. Una vez iniciada la clase, se procederá a tomar asistencia. Después de 15 minutos ya no podrás ingresar al salón.
  \item Las actividades desarrolladas durante una sesión a la que no asististe no se repondrán.
  \item Los exámenes podrán reponerse con el visto bueno del director de carrera, quien deberá enviar una notificación al profesor (un correo, por ejemplo).
  \item Es tu responsabilidad ponerte al tanto de lo acontecido en la clase durante tu ausencia.
  \item Recuerda que debes cumplir con un porcentaje mínimo de asistencia. No faltes a clase si no es absolutamente necesario.
  \item Sé cortés al usar dispositivos móviles. Asegúrate de que tu celular está apagado o en silencio. Si recibes una llamada o mensaje importante durante una sesión, podrás atenderlo fuera del salón de clases.\footnotemark
  \item Si tienes que usar tu computadora, utilízala sin sonido o con auriculares.\footnotemark[\value{footnote}]
  \end{itemize}

  \footnotetext{El problema principal no es que tú no te concentres, sino que podrías perjudicar al ambiente en que se desenvuelven tus demás compañeros. Sé considerado.}

  \vskip.15in
  \noindent\textbf{Integridad académica:}
  ``Se entiende por \textit{integridad académica} el actuar honesto, comprometido, confiable, responsable, justo, respetuoso con el aprendizaje, la investigación y la difusión de la cultura''. En este curso, pedimos que los alumnos y el profesor se comporten siguiendo estos principios.
  \\[2ex]
  {\color{darkred}{\Large \HandRight}} ~\textbf{La copia en exámenes o tareas va en forma flagrante contra dicha \textit{integridad académica}, y será penalizada}.
  Una cosa es \textit{hacer la tarea juntos} y otra muy distinta es compartir resultados y documentos sin hacer referencia formal de ello.\\[2ex]
  {\color{darkred}{\Large \HandRight}} ~El nuevo reglamento académico establece que el profesor asignará una \textbf{calificación reprobatoria} a la actividad, examen, período parcial o final. \textbf{La calificación reprobatoria asignada por el profesor será inapelable}, y a esta sanción se sumarán aquellas otras que el Comité de Integridad Académica del Campus determine pertinentes.

  %%%%%% END 
\end{document} 