\documentclass[spanish, handout]{beamer}

\usepackage[utf8]{inputenc}
\usepackage[spanish, mexico]{babel}
\usepackage{hyperref}
\usepackage{xcolor}
\usepackage{color}
\usepackage{ragged2e}
\usepackage{tikz}
\usepackage{mathrsfs}
\usepackage{textcomp}
\usepackage{booktabs}

\usetikzlibrary{arrows, automata, positioning, fit, shapes.geometric, backgrounds}
\tikzset{%
    automaton/.style={->, >=stealth', shorten >=1pt, auto, semithick, initial text=$ $}
}

% \usepackage{courier}
% \usepackage{subfigure}
% \usepackage{enumerate}
% \usepackage{algorithmic}
% \usepackage{algorithm}


% \usepackage{listings}
% \usepackage{lstlinebgrd}

\newcommand\blfootnote[1]{%
\begingroup
\renewcommand\thefootnote{}\footnote{#1}%
\addtocounter{footnote}{-1}%
\endgroup
}

\usetheme{Boadilla}
%\usecolortheme{default}
\usefonttheme[onlymath]{serif}
\useoutertheme{infolines}

% Sets the templates

\definecolor{navyblue}{RGB}{0, 0, 128}

\setbeamertemplate{navigation symbols}{}
%\setbeamertemplate{headline}{}
\setbeamertemplate{footline}[frame number]
\setbeamertemplate{bibliography item}[text]
\setbeamertemplate{theorems}[numbered]

\setbeamercolor{title}{fg = navyblue, bg = white}
\setbeamercolor{frametitle}{fg = navyblue, bg = white}
\setbeamercolor{structure}{fg = navyblue}
\setbeamercolor{button}{fg = white,bg = navyblue}

\setbeamercovered{transparent}

\title{Gramáticas Libres de Contexto y Autómatas de Pila}
\subtitle{Matemáticas Computacionales \\ (TC2020)}
\author{
\texorpdfstring{
\begin{center}
    M.C. Xavier Sánchez Díaz \\
    \href{mailto:sax@itesm.mx}{\texttt{sax@itesm.mx}}
\end{center}
}
{M.C. Xavier Sánchez Díaz}
}

\institute[Tecnológico de Monterrey]{\includegraphics[scale=0.5]{../img/logo}}
\date{}

\begin{document}


\setlength{\rightskip}{0pt}

\begin{frame}[plain]
\titlepage
\end{frame}

% \begin{frame}{Tabla de contenidos}
% \tableofcontents
% \end{frame}

\section{Compiladores LL y LR}

\begin{frame}{Crecimiento y no-determinismo en GLCs}{Compiladores LL y LR}
    Recordemos una GLC, por ejemplo, para generar palabras del lenguaje $\{a^nb^n\}$: \pause

    \bigskip

    \begin{enumerate}
        \item $S \to aSb$ \pause
        \item $S \to \varepsilon$ \pause
    \end{enumerate}

    \bigskip

    ¿Hacia dónde \textbf{crece} la palabra? \pause
    Deriva usando $w = \langle 1, 1, 1, 2 \rangle$. \pause

    Sin usar una secuencia dada, al momento de derivar una palabra usando las reglas de la gramática, tenemos que \alert{decidir} entre cuál de las reglas disponibles usar.
    
\end{frame}

\begin{frame}{Crecimiento y no-determinismo en GLCs}{Compiladores LL y LR}

    \begin{enumerate}
        \item $S \to aSb$ \pause
        \item $S \to \varepsilon$ \pause
    \end{enumerate}

    \bigskip

    ¿Cómo sabemos qué regla utilizar? \pause

    \bigskip

    Nosotros podemos decidir porque sabemos a la palabra a la que queremos llegar. Un compilador no puede hacer eso, por lo que necesita de algún otro mecanismo para \textbf{prever} qué regla utilizar. \pause

    Generar la función de transición completa para un autómata de pila para $\{a^nb^n\}$.

\end{frame}

\begin{frame}{Crecimiento y no-determinismo en GLCs}{Compiladores LL y LR}

    \begin{center}
        \begin{table}[H]
            \begin{tabular}{@{}llll@{}}
            \toprule
            Tape & Pop & Move & Push \\ \midrule
            $aabb$ & \$ & $R$ & $A$ \\
            $abb$ & $A$ & $R$ & $AA$ \\
            $bb$ & $A$ & $R$ & $\varepsilon$ \\
            $b$ & $A$ & $R$ & $\varepsilon$ \\
            $\blacksquare$ & \$ & $N$ & $\varepsilon$ \\ \bottomrule
            \end{tabular}
        \end{table}
    \end{center}

    \bigskip

    Hay muchas reglas que no estamos utilizando, y nos guiamos por el hecho de ver hacia \textit{adelante}. \pause    
    Sabiendo que busco $aabb$, entonces lo más lógico es usar $S \to aSb$ dos veces, para poder \textit{finalizar} con $S\to \varepsilon$. \pause
    
    \bigskip

    Este tipo de compiladores, que leen de izquierda a derecha, y que tienen una ventana de ``predicción'' se conocen como \textit{Left to right Leftmost derivation}, o \alert{LL} por sus siglas en inglés.

\end{frame}

\begin{frame}{Crecimiento y no-determinismo en GLCs}{Compiladores LL y LR}

    Por el contrario, hay compiladores que leen \textit{al revés}, de derecha a izquierda (con respecto a las reglas de la gramática, primero $aSB$ y luego $S \to$).
    De este modo, el árbol de derivación se recorre hacia atrás. \pause

    \bigskip

    Este tipo de compiladores se conoce como \textit{Left to right Rightmost derivation}, o \alert{LR} por sus siglas en inglés. \pause

    \bigskip

    ¿Pero cómo convertimos de una GLC a un AP y viceversa?

\end{frame}

\begin{frame}{Múltiples maneras en la literatura}{Compiladores LL y LR}
    Como en el caso pasado, distintas personas lo manejan de distinta manera:

    \begin{enumerate}
        \itemsep2.5ex
        \item Brena (2003) diferencia entre el tipo de compilador, y convierte una gramática de cierto tipo a un AP utilizando un algoritmo que depende del tipo de compilador. \pause
        \item Ullman (1979), uno de los autores del libro más usado para este curso a nivel mundial, utiliza otra versión, sólo para compiladores LL y es la que utilizaremos.
    \end{enumerate}
\end{frame}

\section{Conversión GLC a AP}

\begin{frame}{Conversión GLC a AP}
    Sea $L$ un lenguaje de una GLC $G$. \pause

    \bigskip

    Para construir un AP $M$ que acepte $L$, hay que considerar lo siguiente: \pause

    \begin{itemize}
        \item $M$ tiene \textbf{un estado} $q$ \pause
        \item El \textbf{alfabeto de la cinta} de $M, \Sigma$ es igual a las terminales de $G$ \pause
        \item El \textbf{alfabeto de la pila}  de $M, \Gamma \subseteq V \cup \Sigma$ \pause
        \item El \textbf{símbolo inicial (de la pila)} de $M$ sigue siendo $\$$ (como antes) \pause
    \end{itemize}

    La \textbf{función de transición} puede calcularse con respecto a eso (como antes) \pause

    \bigskip

    Las \textbf{condiciones de aceptación} son:

    \begin{itemize}
        \item La pila está vacía (como antes) \pause
        \item La cinta llegó al final (como antes)
    \end{itemize}
    
\end{frame}

\begin{frame}{Ejemplo: $\{a^nb^n\}$}{Formalización y diseño de APs}

    $M = G$:

    \begin{columns}
        \begin{column}{0.4\textwidth}
            \begin{enumerate}
                \item $S \to aSb$
                \item $S \to ab$
            \end{enumerate}

            \begin{itemize}
                \item $Q = \{q\}$
                \item $\Sigma = \{a, b\}$
                \item $\Gamma = \{\$, S\}$
                \item $\delta =$
                \begin{itemize}
                    \item $((q,a,\$)(q,\$S))$
                    \item $((q,a,S)(q,SS))$
                    \item $((q,b,S)(q,\varepsilon))$
                    \item $((q,\square,\$),(q,\varepsilon))$
                \end{itemize}
                \item $q = q$
                \item $F = \{q\}$
            \end{itemize}
        \end{column}
        \begin{column}{0.6\textwidth}
            \begin{center}
                \begin{tikzpicture}[node distance=3.7cm, automaton]
                    \node[state, initial, accepting] (q) {$q$};
        
                    \path
                    (q) edge [loop above] node {$(a,\$,\$S), (a, S, SS)$} (q)
                    (q) edge [loop below] node {$(b,S, \varepsilon), (\square,\$, \varepsilon)$} (q);
                \end{tikzpicture}
            \end{center}
        \end{column}
    \end{columns}
\end{frame}


% \section*{Referencias}
% \begin{frame}[t]{Referencias}
% \nocite{bibID01}
% \nocite{bibID02}

% \bibliographystyle{IEEE}
% \bibliography{biblio}
% \end{frame}

\end{document}