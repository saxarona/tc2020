\documentclass[spanish]{beamer}

\usepackage[utf8]{inputenc}
\usepackage[spanish, mexico]{babel}
\usepackage{hyperref}
\usepackage{xcolor}
\usepackage{color}
\usepackage{ragged2e}
\usepackage{tikz}
\usepackage{mathrsfs}
\usepackage{textcomp}
\usepackage{booktabs}

\usetikzlibrary{arrows, automata, positioning, fit, shapes.geometric, backgrounds}
\tikzset{%
    automaton/.style={->, >=stealth', shorten >=1pt, auto, semithick, initial text=$ $}
}

% \usepackage{courier}
% \usepackage{subfigure}
% \usepackage{enumerate}
% \usepackage{algorithmic}
% \usepackage{algorithm}


% \usepackage{listings}
% \usepackage{lstlinebgrd}

\newcommand\blfootnote[1]{%
\begingroup
\renewcommand\thefootnote{}\footnote{#1}%
\addtocounter{footnote}{-1}%
\endgroup
}

\usetheme{Boadilla}
%\usecolortheme{default}
\usefonttheme[onlymath]{serif}
\useoutertheme{infolines}

% Sets the templates

\definecolor{navyblue}{RGB}{0, 0, 128}
\definecolor{crimson}{RGB}{128, 16, 0}

\setbeamertemplate{navigation symbols}{}
\setbeamertemplate{headline}{}
\setbeamertemplate{title page}[default][colsep=-4bp,rounded=true]
\setbeamertemplate{footline}[frame number]
\setbeamertemplate{bibliography item}[text]
\setbeamertemplate{theorems}[numbered]

\setbeamercolor{title}{fg=navyblue, bg=white}
\setbeamercolor{frametitle}{fg=navyblue, bg=white}
\setbeamercolor{structure}{fg=navyblue}
\setbeamercolor{button}{fg=white,bg=navyblue}

\setbeamercovered{transparent}

\title{Máquinas de Turing}
\subtitle{Matemáticas Computacionales \\ (TC2020)}
\author{
\texorpdfstring{
\begin{center}
    M.C. Xavier Sánchez Díaz \\
    \href{mailto:mail@tec.mx}{\texttt{mail@tec.mx}}
\end{center}
}
{M.C. Xavier Sánchez Díaz}
}

\institute[Tecnológico de Monterrey]{\includegraphics[scale=0.5]{../img/logo}}
\date{}

\begin{document}


\setlength{\rightskip}{0pt}

\begin{frame}[plain]
\titlepage
\end{frame}

% \begin{frame}{Tabla de contenidos}
% \tableofcontents
% \end{frame}

\section{Lo que hemos visto hasta ahora}

\begin{frame}{Lenguajes}{Lo que hemos visto hasta ahora}
    Un \textbf{lenguaje} es un conjunto de \textbf{palabras}. \pause

    \bigskip

    ¿Cómo \textbf{demostramos} que un lenguaje es \alert{regular}? \pause

    \bigskip

    ¿Cómo \textbf{demostramos} que un lenguaje es \alert{libre de contexto}?
    
\end{frame}

\begin{frame}{Las máquinas y sus lenguajes}{Lo que hemos visto hasta ahora}

    Los \textbf{Autómatas Finitos}, las \textbf{expresiones regulares} y las \textbf{gramáticas regulares} sirven para representar \alert{lenguajes regulares}. \pause

    \bigskip

    Los \textbf{Autómatas de Pila} y las \textbf{gramáticas libres de contexto} sirven para representar \alert{lenguajes libres de contexto}. \pause

    \bigskip

    Sin embargo, hay otros lenguajes que no podemos representar con ninguna de estas herramientas.

\end{frame}

\begin{frame}{Las máquinas y sus lenguajes}{Lo que hemos visto hasta ahora}
    
    Por ejemplo, intentemos representar $\{a^nb^nc^n | n \geq 0\}$ con un AP: \pause

    \bigskip

    \begin{itemize}
        \item Por cada $a$ ponemos un contador en la pila \pause
        \item Por cada $b$ quitamos un contador de la pila \pause
        \item  No podemos contar las $c$s. \pause
    \end{itemize}

    O si cambiamos el orden o usamos más contadores, simplemente no es posible hacerlo con un autómata de pila.

\end{frame}

\section{Turing y su máquina}

\begin{frame}{Un mejor uso de la memoria}{Turing y su máquina}
    Claramente el problema que tenemos es de memoria: no tenemos cómo recordar las $c$s. \pause

    \bigskip

    En un AP sólo leemos el tope de la pila. ¿Qué pasaría si leyéramos cualquier parte de la misma? \pause

    \bigskip

    ¿Cuál es el límite de la pila? ¿Cuántos símbolos puede guardar?
\end{frame}

\begin{frame}{Una nueva máquina}{Turing y su máquina}
     Una \alert{máquina de Turing} (MT) soluciona estos problemas, \textit{uniendo} el \textit{input} y la memoria. Ahora nuestro \texttt{Autómata Reloaded} tiene los siguientes elementos: \pause

     \begin{itemize}
        \itemsep1.5ex
        \item Un \alert{conjunto de estados de control} que es \textbf{finito} \pause
        \item Una \alert{cinta} \textbf{infinita} que utiliza como su \textbf{memoria} \pause
        \item Un \alert{cabezal en la cinta} que puede \textbf{leer} y \textbf{escribir} en una celda a la vez. \pause
     \end{itemize}

     En cada \textit{paso} del cómputo, la máquina \pause
     \begin{itemize}
         \itemsep1.5ex
         \item Escribe un símbolo en la celda donde está el cabezal, \pause
         \item cambia de estado, y \pause
         \item mueve el cabezal.
     \end{itemize}
\end{frame}

\begin{frame}{Terminología y lenguaje}{Turing y su máquina}
    Sea $M$ una máquina de Turing:

    \bigskip

    \begin{itemize}
        \itemsep1.5ex
        \item $M$ \alert{acepta} una palabra $w$ si entra al estado de aceptación cuando se lee $w$. En este caso, $M$ \textbf{termina}. \pause
        \item $M$ \alert{rechaza} una palabra $w$ si entra al estado de rechazo cuando se lee $w$. En este caso, $M$ \textbf{termina}. \pause
        \item $M$ entra en \alert{loop} con una palabra $w$ si al leer $w$ no entra ni al estado de aceptación ni al de rechazo. En este caso, $M$ \textbf{no termina}. \pause
    \end{itemize}

    ¿Qué puede pasar si \textit{no se acepta}? \pause ¿Qué puede pasar si \textit{no se rechaza}?
\end{frame}

\begin{frame}{Terminología y lenguaje}{Turing y su máquina}
    El lenguaje de una máquina de Turing $M$, denotado con $L(M)$ o $\mathfrak{L}(M)$ es el conjunto de todas las palabras que $M$ acepta: \pause

    $$\mathfrak{L}(M) = \{w \in \Sigma^* | M \text{ acepta } w\}$$

    Un lenguaje es \alert{reconocible} si y solo si es el lenguaje de alguna máquina de Turing. \pause

    \bigskip
    
    ¿Existen más lenguajes?
    
\end{frame}

\begin{frame}{Turing y el Entscheidungsproblem}{Turing y su máquina}

    En 1928, David Hilbert y Wilhelm Ackermann---ambos matemáticos alemanes---propusieron un problema que llevó cerca de 8 años resolver: \pause

    \bigskip
    
    \begin{block}{El Entscheidungsproblem}
        ¿Existe algún algoritmo que tome como \textit{input} una proposición de lógica de primer orden, y diga si es o no universalmente válido a partir de sus axiomas?
    \end{block}
    
\end{frame}

\begin{frame}{Turing y el Entscheidungsproblem}{Turing y su máquina}

    En 1936, Alonzo Church---matemático estadounidense---lanza una publicación definiendo el concepto de ``calculabilidad efectiva'' mediante el cálculo lambda. Esto da nombre a las funciones lambda en lenguajes de programación. \pause

    \bigskip

    El mismo año, Alan Turing---matemático inglés---publica su trabajo \textit{On Computable Numbers, with an application to the Entscheidungsproblem}\footnote{Es \textit{altamente} probable que tengan que leer el paper para el examen...}, donde propone la MT y delimita todo aquello que puede ser ``computable''. \pause
    
\end{frame}

\begin{frame}{¿Qué puede hacer la máquina de Turing?}{Turing y su máquina}
    Aunque dos personas dieron la misma respuesta al \textit{Entscheidungsproblem}---no existe algoritmo alguno para responder---usualmente nos quedamos con la versión de Turing, pues es más contundente al definir lo que es un algoritmo: \pause

    \bigskip

    \begin{theorem}
        Si lo puede hacer una máquina de Turing, entonces hay un algoritmo para ello.
    \end{theorem} \pause

    \bigskip

    Lo que significa que la MT puede hacer operaciones aritméticas, trabajar con listas, aceptar palabras y hacer \textbf{funciones}. \pause

\end{frame}

\section{Formalidades y ejemplos}

\begin{frame}{Definición formal}{Formalidades y ejemplos}
    De nuevo, hay muchas maneras de abordar la definición formal.
    Por ello, nos enfocaremos en usar lo más parecido a lo que hayamos usado. \pause

    \bigskip

    Primero, recordemos que una MT \textit{para reconocer palabras de un lenguaje} tiene
    \begin{itemize}
        \item Un conjunto de estados. \pause
        \item Un input. \pause
        \item Una cinta que usa como memoria. \pause
        \item Un estado inicial. \pause
        \item Un estado de aceptación. \pause
        \item Un estado de rechazo. \pause
        \item Una función de transición que define cómo pasar de un estado a otro, dadas ciertas condiciones.
    \end{itemize}
\end{frame}

\begin{frame}{Definición formal}{Formalidades y ejemplos}
    \begin{block}{Definición formal de una máquina de Turing}
        Una máquina de Turing $M$ para reconocer palabras de un lenguaje es una tupla de la forma $M = (Q, \Sigma, \Gamma, \delta, q, a, r)$ donde: \pause

        \begin{itemize}
            \item $Q$ es un conjunto finito de \textbf{estados}, \pause
            \item $\Sigma$ es el \textbf{alfabeto del input}, donde $\square \not \in \Sigma$, \pause
            \item $\Gamma$ es el \textbf{alfabeto de la cinta}, donde $\square \in \Gamma$ y $\Sigma \subseteq \Gamma$, \pause
            \item $q \in Q$ es el \textbf{estado inicial}, \pause
            \item $a \in Q$ es el \textbf{estado de aceptación}, \pause
            \item $r \in Q$ es el \textbf{estado de rechazo}, y \pause
            \item $\delta$ es la función de transición.
        \end{itemize}
    \end{block}
\end{frame}

\begin{frame}{Definición formal}{Formalidades y ejemplos}
    La función de transición $\delta$ es una función de la forma:
    
    $$\delta : \alert<3>{Q} \times \alert<4>{\Gamma} \to \alert<5>{Q} \times \alert<6>{\Gamma} \times \alert<7>{\{L,N,R\}}$$ \pause

    \begin{exampleblock}{Ejemplo}
        Podemos escribir $\alert<3>{q_0} \alert<4>{1} \to \alert<5>{q_1} \alert<6>{1} \alert<7>{R}$ que significaría que: \pause

        \begin{itemize}
            \item Estando en el estado $\alert<3>{q_0}$, \pause
            \item y al leer un $\alert<4>{1}$ en la cinta \pause
        \end{itemize}
        Entonces la MT
        \begin{itemize}
            \item cambia al estado $\alert<5>{q_1}$, \pause
            \item escribe un $\alert<6>{1}$ en la celda actual, y \pause
            \item mueve el cabezal hacia la derecha ($\alert<7>{R}$ight)
        \end{itemize}
        
    \end{exampleblock}
    
\end{frame}

\begin{frame}{Definición formal}{Formalidades y ejemplos}
    La verdad es que... \pause

    \bigskip
    
    ...es más fácil pensarlo como \textbf{flechas} que van \textbf{de un estado} $q0$ \textbf{a otro estado} $q_1$ de la forma $x \to y, D$--- si \alert{lees $x$}, \alert{escribes $y$} y te \alert{mueves hacia $D$}: \pause

    \bigskip

    \begin{center}
        \begin{tikzpicture}[node distance=3.7cm, automaton]
            \node[state] (q0) {$q_0$};
            \node[state] (q1) [right of=q0] {$q_1$};

            \path
            (q0) edge node {$\square \to 1, R$} (q1);
        \end{tikzpicture}
    \end{center}
\end{frame}

\begin{frame}{Configuración inicial}{Formalidades y ejemplos}
    Antes de comenzar el cómputo, la máquina de Turing debe estar en una configuración específica. En esta configuración:

    \bigskip

    \begin{enumerate}
        \item La cinta está vacía, es decir que tiene sólo símbolos $\square$ en ella. \pause
        \item La palabra de entrada \textbf{se copia} a algún lugar en la cinta. \pause
        \item El cabezal se mueve al \textbf{inicio} de la palabra de entrada.
    \end{enumerate}
    
\end{frame}

\begin{frame}{Ejemplos}{Formalidades y ejemplos}
    \begin{center}
        {\Huge ¿Y los ejemplos?}
    \end{center}
    \begin{center}
        {\Large Se irán agregando con el tiempo.}

        \bigskip

        Mientras, hay que revisar otros enfoques:

        \bigskip
        
        \begin{enumerate}
            \item Slides 26 de Ramón Brena/Santiago Conant.
            \item Slides 18, 19, 20 y 21 de Keith Schwarz.
            \item Capítulos 4 de Maheshwari y Smid.
        \end{enumerate}
    \end{center}

\end{frame}


% \section*{Referencias}
% \begin{frame}[t]{Referencias}
% \nocite{bibID01}
% \nocite{bibID02}

% \bibliographystyle{IEEE}
% \bibliography{biblio}
% \end{frame}

\end{document}