\documentclass[spanish]{beamer}

\usepackage[utf8]{inputenc}
\usepackage[spanish, mexico]{babel}
\usepackage{hyperref}
\usepackage{xcolor}
\usepackage{color}
\usepackage{ragged2e}
\usepackage{tikz}
\usepackage{mathrsfs}

\usetikzlibrary{arrows, automata}

% \usepackage{courier}
% \usepackage{subfigure}
% \usepackage{enumerate}
% \usepackage{algorithmic}
% \usepackage{algorithm}


% \usepackage{listings}
% \usepackage{lstlinebgrd}

\newcommand\blfootnote[1]{%
  \begingroup
  \renewcommand\thefootnote{}\footnote{#1}%
  \addtocounter{footnote}{-1}%
  \endgroup
}

\usetheme{Boadilla}
\usefonttheme[onlymath]{serif}

% Sets the templates
\definecolor{navyblue}{RGB}{0, 0, 128}

\setbeamertemplate{navigation symbols}{} 
\setbeamertemplate{headline}{}
\setbeamertemplate{footline}[frame number]
\setbeamertemplate{bibliography item}[text]
\setbeamertemplate{theorems}[numbered]

\setbeamercolor{title}{fg=navyblue, bg=white}
\setbeamercolor{frametitle}{fg=navyblue, bg=white}
\setbeamercolor{structure}{fg=navyblue}
\setbeamercolor{button}{fg=white,bg=navyblue}

\setbeamercovered{transparent}

\title{Lenguajes y Modelado}
\subtitle{Matemáticas Computacionales \\ (TC2020)}
\author{
    \texorpdfstring{
        \begin{center}
            M.C. Xavier Sánchez Díaz \\
            \href{mailto:sax@tec.mx}{\texttt{sax@itesm.mx}}
        \end{center}
    }
    {M.C. Xavier Sánchez Díaz}
}

\institute[Tecnológico de Monterrey]{\includegraphics[scale=0.5]{../img/logo}}
\date{}

\begin{document}

\setlength{\rightskip}{0pt}

\begin{frame}[plain]
    \titlepage        
\end{frame}

\begin{frame}{Tabla de contenidos}
    \tableofcontents
\end{frame}

\section{Lenguajes}
\label{sec:langs}

\subsection{Elementos básicos de un lenguaje}
\label{subsec:lang-elements}

\begin{frame}{¿Qué es un lenguaje?}{Elementos básicos de un lenguaje}

    Según la RAE, un lenguaje es un conjunto de signos y reglas que permite la comunicación (con una computadora). A nivel matemático, usamos otra definición: \pause

    \bigskip

    \begin{definition}
        Un \alert{lenguaje} es un conjunto de \textbf{palabras}.
    \end{definition} \pause

    \bigskip

    \begin{exampleblock}{Ejemplo de lenguaje}
        \(L = \{hola, pueblo\}\)
    \end{exampleblock}

\end{frame}

\begin{frame}{¿Qué es una palabra?}{Elementos básicos de un lenguaje}
    
    La RAE define a \textbf{palabra} como una unidad lingüística dotada generalmente de significado, que se separa de las demás mediante pausas potenciales en la pronunciación y blancos en la escritura. \pause

    \bigskip

    \begin{definition}
        Una \alert{palabra} es una sucesión de símbolos de algún \textbf{alfabeto}.
    \end{definition} \pause

    \bigskip

    \begin{exampleblock}{Ejemplos de palabras}
        Tanto $hola$ como $pueblo$ son palabras.
    \end{exampleblock}
\end{frame}

\begin{frame}{¿Qué es un alfabeto?}{Elementos básicos de un lenguaje}

    \begin{definition}
        Un \alert{alfabeto} es un conjunto finito no vacío de \textbf{símbolos}.
    \end{definition} \pause

    \bigskip

    \begin{exampleblock}{Ejemplo de alfabeto}
        $ \Sigma = \{ a, b, c, \dots , z \} $ es un alfabeto 
    \end{exampleblock} \pause

    \bigskip

    \begin{definition}
        Un \alert{símbolo} es una unidad atómica de información.
    \end{definition} \pause

    \bigskip

    \begin{exampleblock}{Ejemplos de símbolos}
        $a, b, e, h, l, o, p, u$ son todos símbolos del \textbf{alfabeto} $\Sigma$.
    \end{exampleblock}
    
\end{frame}

\begin{frame}{Recapitulación}{Elementos básicos de un lenguaje}

    Es decir, los \textbf{símbolos} $h, o, l, a, p, u, e, b$ son elementos del \textbf{alfabeto} $\Sigma = \{a, b, c, \dots , z\}$. \pause
    
    \bigskip

    Dos palabras que podemos formar con $\Sigma$ son \textit{hola} y \textit{pueblo}. \pause

    \bigskip

    Podemos agrupar \textit{hola} y \textit{pueblo} en un lenguaje: $L = \{hola, pueblo\}$
\end{frame}

\subsection{Operaciones con lenguajes}
\label{subsec:lang-ops}

\begin{frame}{¿Qué podemos hacer con los lenguajes?}{Operaciones con lenguajes}
    Cuando dos lenguajes son definidos con respecto al mismo alfabeto, podemos aplicarles las mismas operaciones de conjuntos que ya conocemos. \pause
    
    \bigskip

    \begin{itemize}
        \itemsep1.5ex
        \item Unión \pause
        \item Intersección \pause
        \item Diferencia
    \end{itemize}

    \pause
    \bigskip

    Sin embargo también hay otras operaciones que aplican a los lenguajes (y también a las palabras y a los símbolos).
\end{frame}

\begin{frame}{Concatenación}{Operaciones con lenguajes}

    \begin{definition}
        La \alert{concatenación} de dos lenguajes $A$ y $B$ se define como
        \[AB = \{ww' : w \in A, w' \in B\}\]
    \end{definition} \pause

    \bigskip

    Es decir, $AB$ es el conjunto de todas las palabras obtenidas tomando una palabra arbitraria $w$ en $A$ y otra palabra arbitraria $w'$ en $B$, y juntándolas. \pause

    \bigskip

    \begin{exampleblock}{Ejemplo de concatenación}
        \[A = \{hola, chao\} \quad B = \{pueblo,  mundo\}\]
        \[AB = \{holapueblo, holamundo, chaopueblo, chaomundo\}\]
    \end{exampleblock}
\end{frame}

\begin{frame}{Cerradura de Kleene}{Operaciones con lenguajes}

    \begin{definition}
        La \alert{Kleene Star} (también llamada \textbf{estrella de Kleene}) de un lenguaje $A$ se define como
        \[A^{*} = \{u_1 u_2 u_3 \dots u_k : k \geq 0, u_i \in A, i = 1, 2, 3, \dots , k\}\]
    \end{definition} \pause
    
    \bigskip

    En otras palabras, la concatenación de \textbf{todas} las palabras \textbf{posibles} en $A$, incluyendo la \alert{palabra vacía} (de longitud 0, que representamos con $\varepsilon$). \pause

    \bigskip

    \begin{exampleblock}{Ejemplo de Kleene Star}
        \[A* = \{\varepsilon, hola, ola, holaola, holahola, olaolaola, olaholaolahola, \dots \}\]
    \end{exampleblock}
\end{frame}

\begin{frame}{Kleene Plus}{Operaciones con lenguajes}
    Existe una variante de la cerradura de Kleene llamada \alert{Kleene Plus}: \pause

    \bigskip

    \begin{definition}
        \[A^{+} = \{u_1 u_2 u_3 \dots u_k : k \geq 1, u_i \in A, i = 1, 2, 3, \dots , k\}\]
    \end{definition} \pause

    \bigskip

    Es decir, $A^{+} = A^{*} - \{\varepsilon\}$
\end{frame}

\section{Modelado con Autómatas}
\label{sec:modeling}

\begin{frame}{¿Qué se puede modelar?}{Modelado con autómatas}

    \begin{itemize}
        \itemsep1.5ex
        \item Procesos por medio de \textbf{estados} y \textbf{eventos} o \textbf{transiciones}. \pause
        \item Los estados son situaciones por las que el proceso atraviesa. Algunos de los estados son transitorios. \pause
        \item Los eventos son \textbf{acciones instantáneas} que provocan cambios en el estado del proceso modelado.
    \end{itemize}

\end{frame}

\begin{frame}{Ejemplo}{Modelado con autómatas}
    \begin{center}
        \begin{tikzpicture}[->,>=stealth',shorten >=1pt,auto,node distance=4cm, semithick]
            \tikzstyle{every state}=[fill=gray,draw=none]
    
            \node[initial,state] (A) {soltera};
            \node[state] (B) [right of=A] {casada};
            \node[state] (C) [below left of=B]{divorciada};
            \node[state] (D) [below right of=B] {viuda};
    
            \path (A) edge [bend left] node {boda} (B)
                  (B) edge [bend right] node {divorcio} (C)
                      edge [bend right] node [above right]{pareja$^{\dag}$} (D)
                  (C) edge [bend right] node [below right] {boda} (B)
                  (D) edge [bend right] node [above right] {boda} (B);
        \end{tikzpicture}
    \end{center}
    \blfootnote{Ejemplo de \texttt{sconant}}
\end{frame}

\begin{frame}{Notación de autómatas}{Modelado con autómatas}
    \begin{definition}
        Un \alert{autómata finito determinista} (AFD) es una \textbf{quíntupla} de la forma
        \[M = (Q, \Sigma, \delta, q, F)\] \pause
        \begin{itemize}
            \item $Q$ es un \textbf{conjunto de estados} que es finito, \pause
            \item $\Sigma$ es el \textbf{alfabeto} aceptado, \pause
            \item $\delta : Q \times \Sigma \to Q$ es la \textbf{función de transición}, \pause
            \item $q \in Q$ es el \textbf{estado inicial}, \pause
            \item $F \subseteq Q$ es un \textbf{conjunto de estados finales}.
        \end{itemize}
    \end{definition}
\end{frame}

\begin{frame}{Determinismo}{Modelado con autómatas}
    \begin{itemize}
        \itemsep1.5ex
        \item Dada una acción, el siguiente estado será \textbf{siempre el mismo}. \pause
        \item Para cada par de estados y acciones del AFD hay \textbf{un solo estado siguiente}. \pause
        \item La función de transición está definida para \textbf{todas} las entradas posibles. \pause
        \item Hay \textbf{un solo estado inicial} pero \textbf{cualquier cantidad de estados finales}.
    \end{itemize}
    
\end{frame}

% \section*{Referencias}

% \begin{frame}[t]{Referencias}
    % \nocite{bibID01}
    % \nocite{bibID02}

    % \bibliographystyle{IEEE}
    % \bibliography{biblio}
% \end{frame}

\end{document}