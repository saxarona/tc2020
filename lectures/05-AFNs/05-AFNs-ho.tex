\documentclass[spanish, handout]{beamer}
  
  \usepackage[utf8]{inputenc}
  \usepackage[spanish, mexico]{babel}
  \usepackage{hyperref}
  \usepackage{xcolor}
  \usepackage{color}
  \usepackage{ragged2e}
  \usepackage{tikz}
  \usepackage{mathrsfs}
  
  \usetikzlibrary{arrows, automata, positioning, fit, shapes.geometric, backgrounds}
  
  \tikzset{
    stylename/.style={
      ->, %arrow type
      >=stealth', %arrow head type (bold)
      shorten >=1pt, 
      auto,
      %semithick,
      initial text=$ $, %no start text
    }
  }
  
  % \usepackage{courier}
  % \usepackage{subfigure}
  % \usepackage{enumerate}
  % \usepackage{algorithmic}
  % \usepackage{algorithm}
  
  
  % \usepackage{listings}
  % \usepackage{lstlinebgrd}
  
  \newcommand\blfootnote[1]{%
  \begingroup
  \renewcommand\thefootnote{}\footnote{#1}%
  \addtocounter{footnote}{-1}%
  \endgroup
  }
  
  \usetheme{Boadilla}
  \usefonttheme[onlymath]{serif}
  
  % Sets the templates
  \definecolor{navyblue}{RGB}{0, 0, 128}
  
  \setbeamertemplate{navigation symbols}{} 
  \setbeamertemplate{headline}{}
  \setbeamertemplate{footline}[frame number]
  \setbeamertemplate{bibliography item}[text]
  \setbeamertemplate{theorems}[numbered]
  
  \setbeamercolor{title}{fg=navyblue, bg=white}
  \setbeamercolor{frametitle}{fg=navyblue, bg=white}
  \setbeamercolor{structure}{fg=navyblue}
  \setbeamercolor{button}{fg=white,bg=navyblue}
  
  \setbeamercovered{transparent}
  
  \title{Autómatas Finitos No deterministas}
  \subtitle{Matemáticas Computacionales \\ (TC2020)}
  \author{
  \texorpdfstring{
  \begin{center}
    M.C. Xavier Sánchez Díaz \\
    \href{mailto:sax@itesm.mx}{\texttt{sax@itesm.mx}}
  \end{center}
  }
  {M.C. Xavier Sánchez Díaz}
  }
  
  \institute[Tecnológico de Monterrey]{\includegraphics[scale=0.5]{../img/logo}}
  \date{}
  
\begin{document}

\setlength{\rightskip}{0pt}

\begin{frame}[plain]
  \titlepage        
\end{frame}

\begin{frame}{Tabla de contenidos}
  \tableofcontents
\end{frame}

\section{AFNs: las diferencias}
\label{sec:afn}

\begin{frame}{Palabras como transiciones}{AFNs: las diferencias}
  Las \textbf{transiciones} se etiquetan con \alert{palabras}, no con símbolos del alfabeto. \pause
  \bigskip
  \begin{center}
    \begin{tikzpicture}[node distance=2.5cm, stylename]
      \node[initial, state] (3) {$3$};
      \node[state, accepting, right of=3] (4) {$4$};
      
      \path
      (3) edge [loop below] node {$a,b$} (3)
      (3) edge node {$aab$} (4)
      (4) edge [loop below] node {$a,b$} (4);
    \end{tikzpicture}
  \end{center} \pause
  \bigskip
  ¿Cuál es su definición formal? \pause
  \begin{align*}
    M = & (\{3,4\},\{a,b\},\\
    & \{(3,a,3), (3,b,3), (3,aab,4), (4,a,4), (4,b,4)\},\\
    &3, \{4\})
  \end{align*}
\end{frame}

\begin{frame}{Transiciones repetidas}{AFNs: las diferencias}
  Pueden existir \alert{varias transiciones con la misma palabra} a partir de un mismo estado. \pause
  
  \bigskip
  
  Un estado puede no tener transiciones. \pause
  
  \begin{center}
    \begin{tikzpicture}[node distance=2.5cm, stylename]
      \node[initial, state] (3) {$3$};
      \node[state, accepting, right of=3] (4) {$4$};
      
      \path
      (3) edge [loop below] node {$ab$} (3)
      (3) edge node {$ab$} (4);
    \end{tikzpicture}
  \end{center} \pause
  \[M = (\{3,4\}, \{a,b\}, \{(3,ab,3), (3,ab,4) \}, 3, \{4\})\]
\end{frame}

\begin{frame}{Transiciones vacías}{AFNs: las diferencias}
  Se puede pasar de un estado a otro \alert{sin consumir caracteres} de la entrada. \pause
  
  \bigskip
  
  \begin{center}
    \begin{tikzpicture}[node distance=2.5cm, stylename]
      \node[initial, state] (3) {$3$};
      \node[state, accepting, right of=3] (4) {$4$};

      \path
      (3) edge [loop above] node {$aa$} (3)
      (3) edge [loop below] node {$b$} (3)
      (3) edge node {$\varepsilon$} (4)
      (4) edge [loop above] node {$aaa$} (4)
      (4) edge [loop below] node {$b$} (4);
    \end{tikzpicture}
  \end{center} \pause
  
  \bigskip

  \begin{align*}
    M = & (\{3,4\}, \{a,b\}, \\
    & \{(3,aa,3), (3,b,3), (3,\varepsilon,4), (4,aaa,4), (4,b,4)\},\\
    & 3, \{4\})
  \end{align*}
  
\end{frame}

\begin{frame}{Aceptación de palabras}{AFNs: las diferencias}
  Una palabra es \alert{aceptada} por un AFN si existe una secuencia de transiciones a partir del estado inicial que la acepte:\pause
  \bigskip
  \begin{itemize}
    \item Que la consuma completamente. \pause
    \item Que termine en un estado final.
  \end{itemize} \pause

  \bigskip

  \begin{center}
    \begin{tikzpicture}[node distance=2cm, stylename]
      \node[initial, state] (0) {$0$};
      \node[state, above of=0] (1) {$1$};
      \node[state, accepting, right of=0, xshift=1.5cm] (2) {$2$};
      \node[state, above right of=2] (3) {$3$};
      \node[state, above left of=2] (4) {$4$};

      \path
      (0) edge [loop below] node {$b$} (0)
      (0) edge [bend right] node {$a$} (1)
      (0) edge node {$\varepsilon$} (2)
      (1) edge [bend right] node {$a$} (0)
      (2) edge [bend right] node {$a$} (3)
      (2) edge [loop below] node {$b$} (2)
      (3) edge [bend right] node {$a$} (4)
      (4) edge [bend right] node {$a$} (2);
    \end{tikzpicture}
  \end{center} \pause

  ¿Acepta $aaaaa$? ¿Qué tal $abba$?
\end{frame}

\section{Diseño de AFNs}
\label{sec:nfa-design}

\begin{frame}{Diseño por unión}{Diseño de AFNs}
  \begin{block}{Definición}
    Sean $M_1$ y $M_2$ autómatas finitos no deterministas. Si $M_1$ y $M_2$ aceptan $L_1$ y $L_2$ respectivamente, la \alert{combinación} acepta $L_1 \cup L_2$.
  \end{block} \pause

  \begin{center}
    \begin{tikzpicture}[node distance=2.5cm, stylename]
      \node[initial, state] (q) {$q$};
      \node[state, above right of=q, draw=gray, fill=gray!50] (m1) {$~~M_1~~$};
      \node[state, below right of=q, draw=gray, fill=gray!50] (m2) {$~~M_2~~$};

      \path
      (q) edge node {$\varepsilon$} (m1)
      (q) edge node {$\varepsilon$} (m2);

    \end{tikzpicture}
  \end{center}
\end{frame}

\begin{frame}{Diseño por concatenación}{Diseño de AFNs}
  \begin{block}{Definición}
    Sean $M_1$ y $M_2$ autómatas finitos no deterministas. Si $M_1$ y $M_2$ aceptan $L_1$ y $L_2$ respectivamente, la \alert{concatenación} acepta $L_1 L_2$.
  \end{block} \pause

  \bigskip

  \begin{center}
    \begin{tikzpicture}[node distance=2.5cm, stylename]
      \node[state, initial, draw=gray, fill=gray!50] (m1) {$~~M_1~~$};
      \node[state, below right of=m1, draw=gray, fill=gray!50] (m2) {$~~M_2~~$};

      \path
      (m1) edge [bend left] node {$\varepsilon$} (m2)
      (m1) edge [bend right] node {$\varepsilon$} (m2);
    \end{tikzpicture}
  \end{center}
\end{frame}

\begin{frame}{Diseño por intersección}{Diseño de AFNs}
  \begin{block}{Definición}
    Sean $M_1$ y $M_2$ autómatas finitos no deterministas. Si $M_1$ y $M_2$ aceptan $L_1$ y $L_2$ respectivamente, la \alert{intersección}:
    \[L_1 \cap L_2 \equiv (L_1^\complement \cup L_2^\complement)^\complement\]
  \end{block} \pause

  \bigskip
  \begin{center}
    {\huge ¿La intersección?}  
  \end{center}
\end{frame}


% \section*{Referencias}

% \begin{frame}[t]{Referencias}
  % \nocite{bibID01}
  % \nocite{bibID02}
  
  % \bibliographystyle{IEEE}
  % \bibliography{biblio}
% \end{frame}

\end{document}