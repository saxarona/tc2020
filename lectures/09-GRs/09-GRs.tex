\documentclass[spanish]{beamer}

\usepackage[utf8]{inputenc}
\usepackage[spanish, mexico]{babel}
\usepackage{hyperref}
\usepackage{xcolor}
\usepackage{color}
\usepackage{ragged2e}
\usepackage{tikz}
\usepackage{mathrsfs}
\usepackage{textcomp}

\usetikzlibrary{arrows, automata, positioning, fit, shapes.geometric, backgrounds}
\tikzset{%
    stylename/.style={->, >=stealth', shorten >=1pt, auto, semithick, initial text=$ $}
}

% \usepackage{courier}
% \usepackage{subfigure}
% \usepackage{enumerate}
% \usepackage{algorithmic}
% \usepackage{algorithm}


% \usepackage{listings}
% \usepackage{lstlinebgrd}

\newcommand\blfootnote[1]{%
\begingroup
\renewcommand\thefootnote{}\footnote{#1}%
\addtocounter{footnote}{-1}%
\endgroup
}

\usetheme{Boadilla}
%\usecolortheme{default}
\usefonttheme[onlymath]{serif}
\useoutertheme{infolines}

% Sets the templates
\definecolor{navyblue}{RGB}{0, 0, 128}
\definecolor{crimson}{RGB}{128, 16, 0}

\setbeamertemplate{navigation symbols}{}
\setbeamertemplate{headline}{}
\setbeamertemplate{title page}[default][colsep=-4bp,rounded=true]
\setbeamertemplate{footline}[frame number]
\setbeamertemplate{bibliography item}[text]
\setbeamertemplate{theorems}[numbered]

\setbeamercolor{title}{fg=navyblue, bg=white}
\setbeamercolor{frametitle}{fg=navyblue, bg=white}
\setbeamercolor{structure}{fg=navyblue}
\setbeamercolor{button}{fg=white,bg=navyblue}

\setbeamercovered{transparent}

\title{Gramáticas Regulares}
\subtitle{Matemáticas Computacionales \\ (TC2020)}
\author{
\texorpdfstring{
\begin{center}
    M.C. Xavier Sánchez Díaz \\
    \href{mailto:mail@tec.mx}{\texttt{mail@tec.mx}}
\end{center}
}
{M.C. Xavier Sánchez Díaz}
}

\institute[Tecnológico de Monterrey]{\includegraphics[scale=0.5]{../img/logo}}
\date{}

\begin{document}


\setlength{\rightskip}{0pt}

\begin{frame}[plain]
\titlepage
\end{frame}

\begin{frame}{Tabla de contenidos}
\tableofcontents
\end{frame}

\section{Gramáticas como representaciones de lenguajes}

\begin{frame}{Definición}{Gramáticas como representaciones de lenguajes}

    \begin{block}{Definición}
        Según la Real Academia Española, la \alert{gramática} es una parte de la lingüística que estudia los elementos de una lengua, así como \textbf{la forma} en que estos se organizan y se combinan.
    \end{block} \pause

    \bigskip

    Es decir, la \alert{gramática} estudia la \textbf{forma} en que las palabras se organizan y combinan. \pause
    
    \bigskip

    Tenemos gramáticas \textit{naturales} como las que usamos en español, en inglés, en francés, ... \pause
    Pero también existen gramáticas \textit{artificiales} como las que usamos en Python, C\#, Java, ... \pause

    \bigskip

    ¿Qué las hace naturales o artificiales?
\end{frame}

\begin{frame}{Ejemplo: gramática \texttt{es-MX}}{Gramáticas como representaciones de lenguajes}

    Una \alert{oración} suele tener \textbf{sujeto} y \textbf{predicado}. \pause

    {\large \texttt{<frase> $\to$ <sujeto><predicado>}} \pause

    \bigskip

    El \alert{sujeto} suele ser un \textbf{sustantivo}. \pause

    {\large \texttt{<sujeto> $\to$ <sustantivo>}} \pause

    \bigskip

    \textbf{Juan}, \textbf{María} y \textbf{Gustavo} (el perro) son todos \alert{sustantivos}. \pause

    {\large \texttt{<sustantivo> $\to$ Gustavo}}\\
    {\large \texttt{<sustantivo> $\to$ Juan}}\\
    {\large \texttt{<sustantivo> $\to$ María}} \pause

    \bigskip

    El \alert{predicado} suele llevar un \textbf{verbo} y a veces un \textbf{objeto}. Y un \alert{objeto} es algún \textbf{conectivo} y un \textbf{sustantivo}, y ... \pause

    {\large \texttt{<predicado> $\to$ <verbo intransitivo>}}\\
    {\large \texttt{<predicado> $\to$ <verbo transitivo><objeto>}}\\
    {\large \texttt{<objeto> $\to$ a <sustantivo>}}

\end{frame}

\begin{frame}{Ejemplo: gramática \texttt{es-MX}}{Gramáticas como representaciones de lenguajes}

    \begin{align*}
        \langle \text{frase} \rangle & \to \langle \text{sujeto} \rangle \langle \text{predicado} \rangle\\
        \langle \text{sujeto} \rangle & \to \langle \text{sustantivo} \rangle\\
        \langle \text{sustantivo} \rangle & \to \alert<2->{\text{Gustavo}}\\
        \langle \text{sustantivo} \rangle & \to \alert<2->{\text{Juan}}\\
        \langle \text{sustantivo} \rangle & \to \alert<2->{\text{María}}\\
        \langle \text{predicado} \rangle & \to \langle \text{verbo intransitivo} \rangle\\
        \langle \text{predicado} \rangle & \to \langle \text{verbo transitivo} \rangle \langle \text{objeto}\rangle\\
        \langle \text{verbo intransitivo} \rangle & \to \alert<2->{\text{patina}}\\
        \langle \text{verbo transitivo} \rangle & \to \alert<2->{\text{abraza}}\\
        \langle \text{objeto} \rangle & \to \text{a } \langle \text{sustantivo} \rangle
    \end{align*}

    \onslide<3>{
        Conjunto de reglas con \textbf{\textlangle variables\textrangle} y \alert{símbolos terminales}.
    }

\end{frame}

\section{Reglas y derivación}

\begin{frame}{La gramática como un conjunto de reglas}{Reglas y derivación}
    Cada línea del ejemplo anterior es una \alert{regla}:
    
    \bigskip

    \begin{center}
        {\large \texttt{\textlangle objeto\textrangle \quad $\to$ \quad \textlangle sustantivo\textrangle}}
    \end{center}  \pause
    
    \bigskip

    ¿Qué \textbf{forma} tienen las reglas? \pause
    
    \[\alpha \to \beta\] \pause

    Tanto $\alpha$ como $\beta$ son símbolos---algunos son terminales, otros no.

\end{frame}

\begin{frame}{Aplicación de reglas}{Reglas y derivación}
    \only<1>{Las reglas se aplican recursivamente:} \pause
    
    \textlangle frase\textrangle $\to$ \textlangle \alert<3>{sujeto}\textrangle \textlangle \alert<5>{predicado}\textrangle \pause

    \bigskip

    \textlangle \alert<3>{sujeto}\textrangle $\to$ \textlangle \alert<4>{sustantivo}\textrangle \pause

    \bigskip
            
    \textlangle \alert<4>{sustantivo}\textrangle $\to$ \alert<9>{María} \pause

    \bigskip

    \textlangle \alert<5>{predicado}\textrangle $\to$ \textlangle \alert<6>{verbo transitivo}\textrangle \textlangle \alert<7>{objeto}\textrangle \pause

    \bigskip

    \textlangle \alert<6>{verbo transitivo}\textrangle $\to$ \alert<9>{abraza} \pause

    \bigskip

    \textlangle \alert<7>{objeto}\textrangle $\to$ \alert<9>{a} \textlangle \alert<8>{sustantivo}\textrangle \pause

    \bigskip

    \textlangle \alert<8>{sustantivo}\textrangle $\to$ \alert<9>{Gustavo} \pause

    \bigskip

    \begin{center}
        \only<9>{\huge \alert{María abraza a Gustavo}}
    \end{center}

\end{frame}

\begin{frame}{Derivación de reglas}{Reglas y derivación}
    
    Al aplicar una regla de forma $\alpha \to \beta$ a la expresión $u \alpha v$ da como resultado la expresión $u \beta v$. \pause

    \bigskip

    A este proceso le llamamos \alert{derivación}, donde pasamos de una expresión a otra:
    
    \[u \alpha v \implies u \beta v\] \pause

    Por ejemplo: María abraza \textlangle objeto\textrangle $\implies$ María abraza a \textlangle sustantivo\textrangle. \pause

    \bigskip

    Usualmente representamos una derivación \textit{completa} en una gramática como si fuera una \textbf{Kleene Star}:

    \[S \implies\mkern-10mu ^{*} \enspace w\]
    
\end{frame}

\section{Definición y diseño de gramáticas formales}

\begin{frame}{Definición de una gramática formal}{Definición y diseño de gramáticas formales}
    
    \begin{block}{Definición}
        Formalmente, una \alert{gramática} es un cuádruplo $G = (V, \Sigma, R, S)$, donde \pause

        \begin{enumerate}
            \itemsep1.5ex
            \item $V$ es un conjunto finito de \textbf{variables}, \pause
            \item $\Sigma$ es un conjunto finito de \textbf{terminales}, \pause
            \item $R$ es un conjunto finito de \textbf{reglas} de la forma $A \to w$, tal que $A \in V$ y $w \in (V \cup \Sigma)^*$ \pause
            \item $S \in V$ es la \textbf{variable inicial}.
        \end{enumerate}
        
    \end{block}

    \begin{exampleblock}{Lenguaje $L$ generado por $G$}
        \[L = \{w \in \Sigma^* : S \implies\mkern-10mu ^{*} \enspace w\}\]
    \end{exampleblock}
\end{frame}

\begin{frame}{Ejemplo: gramática formal}{Definición y diseño de gramáticas formales}
    \begin{align*}
        \langle \text{frase} \rangle & \to \langle \text{sujeto} \rangle \langle \text{predicado} \rangle\\
        \langle \text{sujeto} \rangle & \to \langle \text{sustantivo} \rangle\\
        \langle \text{sustantivo} \rangle & \to \text{Gustavo}\\
        \langle \text{sustantivo} \rangle & \to \text{Juan}\\
        \langle \text{sustantivo} \rangle & \to \text{María}\\
        \langle \text{predicado} \rangle & \to \langle \text{verbo intransitivo} \rangle\\
        \langle \text{predicado} \rangle & \to \langle \text{verbo transitivo} \rangle \langle \text{objeto}\rangle\\
        \langle \text{verbo intransitivo} \rangle & \to \text{patina}\\
        \langle \text{verbo transitivo} \rangle & \to \text{abraza}\\
        \langle \text{objeto} \rangle & \to \text{a } \langle \text{sustantivo} \rangle
    \end{align*}

    {\large ¿Cuál es la definición formal de esta gramática?}
\end{frame}

\begin{frame}{Ejemplo: diseño de gramática formal}{Definición y diseño de gramáticas formales}
    {\large \textit{Crear una Gramática Regular que genere el lenguaje de palabras sobre $\{a,b\}$ de longitud par y terminadas en $a$.}} \pause

    \bigskip

    Necesitamos \textbf{tres} variables: \pause

    \begin{enumerate}
        \item $S$ para recordar que no hemos recibido nada. Si agregamos un símbolo, entonces se hace impar. \pause
        \item $I$ para recordar que llevamos una palabra de longitud impar. Si agregamos un símbolo, entonces se hace par. \pause
        \item $P$ para recordar que llevamos una palabra de longitud par. Si agregamos un símbolo, entonces se hace impar.
    \end{enumerate} \pause

    $S$ nos lleva a $I$ que nos lleva a $P$ que nos lleva a $I$...
    
\end{frame}

\begin{frame}{Ejemplo: diseño de gramática formal}{Definición y diseño de gramáticas formales}
    Cada variable tiene distintas reglas: \pause

    \bigskip

    \only<2>{\alert{$I$} para recordar que llevamos una palabra de longitud impar:}
    \only<3>{\alert{$P$} para recordar que llevamos una palabra de longitud par:}
    \only<4>{Completamos las reglas que hagan falta:}

    \begin{enumerate}
        \itemsep1.5ex
        \item<2-> $S \to a\alert<2>{I}$ si lo primero que recibimos es $a$
        \item<2-> $S \to b\alert<2>{I}$ si lo primero que recibimos es $b$
        \item<3-> $I \to a\alert<3>{P}$ si lo siguiente que recibimos es $a$
        \item<3-> $I \to b\alert<3>{P}$ si lo siguiente que recibimos es $b$
        \item<4-> $P \to aI$ si lo siguiente que recibimos es $a$ \pause
        \item<4-> $P \to bI$ si lo siguiente que recibimos es $b$ \pause
        \item<4-> $I \to a$ para terminar con una $a$ y con una longitud par.
    \end{enumerate}
    
\end{frame}

% \section*{Referencias}
% \begin{frame}[t]{Referencias}
% \nocite{bibID01}
% \nocite{bibID02}

% \bibliographystyle{IEEE}
% \bibliography{biblio}
% \end{frame}

\end{document}