\documentclass[spanish]{beamer}

\usepackage[utf8]{inputenc}
\usepackage[spanish, mexico]{babel}
\usepackage{hyperref}
\usepackage{xcolor}
\usepackage{color}
\usepackage{ragged2e}
\usepackage{tikz}
\usepackage{mathrsfs}
\usepackage{textcomp}
\usepackage{enumitem}

\newlist{longenum}{enumerate}{4}

\usetikzlibrary{arrows, automata, positioning, fit, shapes.geometric, backgrounds}
\tikzset{%
    stylename/.style={->, >=stealth', shorten >=1pt, auto, semithick, initial text=$ $}
}

% \usepackage{courier}
% \usepackage{subfigure}
% \usepackage{enumerate}
% \usepackage{algorithmic}
% \usepackage{algorithm}


% \usepackage{listings}
% \usepackage{lstlinebgrd}

\newcommand\blfootnote[1]{%
\begingroup
\renewcommand\thefootnote{}\footnote{#1}%
\addtocounter{footnote}{-1}%
\endgroup
}

\usetheme{Boadilla}
%\usecolortheme{default}
\usefonttheme[onlymath]{serif}
\useoutertheme{infolines}

% Sets the templates

\definecolor{navyblue}{RGB}{0, 0, 128}

\setbeamertemplate{navigation symbols}{}
%\setbeamertemplate{headline}{}
\setbeamertemplate{footline}[frame number]
\setbeamertemplate{bibliography item}[text]
\setbeamertemplate{theorems}[numbered]

\setbeamercolor{title}{fg = navyblue, bg = white}
\setbeamercolor{frametitle}{fg = navyblue, bg = white}
\setbeamercolor{structure}{fg = navyblue}
\setbeamercolor{button}{fg = white,bg = navyblue}

\setbeamercovered{transparent}

\title{Gramáticas Regulares}
\subtitle{Matemáticas Computacionales \\ (TC2020)}
\author{
\texorpdfstring{
\begin{center}
    M.C. Xavier Sánchez Díaz \\
    \href{mailto:sax@itesm.mx}{\texttt{sax@itesm.mx}}
\end{center}
}
{M.C. Xavier Sánchez Díaz}
}

\institute[Tecnológico de Monterrey]{\includegraphics[scale=0.5]{../img/logo}}
\date{}

\begin{document}


\setlength{\rightskip}{0pt}

\begin{frame}[plain]
\titlepage
\end{frame}

\begin{frame}{Tabla de contenidos}
\tableofcontents
\end{frame}

\section{Gramáticas como representaciones de lenguajes}

\begin{frame}{Definición}{Gramáticas como representaciones de lenguajes}

    \begin{block}{Definición}
        Según la Real Academia Española, la \alert{gramática} es una parte de la lingüística que estudia los elementos de una lengua, así como \textbf{la forma} en que estos se organizan y se combinan.
    \end{block} \pause

    \bigskip

    Es decir, la \alert{gramática} estudia la \textbf{forma} en que las palabras se organizan y combinan. \pause
    
    \bigskip

    Tenemos gramáticas \textit{naturales} como las que usamos en español, en inglés, en francés, ... \pause
    Pero también existen gramáticas \textit{artificiales} como las que usamos en Python, C\#, Java, ... \pause

    \bigskip

    ¿Qué las hace naturales o artificiales?
\end{frame}

\begin{frame}{Ejemplo: gramática \texttt{es-MX}}{Gramáticas como representaciones de lenguajes}

    Una \alert{oración} suele tener \textbf{sujeto} y \textbf{predicado}. \pause

    {\large \texttt{<frase> $\to$ <sujeto><predicado>}} \pause

    \bigskip

    El \alert{sujeto} suele ser un \textbf{sustantivo}. \pause

    {\large \texttt{<sujeto> $\to$ <sustantivo>}} \pause

    \bigskip

    \textbf{Juan}, \textbf{María} y \textbf{Gustavo} (el perro) son todos \alert{sustantivos}. \pause

    {\large \texttt{<sustantivo> $\to$ Gustavo}}\\
    {\large \texttt{<sustantivo> $\to$ Juan}}\\
    {\large \texttt{<sustantivo> $\to$ María}} \pause

    \bigskip

    El \alert{predicado} suele llevar un \textbf{verbo} y a veces un \textbf{objeto}. Y un \alert{objeto} es algún \textbf{conectivo} y un \textbf{sustantivo}, y ... \pause

    {\large \texttt{<predicado> $\to$ <verbo intransitivo>}}\\
    {\large \texttt{<predicado> $\to$ <verbo transitivo><objeto>}}\\
    {\large \texttt{<objeto> $\to$ a <sustantivo>}}

\end{frame}

\begin{frame}{Ejemplo: gramática \texttt{es-MX}}{Gramáticas como representaciones de lenguajes}

    \begin{align*}
        \langle \text{frase} \rangle & \to \langle \text{sujeto} \rangle \langle \text{predicado} \rangle\\
        \langle \text{sujeto} \rangle & \to \langle \text{sustantivo} \rangle\\
        \langle \text{sustantivo} \rangle & \to \alert<2->{\text{Gustavo}}\\
        \langle \text{sustantivo} \rangle & \to \alert<2->{\text{Juan}}\\
        \langle \text{sustantivo} \rangle & \to \alert<2->{\text{María}}\\
        \langle \text{predicado} \rangle & \to \langle \text{verbo intransitivo} \rangle\\
        \langle \text{predicado} \rangle & \to \langle \text{verbo transitivo} \rangle \langle \text{objeto}\rangle\\
        \langle \text{verbo intransitivo} \rangle & \to \alert<2->{\text{patina}}\\
        \langle \text{verbo transitivo} \rangle & \to \alert<2->{\text{abraza}}\\
        \langle \text{objeto} \rangle & \to \text{a } \langle \text{sustantivo} \rangle
    \end{align*}

    \onslide<3>{
        Conjunto de reglas con \textbf{\textlangle variables\textrangle} y \alert{símbolos terminales}.
    }

\end{frame}

\section{Reglas y derivación}

\begin{frame}{La gramática como un conjunto de reglas}{Reglas y derivación}
    Cada línea del ejemplo anterior es una \alert{regla}:
    
    \bigskip

    \begin{center}
        {\large \texttt{\textlangle objeto\textrangle \quad $\to$ \quad \textlangle sustantivo\textrangle}}
    \end{center}  \pause
    
    \bigskip

    ¿Qué \textbf{forma} tienen las reglas? \pause
    
    \[\alpha \to \beta\] \pause

    Tanto $\alpha$ como $\beta$ son símbolos---algunos son terminales, otros no.

\end{frame}

\begin{frame}{Aplicación de Reglas}{Reglas y derivación}
    Las reglas se aplican recursivamente:

    \begin{longenum}
        \item \textlangle frase\textrangle $\to$ \textlangle sujeto\textrangle \textlangle predicado\textrangle

        \begin{longenum}

            \item \textlangle sujeto\textrangle $\to$ \textlangle sustantivo\textrangle

            \begin{longenum}
               \item \textlangle sustantivo\textrangle $\to$ María
            \end{longenum}

            \item \textlangle predicado\textrangle $\to$ \textlangle verbo transitivo\textrangle \textlangle objeto\textrangle
            
            \begin{longenum}
                \textlangle verbo transitivo\textrangle $\to$ \textlangle abraza\textrangle
                \textlangle objeto\textrangle $\to$ a \textlangle sustantivo\textrangle
                \begin{longenum}
                    \item \textlangle sustantivo\textrangle $\to$ Gustavo
                \end{longenum}
            \end{longenum}
        \end{longenum}
    \end{longenum}

        {\huge María abraza a Gustavo}

\end{frame}


% \section*{Referencias}
% \begin{frame}[t]{Referencias}
% \nocite{bibID01}
% \nocite{bibID02}

% \bibliographystyle{IEEE}
% \bibliography{biblio}
% \end{frame}

\end{document}