\documentclass[spanish]{beamer}

\usepackage[utf8]{inputenc}
\usepackage[spanish, mexico]{babel}
\usepackage{hyperref}
\usepackage{xcolor}
\usepackage{color}
\usepackage{ragged2e}
\usepackage{tikz}
\usepackage{mathrsfs}

\usetikzlibrary{fit, shapes}

% \usepackage{courier}
% \usepackage{subfigure}
% \usepackage{enumerate}
% \usepackage{algorithmic}
% \usepackage{algorithm}


% \usepackage{listings}
% \usepackage{lstlinebgrd}

\usetheme{Boadilla}
\usefonttheme[onlymath]{serif}

% Sets the templates
\definecolor{navyblue}{RGB}{0, 0, 128}

\setbeamertemplate{navigation symbols}{} 
\setbeamertemplate{headline}{}
\setbeamertemplate{footline}[frame number]
\setbeamertemplate{bibliography item}[text]
\setbeamertemplate{theorems}[numbered]

\setbeamercolor{title}{fg=navyblue, bg=white}
\setbeamercolor{frametitle}{fg=navyblue, bg=white}
\setbeamercolor{structure}{fg=navyblue}

\setbeamercovered{transparent}

\title{Conceptos matemáticos preliminares}
\subtitle{Matemáticas Computacionales \\ (TC2020)}
\author{
    \texorpdfstring{
        \begin{center}
            M.C. Xavier Sánchez Díaz \\
            \href{mailto:sax@itesm.mx}{\texttt{sax@itesm.mx}}
        \end{center}
    }
    {M.C. Xavier Sánchez Díaz}
}

\institute[Tecnológico de Monterrey]{\includegraphics[scale=0.5]{img/logo}}
\date{}

\begin{document}

\setlength{\rightskip}{0pt}

\begin{frame}[plain]
    \titlepage        
\end{frame}

\begin{frame}{Tabla de contenidos}
    \tableofcontents
\end{frame}

\section{Conjuntos}
\label{sec:conjuntos}

\begin{frame}{Definición de conjunto}{Conjuntos}
    \begin{definition}
        Un \alert{conjunto} es una colección de elementos.
        Usamos letras mayúsculas $A, B, C, \dots$ para representarlos, y letras minúsculas $a, b, c, \dots$ para representar sus elementos.
    \end{definition}
    \bigskip
    \[A = \{a, b, c, d\}\]
\end{frame}

\begin{frame}{Describiendo un conjunto}{Conjuntos}
    Podemos definirlos por \textit{enumeración} o por \textit{descripción}. \pause
    \bigskip
    \begin{center}
        $A$ es el conjunto de todos los números naturales menores que 6. \pause
    \end{center}

    \begin{exampleblock}{Enumerando sus elementos}
        $A = \{1, 2, 3, 4, 5\} = \{2, 3, 1, 5, 4\}$
    \end{exampleblock} \pause

    \begin{exampleblock}{Describiendo sus elementos}
        $A = \{a \in \mathbb{N} : a < 6\}$
    \end{exampleblock}
\end{frame}

\begin{frame}{Notación de conjuntos}{Conjuntos}
    \begin{itemize}
    \justifying
    \itemsep1.5em
        \item <1-> \textbf{Pertenencia}: $a \in A$, cuando $a$ es un elemento de $A$.
        \item <2-> \textbf{Cardinalidad}: $|A|$ representa el número de elementos en $A$.
        \item <3-> \textbf{Inclusión}: $A \subseteq B$ si todos los elementos de $A$ son elementos de $B$.
        \item <4-> \textbf{Igualdad}: si $A \subseteq B$ y $B \subseteq A$, entonces $A = B$.
        \item <5-> \textbf{Inclusión propia}: $A \subset B$ si todos los elementos de $A$ son elementos de $B$ y $A \neq B$.
        \item <6-> \textbf{Conjunto vacío}: $\varnothing$ o \{\} para representar un conjunto sin elementos.
    \end{itemize}
\end{frame}

\begin{frame}{Pertenencia e inclusión}{Conjuntos}
    \begin{columns}
        \begin{column}{0.5\textwidth}
        \begin{enumerate}
            \itemsep1.5em
            \item <1,11> $\{a\} \subseteq \{\{a\}\}$
            \item <2,11> \alert<11>{$\{a\} \subseteq \{b, c, \{a\}\}$}
            \item <3,11> \alert<11>{$a \subseteq \{a, b, c\}$}
            \item <4,11> $\{a\} \in \{b,c, \{a\}\}$
            \item <5,11> $a \in \{a,b,c\}$
        \end{enumerate}
    \end{column}

        \begin{column}{0.5\textwidth}
        \begin{enumerate}
            \itemsep1.5em
            \setcounter{enumi}{5}
            \item <6,11> $\{b\} \in \{a, c, \{b\}\}$
            \item <7,11> $b \in \{b\}$
            \item <8,11> \alert<11>{$\{b\} \subseteq \{\{b,c\}\}$}
            \item <9,11> \alert<11>{$\{b\} \subseteq \{\{a\}, c, \{b\}\}$}
            \item <10,11> $\{c\} \subset \{\{a\}, c, \{b\}\}$
        \end{enumerate}
        \end{column}
    \end{columns}
\end{frame}

\begin{frame}{Conjunto vacío}{Conjuntos}
    \begin{enumerate}
        \itemsep1.5em
        \item <1,6> $\varnothing \subset \{\varnothing\}$ 
        \item <2,6> $\varnothing \in \{\varnothing\}$
        \item <3,6> \alert<6>{$0 = \varnothing$}
        \item <4,6> $\varnothing \subseteq \varnothing$
        \item <5,6> \alert<6>{$\varnothing \subset \varnothing$}
    \end{enumerate}
\end{frame}

\begin{frame}{Operaciones de conjuntos}{Conjuntos} 
    \def \setA{(-0.7, 0) circle (1)}
    \def \setB{(0.7, 0) circle (1)}
    \def \setC{(0, -1) circle (1)}

    \begin{columns}
        \begin{column}{0.5\textwidth}
            \begin{tikzpicture}
                \draw (-2, 1.5) rectangle (2, -2.5);
                \draw[color=red, line width=1pt] \setA node[above left] {$A$};
                \draw[color=blue, line width=1pt] \setB node[above] {$B$};
                \draw[color=green, line width=1pt] \setC node[below] {$C$};
            \end{tikzpicture}
        \end{column}
        \begin{column}{0.5\textwidth}
            \begin{enumerate}
                \item <1> $B \cup C$
                \item <2> $C \cup (A \cap B)$
                \item <3> $C - (A \cap B)$
                \item <4> $A - (B \cup C)$
                \item <5> $A^\complement$
                \item <6> $B - (A \cup C)$
                \item <7> $(B - A)^\complement$
            \end{enumerate}
        \end{column}
    \end{columns}
\end{frame}

\begin{frame}{Producto Cartesiano}{Conjuntos}
    \begin{definition}
        El \alert{producto Cartesiano} entre $A$ y $B$ se define como:
        \[A \times B = \{(x,y) : x \in A, y \in B\}\] 
    \end{definition}
        \pause
    \begin{exampleblock}{Ejemplo}
        \vspace{-2.5ex}
        \begin{align*}
        \onslide<2->{& A = \{1,2,3\} \quad B = \{1,2\}\\}
        \onslide<3->{A \times B & = \{(1,1), (1,2), (2,1), (2,2), (3,1), (3,2)\}}
        \end{align*}
    \end{exampleblock}
        \onslide<4->{\[|A \times B| = |A| \times |B|\]}
\end{frame}

\begin{frame}{Conjunto potencia}{Conjuntos}
    \begin{definition}
        Sea $A$ cualquier conjunto. El \alert{conjunto potencia} de $A$---denotado por $\mathscr{P}(A)$ o $\wp(A)$---consiste en el conjunto de todos (y únicamente) los subconjuntos de $A$.
    \end{definition}
    \pause
    \bigskip
    \[\mathscr{P}(\{a,b,c\}) = \{\varnothing, \{a\}, \{b\}, \{c\}, \{a,b\}, \{a,c\}, \{b,c\}, \{a,b,c\}\}\] \pause
    \bigskip
    \[|\mathscr{P}(A)| = 2^{|A|}\]
\end{frame}

\begin{frame}{Equivalencias}{Conjuntos}
    La \textbf{unión} y la \textbf{intersección} son \alert{conmutativas}. \pause
    \bigskip
    \[A \cup B = B \cup A\] \pause
    \bigskip
    \[A \cap B = B \cap A\]
\end{frame}

\begin{frame}{Equivalencias}{Conjuntos}
    La \textbf{unión} y la \textbf{intersección} son \alert{asociativas}. \pause
    \bigskip
    \[(A \cup B) \cup C = A \cup (B \cup C)\] \pause
    \bigskip
    \[(A \cap B) \cap C = A \cap (B \cap C)\]
\end{frame}

\begin{frame}{Equivalencias}{Conjuntos}
    La \textbf{unión} y la \textbf{intersección} son \alert{distributivas} entre ellas. \pause
    \bigskip
    \[A \cup (B \cap C) = (A \cup B) \cap (A \cup C)\] \pause
    \bigskip
    \[A \cap (B \cup C) = (A \cap B) \cup (A \cap C)\]
\end{frame}

\begin{frame}{Equivalencias}{Conjuntos}
    Leyes de \alert{De Morgan}: \pause
    \bigskip
    \[(A \cup B)^\complement = A^\complement \cap B^\complement\] \pause
    \[(A \cap B)^\complement = A^\complement \cup B^\complement\]
\end{frame}

\section{Relaciones y Funciones}
\label{sec:relfunc}

\begin{frame}{Relaciones}{Relaciones y Funciones}
    \begin{definition}
        Sean $A$ y $B$ dos conjuntos cualesquiera. Una \alert{relación binaria} $R$ \textit{de} $A$ \textit{a} $B$ se define como cualquier subconjunto del producto Cartesiano $A \times B$.
        Es decir, cualquier conjunto de pares ordenados de la forma $(a,b)$ tal que $a \in A$ y $b \in B$.
        También se dice que una relación $R$ puede ser \textit{sobre} $A \times B$.
    \end{definition} \pause
    \bigskip
    \begin{exampleblock}{Ejemplo}
        \textit{Menor o igual} ($\leq$) es una \textbf{relación} sobre $\mathbb{N} \times \mathbb{N}$:
        \[\leq \, = \{(1,1), (1,2), (1,3), \dots , (2,2), (2,3), \dots , (3,4), \dots \}\]
    \end{exampleblock}
\end{frame}

\begin{frame}{Reflexividad}{Relaciones y funciones}

    \begin{definition}
        Sea $R$ una relación binaria. Se dice que es \alert{reflexiva} sobre un conjunto $A$ sí y solo sí $(a,a) \in R$ para todo $a \in A$.
    \end{definition} \pause
    \bigskip
    \begin{exampleblock}{Ejemplo}
        \textit{Menor o igual} ($\leq$) es una relación \textbf{reflexiva} sobre $\mathbb{N} \times \mathbb{N}$:
        \[\leq \, = \{\mathbf{(1,1)}, (1,2), (1,3), \dots , \mathbf{(2,2)}, (2,3), \dots , (3,4), \dots \}\]
    \end{exampleblock} \pause
    \bigskip
    ¿Es la relación \textit{menor que} $(<)$ reflexiva sobre los naturales?
\end{frame}

\begin{frame}{Reflexividad}{Relaciones y funciones}
    \begin{center}
        \begin{tikzpicture}[
            >=stealth,
            bullet/.style={
              fill=black,
              circle,
              minimum width=1pt,
              inner sep=1pt
            },
            projection/.style={
              ->,
              thick,
              shorten <=2pt,
              shorten >=2pt
            },
            every fit/.style={
              ellipse,
              draw,
              inner sep=0pt
            }
          ]
            \foreach \y/\l in {1/d,2/c/,3/b,4/a}
              \node[bullet,label=left:$\y$] (a\y) at (0,\y) {};
        
            \foreach \y/\l in {1/1,2/2,3/3,4/4}
              \node[bullet,label=right:$\l$] (b\y) at (4,\y) {};
        
            \node[draw,fit=(a1) (a2) (a3) (a4),minimum width=1.8cm] {} ;
            \node[draw,fit=(b1) (b2) (b3) (b4),minimum width=1.8cm] {} ;
        
            \draw[projection] (a1) -- (b1);
            \draw[projection] (a2) -- (b2);
            \draw[projection] (a3) -- (b3);
            \draw[projection] (a4) -- (b4);
          \end{tikzpicture}
    \end{center}
\end{frame}

\begin{frame}{Transitividad}{Relaciones y funciones}
    \begin{definition}
        Decimos que $R$ es \alert{transitiva} si y sólo si cuando $(a,b) \in R$ y $(b,c) \in R$, entonces $(a,c) \in R$.
    \end{definition} \pause
    \bigskip
    \begin{exampleblock}{Ejemplo}
        \textit{Menor o igual} ($\leq$) es una relación \textbf{transitiva} sobre $\mathbb{N} \times \mathbb{N}$:
        \[\leq \, = \{(1,1), \mathbf{(1,2)}, \mathbf{(1,3)}, \dots , (2,2), \mathbf{(2,3)}, \dots , (3,4), \dots \}\]
    \end{exampleblock} \pause
    \bigskip
    ¿Es la relación \textit{menor que} $(<)$ transitiva sobre los naturales?
\end{frame}

\begin{frame}{Simetría}{Relaciones y funciones}
    \begin{definition}
        Una relación $R$ es \alert{simétrica} si y sólo si cuando $(a,b) \in R$, entonces $(b,a) \in R$.
    \end{definition} \pause
    \bigskip
    \begin{exampleblock}{Ejemplo 1}
        La relación de \textit{igualdad} $(=)$ es una relación \textbf{simétrica}: si $a = b$, entonces $b = a$.
    \end{exampleblock} \pause
    \bigskip
    \begin{exampleblock}{Ejemplo 2}
        La relación de \textit{hermandad} es \textbf{simétrica}: si $Juan$ es hermano de $Pedro$, entonces $Pedro$ es hermano de $Juan$. 
    \end{exampleblock}
\end{frame}

\begin{frame}{Mapeo o Función}{Relaciones y Funciones}
    
    ¿Cualquier relación es una función? \pause
    
    \bigskip

    ¿Cualquier función es una relación?
\end{frame}

\begin{frame}{Mapeo o Función}{Relaciones y Funciones}
    \begin{definition}
        Una \alert{función} \textit{unitaria} de un conjunto $A$ en un conjunto $B$ es cualquier relación binaria $R$ de $A$ a $B$ que satisfaga la condición de que \textit{para todo} $a \in A$ existe \textit{exactamente un} $b \in B$ tal que $(a,b) \in R$.
    \end{definition} \pause
    \bigskip
    Podemos describir una función $f$ de $A$ en $B$ como $f : A \to B$. \pause
    \bigskip
    \begin{exampleblock}{Ejemplo}
        La relación \textit{sucesor} es una \textbf{función} de los naturales en los naturales $f : \mathbb{N} \to \mathbb{N}$
        \[\mathtt{suc}(n) = \{(1,2), (2,3), (3,4), (4,5), \dots\}\]
    \end{exampleblock}
\end{frame}

\begin{frame}{Dominio}{Funciones y Relaciones}
    \begin{definition}
        El \alert{dominio} de una función $f$ puede definirse como
        \[\mathtt{dom}(f) = \{a \in A : \exists b \in B, f(a) = b\}\]
    \end{definition} \pause
    \bigskip
    En una función de forma $f : A \to B$, el \textbf{dominio} es simplemente $A$.
\end{frame}

\begin{frame}{Codominio o Rango}{Funciones y Relaciones}
    \begin{definition}
        El \alert{codominio} (también conocido como \alert{rango}) de una función $f$ puede definirse como
        \[\mathtt{codom}(f) = \{b \in B : \exists a \in A, f(a) = b\}\]
    \end{definition} \pause
    \bigskip
    En una función de forma $f : A \to B$, el \textbf{codominio} es simplemente $B$.    
\end{frame}

% Falta definición de funciones totales y parciales
% Falta imagen
% Propiedades de funciones (inyectiva...)
% Conjuntos infinitos
% Lógica básica
% 

% \section*{Referencias}

% \begin{frame}[t]{Referencias}
    % \nocite{bibID01}
    % \nocite{bibID02}

    % \bibliographystyle{IEEE}
    % \bibliography{biblio}
% \end{frame}

\end{document}