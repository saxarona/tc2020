\documentclass[]{book}

%These tell TeX which packages to use.
\usepackage{array,epsfig}
\usepackage{amsmath}
\usepackage{amsfonts}
\usepackage{amssymb}
\usepackage{amsxtra}
\usepackage{amsthm}
\usepackage{mathrsfs}
\usepackage{color}
\usepackage[spanish, mexico]{babel}
\usepackage[utf8]{inputenc}

%Here I define some theorem styles and shortcut commands for symbols I use often
\theoremstyle{definition}
\newtheorem{defn}{Definition}
\newtheorem{thm}{Theorem}
\newtheorem{cor}{Corollary}
\newtheorem*{rmk}{Remark}
\newtheorem{lem}{Lemma}
\newtheorem*{joke}{Joke}
\newtheorem{ex}{Example}
\newtheorem*{sol}{Solution}
\newtheorem{prop}{Proposition}

\newcommand{\lra}{\longrightarrow}
\newcommand{\ra}{\rightarrow}
\newcommand{\surj}{\twoheadrightarrow}
\newcommand{\graph}{\mathrm{graph}}
\newcommand{\bb}[1]{\mathbb{#1}}
\newcommand{\Z}{\bb{Z}}
\newcommand{\Q}{\bb{Q}}
\newcommand{\R}{\bb{R}}
\newcommand{\C}{\bb{C}}
\newcommand{\N}{\bb{N}}
\newcommand{\M}{\mathbf{M}}
\newcommand{\m}{\mathbf{m}}
\newcommand{\MM}{\mathscr{M}}
\newcommand{\HH}{\mathscr{H}}
\newcommand{\Om}{\Omega}
\newcommand{\Ho}{\in\HH(\Om)}
\newcommand{\bd}{\partial}
\newcommand{\del}{\partial}
\newcommand{\bardel}{\overline\partial}
\newcommand{\textdf}[1]{\textbf{\textsf{#1}}\index{#1}}
\newcommand{\img}{\mathrm{img}}
\newcommand{\ip}[2]{\left\langle{#1},{#2}\right\rangle}
\newcommand{\inter}[1]{\mathrm{int}{#1}}
\newcommand{\exter}[1]{\mathrm{ext}{#1}}
\newcommand{\cl}[1]{\mathrm{cl}{#1}}
\newcommand{\ds}{\displaystyle}
\newcommand{\vol}{\mathrm{vol}}
\newcommand{\cnt}{\mathrm{ct}}
\newcommand{\osc}{\mathrm{osc}}
\newcommand{\LL}{\mathbf{L}}
\newcommand{\UU}{\mathbf{U}}
\newcommand{\support}{\mathrm{support}}
\newcommand{\AND}{\;\wedge\;}
\newcommand{\OR}{\;\vee\;}
\newcommand{\Oset}{\varnothing}
\newcommand{\st}{\ni}
\newcommand{\wh}{\widehat}
\newcommand{\myqed}{\blacksquare}

%Pagination stuff.
\setlength{\topmargin}{-.3 in}
\setlength{\oddsidemargin}{0in}
\setlength{\evensidemargin}{0in}
\setlength{\textheight}{9.in}
\setlength{\textwidth}{6.5in}
\setlength{\itemsep}{0.45in}
\pagestyle{empty}

\renewcommand{\familydefault}{\sfdefault}



\begin{document}

\begin{center}
{\huge Matemáticas Computacionales (TC2020)}\\[1.5ex]
{\large \textbf{Apéndice 01: Conjuntos}}%\\[1.5ex]} %You should put your name here
% 06.08.18} %You should write the date here.
\end{center}

\vspace{0.2 cm}

\subsection*{Ejercicios de clase}

En clase revisamos 10 expresiones, algunas de las cuales eran incorrectas.
Aquí explicamos (de manera formal) el porqué de las respuestas.
Te puede servir para darte una idea de cómo \textit{formatear} tu tarea o para estudiar.

\begin{enumerate}
	\itemsep0.35in

	\item La expresión $\{a\} \subseteq \{\{a\}\}$ es \textdf{incorrecta}.\\
	Dado a que $a \neq \{a\}$, puede deducirse que $\{a\} \nsubseteq \{\{a\}\}$.
	Otra manera de verlo es por su \textbf{conjunto potencia}: $\mathscr{P}(\{\{a\}\}) = \{\varnothing, \{\{a\}\}\}$.
	Dado a que $\{a\} \notin \mathscr{P}(\{\{a\}\})$ entonces $\{a\} \nsubseteq \{\{a\}\} \myqed$

	\item $\{a\} \nsubseteq \{b, c, \{a\}\}$, debido a la misma razón que en la Expresión 1 (arriba) $\myqed$

	\item $a \not \subseteq \{a,b,c\}$, dado a que $a$ no es un conjunto $\myqed$

	\item La expresión $\{a\} \in \{b,c, \{a\}\}$ es \textdf{verdadera} dado a que $\{a\}$ pertenece a $\{b, c, \{a\}\} \myqed$

	\item La expresión $a \in \{a, b, c\}$ es \textdf{verdadera} por la misma razón $\myqed$

	\item La expresión $\{b\} \in \{a, c, \{b\}\}$ es \textdf{verdadera} por la misma razón $\myqed$

	\item $b \in \{b\}$ también es \textdf{verdadera} por la misma razón $\myqed$

	\item La expresión $\{b\} \subseteq \{\{b, c\}\}$ es \textdf{falsa}\\
	Dado a que $b \not \in \{\{b, c\}\}$, entonces $\{b\} \nsubseteq \{\{b, c\}\} \myqed$

	\item $\{b\} \nsubseteq \{\{a\}, c, \{b\}\}$, debido a la misma razón que en la expresión anterior $\myqed$

	\item $\{c\} \subset \{\{a\}, c, \{b\}\}$ dado a que $c \in \{\{a\}, c, \{b\}\} \myqed$
\end{enumerate}

\end{document}


