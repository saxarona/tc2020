\documentclass[]{book}

%These tell TeX which packages to use.
\usepackage{array,epsfig}
\usepackage{amsmath}
\usepackage{amsfonts}
\usepackage{amssymb}
\usepackage{amsxtra}
\usepackage{amsthm}
\usepackage{mathrsfs}
\usepackage{color}
\usepackage[spanish, mexico]{babel}
\usepackage[utf8]{inputenc}
\usepackage{booktabs}
\usepackage{hyperref}

%Here I define some theorem styles and shortcut commands for symbols I use often
\theoremstyle{definition}
\newtheorem{defn}{Definición}
\newtheorem{thm}{Teorema}
\newtheorem{cor}{Corolario}
\newtheorem*{rmk}{Nota}
\newtheorem{lem}{Lema}
\newtheorem*{joke}{Joke}
\newtheorem{ex}{Ejemplo}
\newtheorem*{sol}{Solución}
\newtheorem{prop}{Proposición}

\newcommand{\lra}{\longrightarrow}
\newcommand{\ra}{\rightarrow}
\newcommand{\surj}{\twoheadrightarrow}
\newcommand{\graph}{\mathrm{graph}}
\newcommand{\bb}[1]{\mathbb{#1}}
\newcommand{\Z}{\bb{Z}}
\newcommand{\Q}{\bb{Q}}
\newcommand{\R}{\bb{R}}
\newcommand{\C}{\bb{C}}
\newcommand{\N}{\bb{N}}
\newcommand{\M}{\mathbf{M}}
\newcommand{\m}{\mathbf{m}}
\newcommand{\MM}{\mathscr{M}}
\newcommand{\HH}{\mathscr{H}}
\newcommand{\Om}{\Omega}
\newcommand{\Ho}{\in\HH(\Om)}
\newcommand{\bd}{\partial}
\newcommand{\del}{\partial}
\newcommand{\bardel}{\overline\partial}
\newcommand{\textdf}[1]{\textbf{\textsf{#1}}\index{#1}}
\newcommand{\img}{\mathrm{img}}
\newcommand{\ip}[2]{\left\langle{#1},{#2}\right\rangle}
\newcommand{\inter}[1]{\mathrm{int}{#1}}
\newcommand{\exter}[1]{\mathrm{ext}{#1}}
\newcommand{\cl}[1]{\mathrm{cl}{#1}}
\newcommand{\ds}{\displaystyle}
\newcommand{\vol}{\mathrm{vol}}
\newcommand{\cnt}{\mathrm{ct}}
\newcommand{\osc}{\mathrm{osc}}
\newcommand{\LL}{\mathbf{L}}
\newcommand{\UU}{\mathbf{U}}
\newcommand{\support}{\mathrm{support}}
\newcommand{\AND}{\;\wedge\;}
\newcommand{\OR}{\;\vee\;}
\newcommand{\Oset}{\varnothing}
\newcommand{\st}{\ni}
\newcommand{\wh}{\widehat}
\newcommand{\myqed}{\blacksquare}

%Pagination stuff.
\setlength{\topmargin}{-.3 in}
\setlength{\oddsidemargin}{0in}
\setlength{\evensidemargin}{0in}
\setlength{\textheight}{9.in}
\setlength{\textwidth}{6.5in}
\setlength{\itemsep}{0.45in}
\pagestyle{empty}



\begin{document}

\begin{center}
{\huge Matemáticas Computacionales (TC2020)}\\[1.5ex]
{\large \textbf{Apéndice 02: Lógica}\\[1.5ex] %You should put your name here
13.08.18} %You should write the date here.
\end{center}

\vspace{0.2 cm}

\subsection*{Equivalencias lógicas}

En clase vimos algunas equivalencias lógicas, sin embargo no son todas las necesarias para poder realizar la tarea.
Aquí se desarrollan las pruebas de dos equivalencias más para que veas cómo operar con ellas.

\vspace{0.5in}

\begin{thm}
    $P \implies Q \equiv \neg P \vee Q$
\end{thm}

\begin{proof}
    Usando verificación de modelos (tablas de verdad), se puede probar que la implicación es equivalente a la disyunción de la negación de la premisa y la conclusión:

    \begin{table}[htbp]
        \centering
        \caption{Tabla de verdad de $P \implies Q$ y $\neg P \vee Q$}
        \label{tab:implies}
        \begin{tabular}{@{}cccccc@{}}
        \toprule
        \multicolumn{1}{l}{$P$} & \multicolumn{1}{l}{$Q$} & \multicolumn{1}{l}{$P \implies Q$} & \multicolumn{1}{l}{$\neg P$} & \multicolumn{1}{l}{$Q$} & \multicolumn{1}{l}{$\neg P \vee Q$} \\ \midrule
        F & F & T & T & F & T \\
        F & T & T & T & T & T \\
        T & F & F & F & F & F \\
        T & T & T & F & T & T \\ \bottomrule
        \end{tabular}
    \end{table}

Como las columnas de la implicación (3) y la disyunción entre la negación de la premisa y la conclusión (6) son idénticas, entonces podemos decir que son expresiones lógicas equivalentes.
\end{proof}
\vspace{0.5in}
\begin{thm}
    La implicación $P \implies Q$ y su contrapositiva $\neg Q \implies \neg P$ son equivalentes.
\end{thm}

\begin{proof}
    Nuevamente utilizando verificación de modelos, podemos comprobarlo.

    \begin{table}[htbp]
        \centering
        \caption{Tabla de verdad de la implicación $P \implies Q$ y su contrapositiva.}
        \label{tab:contrapositive}
        \begin{tabular}{@{}cccccc@{}}
        \toprule
        \multicolumn{1}{l}{$P$} & \multicolumn{1}{l}{$Q$} & \multicolumn{1}{l}{$P \implies Q$} & \multicolumn{1}{l}{$\neg P$} & \multicolumn{1}{l}{$\neg Q$} & \multicolumn{1}{l}{$\neg Q \implies \neg P$} \\ \midrule
        F & F & T & T & T & T \\
        F & T & T & T & F & T \\
        T & F & F & F & T & F \\
        T & T & T & F & F & T \\ \bottomrule
        \end{tabular}
    \end{table}

    Fíjate que la implicación ahora es en sentido inverso, de $\neg Q \implies \neg P$, así que utilizamos las columnas 5 y 4 para la premisa y conclusión de la contrapositiva, respectivamente.
    Como las columnas correspondientes a la implicación y su contrapositiva (3 y 6) son idénticas, entonces podemos decir que son expresiones lógicas equivalentes.
\end{proof}

\begin{rmk}
    La contrapositiva (en inglés \textit{contrapositive}) es un recurso muy usado en la resolución (regla de inferencia)\footnote{Para revisar más sobre pruebas formales de características clásicas de conjuntos y lógica, sugiero revisar la \textit{Proof Wiki}: \url{https://proofwiki.org/wiki/Main_Page}}.
\end{rmk}

\end{document}


