\documentclass[spanish, handout]{beamer}
  
    \usepackage[utf8]{inputenc}
    \usepackage[spanish, mexico]{babel}
    \usepackage{hyperref}
    \usepackage{xcolor}
    \usepackage{color}
    \usepackage{ragged2e}
    \usepackage{tikz}
    \usepackage{mathrsfs}
    
    \usetikzlibrary{arrows, automata, positioning, fit, shapes.geometric, backgrounds}
    
    \tikzset{
        stylename/.style={
            ->, %arrow type
            >=stealth', %arrow head type (bold)
            shorten >=1pt, 
            auto,
            %semithick,
            initial text=$ $, %no start text
            }
        }
    
    
    % \usepackage{courier}
    % \usepackage{subfigure}
    % \usepackage{enumerate}
    % \usepackage{algorithmic}
    % \usepackage{algorithm}
    
    
    % \usepackage{listings}
    % \usepackage{lstlinebgrd}
    
    \newcommand\blfootnote[1]{%
    \begingroup
    \renewcommand\thefootnote{}\footnote{#1}%
    \addtocounter{footnote}{-1}%
    \endgroup
    }
    
    \usetheme{Boadilla}
    \usefonttheme[onlymath]{serif}
    
    % Sets the templates
\definecolor{navyblue}{RGB}{0, 0, 128}
\definecolor{crimson}{RGB}{128, 16, 0}

\setbeamertemplate{navigation symbols}{}
\setbeamertemplate{headline}{}
\setbeamertemplate{title page}[default][colsep=-4bp,rounded=true]
\setbeamertemplate{footline}[frame number]
\setbeamertemplate{bibliography item}[text]
\setbeamertemplate{theorems}[numbered]

\setbeamercolor{title}{fg=navyblue, bg=white}
\setbeamercolor{frametitle}{fg=navyblue, bg=white}
\setbeamercolor{structure}{fg=navyblue}
\setbeamercolor{button}{fg=white,bg=navyblue}
    
    \setbeamercovered{transparent}
    
    \title{Conversión de AFNs a AFDs}
    \subtitle{Matemáticas Computacionales \\ (TC2020)}
    \author{
    \texorpdfstring{
    \begin{center}
      M.C. Xavier Sánchez Díaz \\
      \href{mailto:sax@tec.mx}{\texttt{sax@tec.mx}}
    \end{center}
    }
    {M.C. Xavier Sánchez Díaz}
    }
    
    \institute[Tecnológico de Monterrey]{\includegraphics[scale=0.5]{../img/logo}}
    \date{}
    
  \begin{document}
  
  \setlength{\rightskip}{0pt}
  
  \begin{frame}[plain]
    \titlepage        
  \end{frame}
  
  \begin{frame}{Tabla de contenidos}
    \tableofcontents
  \end{frame}
  
  \section{AFNs versus AFDs}
  \label{sec:afn-v-afd}

  \begin{frame}{¿Por qué convertir?}{AFNs versus AFDs}
    Una máquina \alert{no determinista} hace que el \textbf{diseño} sea más sencillo. \pause
    
    \bigskip
    
    Una máquina \alert{determinista} hace que la \textbf{implementación} sea más sencilla. \pause

    \bigskip

    La realidad es que un lenguaje es aceptado por una \alert{máquina determinista} \textbf{si y solo si} es aceptado por una \alert{máquina no determinista}.
  \end{frame}

  \begin{frame}{Equivalencia}{AFNs versus AFDs}
      \begin{block}{Equivalencia de AFDs y AFNs}
          Sea $N = (Q, \Sigma, \delta, q, F)$ un autómata no determinista. Existe un autómata determinista $M$ tal que $L(M) = L(N)$.
      \end{block} \pause
      
      \bigskip

      El estado en el que se encuentra $M$ después de haber leído la parte inicial de una palabra corresponde exactamente al \textbf{conjunto de todos los estados} que $N$ puede alcanzar tras haber leído la misma parte de la palabra.
  \end{frame}

  \begin{frame}{Equivalencia}{AFNs versus AFDs}
    \[N = (Q, \Sigma, \delta, q, F)\] \pause
    
    \[M = (Q', \Sigma, \delta', q', F')\] \pause
    
    \bigskip

    ¿Cómo pasamos de un AFN a un AFD?
  \end{frame}

  \section{Conversión de AFN a AFD}
  \label{sec:afn-to-afd}

  \begin{frame}{Estados de estados}{Conversión de AFN a AFD}
    El AFN \textbf{acepta} una palabra si existe al menos un estado final dentro del conjunto de estados donde la secuencia termina. \pause

    \only<4>{
        \begin{center}
            \begin{tikzpicture}[node distance=2cm, stylename]
              \node[initial, state, draw=red] (0) {$0$};
              \node[state, above of=0] (1) {$1$};
              \node[state, accepting, right of=0, xshift=1.5cm, draw=red] (2) {$2$};
              \node[state, above right of=2] (3) {$3$};
              \node[state, above left of=2, draw=red] (4) {$4$};
        
              \path
              (0) edge [loop below] node {$b$} (0)
              (0) edge [bend right] node {$a$} (1)
              (0) edge node {$\varepsilon$} (2)
              (1) edge [bend right] node {$a$} (0)
              (2) edge [bend right] node {$a$} (3)
              (2) edge [loop below] node {$b$} (2)
              (3) edge [bend right] node {$a$} (4)
              (4) edge [bend right] node {$a$} (2);
            \end{tikzpicture}
          \end{center}
    }

    \only<1-3>{
        \begin{center}
            \begin{tikzpicture}[node distance=2cm, stylename]
              \node[initial, state] (0) {$0$};
              \node[state, above of=0] (1) {$1$};
              \node[state, accepting, right of=0, xshift=1.5cm] (2) {$2$};
              \node[state, above right of=2] (3) {$3$};
              \node[state, above left of=2] (4) {$4$};
        
              \path
              (0) edge [loop below] node {$b$} (0)
              (0) edge [bend right] node {$a$} (1)
              (0) edge node {$\varepsilon$} (2)
              (1) edge [bend right] node {$a$} (0)
              (2) edge [bend right] node {$a$} (3)
              (2) edge [loop below] node {$b$} (2)
              (3) edge [bend right] node {$a$} (4)
              (4) edge [bend right] node {$a$} (2);
            \end{tikzpicture}
        \end{center}
    }
    
    \pause

      La palabra $aa$ termina en $\{0,2,4\}$. \pause
      
      Podemos pensar entonces que $\{0,2,4\}$ tendría que ser un \textbf{estado final} en el AFD equivalente.
      
  \end{frame}

  \begin{frame}{Estados de estados}{Conversión de AFN a AFD}

    \begin{center}
        \begin{tikzpicture}[node distance=2cm, stylename]
            \node[initial, state] (0) {$0$};
            \node[state, above of=0, draw=red] (1) {$1$};
            \node[state, accepting, right of=0, xshift=1.5cm, draw=red] (2) {$2$};
            \node[state, above right of=2, draw=red] (3) {$3$};
            \node[state, above left of=2] (4) {$4$};
    
            \path
            (0) edge [loop below] node {$b$} (0)
            (0) edge [bend right] node {$a$} (1)
            (0) edge node {$\varepsilon$} (2)
            (1) edge [bend right] node {$a$} (0)
            (2) edge [bend right] node {$a$} (3)
            (2) edge [loop below] node {$b$} (2)
            (3) edge [bend right] node {$a$} (4)
            (4) edge [bend right] node {$a$} (2);
        \end{tikzpicture}
    \end{center}

    La palabra $abbaa$ termina en $\{1,2,3\}$. \pause
    
    Podemos pensar entonces que $\{1,2,3\}$ también tendría que ser un \textbf{estado final} en el AFD equivalente.
      
  \end{frame}

  \begin{frame}{Conjunto de estados}{Conversión de AFN a AFD}
      Significa que el AFD equivalente, $M$, tiene que tener \textbf{estados finales} de conjuntos de estados donde se acepte la palabra en el AFN. \pause

      \bigskip

      Por tanto, $Q' = \wp(Q)$: el conjunto de estados de $M$ es igual al conjunto potencia del conjunto de estados de $N$.
  \end{frame}

  \begin{frame}{Cerradura de vacío ($\varepsilon$)}{Conversión de AFN a AFD}
    Dado a que existe una transición $\varepsilon$ entre $0$ y $2$ en $N$, claramente el estado inicial en $M$ debe considerar $\{0,2\}$.

    \begin{center}
        \begin{tikzpicture}[node distance=2cm, stylename]
          \node[initial, state] (0) {$0$};
          \node[state, above of=0] (1) {$1$};
          \node[state, accepting, right of=0, xshift=1.5cm] (2) {$2$};
          \node[state, above right of=2] (3) {$3$};
          \node[state, above left of=2] (4) {$4$};
    
          \path
          (0) edge [loop below] node {$b$} (0)
          (0) edge [bend right] node {$a$} (1)
          (0) edge node {$\varepsilon$} (2)
          (1) edge [bend right] node {$a$} (0)
          (2) edge [bend right] node {$a$} (3)
          (2) edge [loop below] node {$b$} (2)
          (3) edge [bend right] node {$a$} (4)
          (4) edge [bend right] node {$a$} (2);
        \end{tikzpicture}
    \end{center} \pause

    Para eso, hay que enfocarnos en la cerradura-$\varepsilon$, $C_\varepsilon (r)$, donde $r$ es cualquier estado del AFN $N$.      
  \end{frame}

  \begin{frame}{Cerradura de vacío ($\varepsilon$)}{Conversión de AFN a AFD}

    \begin{block}{Definición}
        Para cada estado $r$ del AFN $N$, la \alert{cerradura de vacío}, representada con $C_\varepsilon (r)$, se define como el conjunto de todos los estados de $N$ que pueden alcanzarse desde $r$, haciendo cero o más transiciones $\varepsilon$.
    \end{block} \pause

    \bigskip

    \begin{exampleblock}{Equivalencia en AFDs}
        Para cada estado $R$ del AFD $M$ (es decir, $R \subseteq Q$):
        \[C_\varepsilon (R) = \bigcup_{r \in R} C_\varepsilon (r)\]
    \end{exampleblock}

  \end{frame}

  \begin{frame}{El estado inicial}{Conversión de AFN a AFD}
      El \alert{estado inicial} $q'$ en un AFD $M$, antes de leer cualquier símbolo `real' de $\Sigma$, hace cero o más transiciones $\varepsilon$. \pause

      \bigskip

      Al momento de que $N$ lee el primer símbolo de $\Sigma$, puede estar en cualquier estado de $C_\varepsilon(q)$. Por tanto:

      \[q' = C_\varepsilon (q) = C_\varepsilon(\{q\})\]
  \end{frame}

  \begin{frame}{El conjunto de estados finales}{Conversión de AFN a AFD}

    El conjunto de \alert{estados finales} $F'$ del AFD $M$, es igual al conjunto de \textbf{todos} los elementos $R$ de $Q'$ que tienen la propiedad de que $R$ contiene al menos un estado final del AFN $N$, es decir: \pause

    \[F'= \{R \in Q' : R \cap F \neq \varnothing\}\]
      
  \end{frame}

  \begin{frame}{La función de transición}{Conversión de AFN a AFD}
      Pensemos que el AFD $M$ está en el estado $R$ y que recibe el símbolo $a$.
      En este momento, el AFN $N$ habría estado en \textbf{cualquier} estado $r$ que esté en $R$. \pause

      \bigskip

      Al leer el símbolo $a$, la máquina $N$ puede cambiar hacia \textbf{cualquier} estado de $\delta(r,a)$, y luego hacer cero o más transiciones-$\varepsilon$. Por lo que, para cada $R \in Q'$, y para cada $a \in \Sigma$, \pause

      \[ \delta'(R,a) = \bigcup_{r \in R} C_\varepsilon (\delta (r,a)) \]
  \end{frame}

  \begin{frame}{Resumen}{Conversión de AFN a AFD}
      Un autómata \textbf{finito no determinista} $N = (Q, \Sigma, \delta, q, F)$ es convertido al autómata \textbf{finito determinista} $M = (Q', \Sigma, \delta', q', F')$, donde: \pause
      
      \bigskip

      \begin{itemize}
        \itemsep1.5ex
        \item $Q' = \wp(Q)$, \pause
        \item $q'= C_\varepsilon(\{q\})$ \pause
        \item $F' = \{R \in Q' : R \cap F \neq \varnothing\}$ \pause
        \item $\delta' : Q' \times \Sigma \to Q'$, donde para cada $R \in Q'$ y para cada $a \in \Sigma$,
        \[\delta'(R,a) = \bigcup_{r \in R} C_\varepsilon (\delta (r,a))\]
      \end{itemize} \pause

      \bigskip

      ¿Qué hace falta? \pause

      Hay que eliminar transiciones de varios símbolos, partiendo en una secuencia de estados.

  \end{frame}

  \begin{frame}{Ejemplo}{Conversión de AFN a AFD}
    \begin{center}
        \begin{tikzpicture}[node distance=3cm, stylename]
            \node [initial, state] (1) {$1$};
            \node [state, accepting, right of=1] (2) {$2$};
            \node [state, below right of=1] (3) {$3$};
    
            \path
            (1) edge [bend left] node {$\varepsilon$} (2)
            (1) edge node [below] {$a$} (3)
            (2) edge node {$a$} (1)
            (3) edge node [right] {$a,b$} (2)
            (3) edge [loop below] node {$b$} (3);
        \end{tikzpicture}
    \end{center}      
  \end{frame}

  \begin{frame}{Ejemplo: complemento}{Conversión de AFN a AFD}
      Si tenemos que hacer un AF que acepte las palabras en $\{a,b\}$ que \textbf{no contienen} ni $abb$ ni $aab$... \pause

      \bigskip

      Se puede hacer primero un AFN que acepte las que contienen $abb$ o bien $aab$. \pause

      \bigskip

      Luego se convierte a AFD. \pause

      \bigskip

      Y luego se encuentra la máquina complemento.
  \end{frame}

  \begin{frame}{Ejemplo: intersección}{Conversión de AFN a AFD}
      Buscamos ahora un AFN que acepte palabras en $\{a,b\}$ con número impar de $b$s y que contiene $aab$... \pause
      
      \bigskip

      ¿Cuál es la solución? \pause $A \cap B = (A^\complement \cup B^\complement)^\complement$ \pause

      \bigskip

      \begin{itemize}
          \item Hacer los AFN de los componentes. \pause
          \item Convertirlos a AFDs. \pause
          \item Complementarlos. \pause
          \item Hacer la unión. \pause
          \item Convertir el nuevo AFN a un AFD. \pause
          \item Encontrar la máquina complemento.
      \end{itemize}
  \end{frame}
  
  
  % \section*{Referencias}
  
  % \begin{frame}[t]{Referencias}
    % \nocite{bibID01}
    % \nocite{bibID02}
    
    % \bibliographystyle{IEEE}
    % \bibliography{biblio}
  % \end{frame}
  
  \end{document}