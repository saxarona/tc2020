\documentclass[spanish, handout]{beamer}

\usepackage[utf8]{inputenc}
\usepackage[spanish, mexico]{babel}
\usepackage{hyperref}
\usepackage{xcolor}
\usepackage{color}
\usepackage{ragged2e}
\usepackage{tikz}
\usepackage{mathrsfs}

\usetikzlibrary{arrows, automata, positioning, fit, shapes.geometric, backgrounds}

\tikzset{
->, %arrow type
>=stealth', %arrow head type (bold)
shorten >=1pt, 
auto,
semithick,
initial text=$ $, %no start text
}

% \usepackage{courier}
% \usepackage{subfigure}
% \usepackage{enumerate}
% \usepackage{algorithmic}
% \usepackage{algorithm}


% \usepackage{listings}
% \usepackage{lstlinebgrd}

\newcommand\blfootnote[1]{%
\begingroup
\renewcommand\thefootnote{}\footnote{#1}%
\addtocounter{footnote}{-1}%
\endgroup
}

\usetheme{Boadilla}
\usefonttheme[onlymath]{serif}

% Sets the templates
\definecolor{navyblue}{RGB}{0, 0, 128}

\setbeamertemplate{navigation symbols}{} 
\setbeamertemplate{headline}{}
\setbeamertemplate{footline}[frame number]
\setbeamertemplate{bibliography item}[text]
\setbeamertemplate{theorems}[numbered]

\setbeamercolor{title}{fg=navyblue, bg=white}
\setbeamercolor{frametitle}{fg=navyblue, bg=white}
\setbeamercolor{structure}{fg=navyblue}
\setbeamercolor{button}{fg=white,bg=navyblue}

\setbeamercovered{transparent}

\title{Expresiones Regulares}
\subtitle{Matemáticas Computacionales \\ (TC2020)}
\author{
\texorpdfstring{
\begin{center}
    M.C. Xavier Sánchez Díaz \\
    \href{mailto:sax@itesm.mx}{\texttt{sax@itesm.mx}}
\end{center}
}
{M.C. Xavier Sánchez Díaz}
}

\institute[Tecnológico de Monterrey]{\includegraphics[scale=0.5]{../img/logo}}
\date{}

\begin{document}

\setlength{\rightskip}{0pt}

\begin{frame}[plain]
\titlepage        
\end{frame}

\begin{frame}{Tabla de contenidos}
\tableofcontents
\end{frame}

\section{Conceptos básicos de ERs}
\label{sec:basics}

\begin{frame}{¿Qué es una expresión regular?}{Conceptos básicos de Expresiones Regulares}

    Recordemos que un \alert{lenguaje} es un \textbf<3->{conjunto} de palabras aceptadas por un autómata. \pause

    \bigskip

    Si existe autómata que pueda representar al lenguaje, entonces es un lenguaje \alert{regular}. \pause

    \bigskip

    \onslide<4-> Como es un conjunto, podemos \textbf{describirlo} usando expresiones.
\end{frame}

\begin{frame}{Lenguajes regulares}{Conceptos básicos de Expresiones Regulares}
    Un lenguaje $L$ es \alert{regular} si y sólo si se cumple al menos una de las condiciones siguientes: \pause

    \bigskip

    \begin{itemize}
        \itemsep1.25em
        \item $L$ es finito. \pause
        \item $L$ es la unión o concatenación de otros lenguajes regulares $R_1$ y $R_2$: $L = R_1 \cup R_2$ o $L = R_1R_2$. \pause
        \item L es la cerradura de Kleene de algún lenguaje regular: $L = R^*$.
    \end{itemize}
\end{frame}

\begin{frame}{Lenguajes regulares}{Conceptos básicos de Expresiones Regulares}

    \textit{El lenguaje en $\{a,b\}$ de las palabras que empiecen con $00$ pero que no tengan $11$}. \pause

    \bigskip

    \begin{center}
        {\huge $00$ ALGO NO$(11)$}
    \end{center} \pause

    \bigskip
    
    Hay un patrón \textit{regular} que podemos modelar usando un AF.
\end{frame}

\begin{frame}{Lenguajes regulares}{Conceptos básicos de Expresiones Regulares}

    \textit{El lenguaje $L$ de palabras formadas por $a$ y $b$, pero que empiezan con $a$}: $aab, ab, a, abaa, \dots$. \pause

    \begin{itemize}
        \itemsep1.25em
        \item $\Sigma = \{a, b\}$ \pause
        \item $L = R_1 R_2$: \pause
            \begin{itemize}
                \item $R_1 = $ el lenguaje que contiene una $a$. \pause
                \item $R_2 = $ el lenguaje que contiene cadenas cualesquiera de $a$ y $b$. \pause
            \end{itemize}
    \end{itemize}

    \[L = \{a\}\{a,b\}^* \]
\end{frame}

\begin{frame}{Expresiones Regulares}{Conceptos básicos de Expresiones Regulares}
    Una \alert{expresión regular} es una representación textual de un \textbf{lenguaje regular}. \pause

    \begin{block}{Definición de Expresiones Regulares}
        Sea $\Sigma$ un alfabeto no vacío. \pause

        \begin{itemize}
            \item \^{} es una expresión regular. \pause
            \item $\varnothing$ es una expresión regular. \pause
            \item Para cada $\sigma \in \Sigma$, $\sigma$ es una expresión regular. \pause
            \item Si $R_1$ y $R_2$ son expresiones regulares, entonces $R_1 \cup R_2$ es una expresión regular. \pause
            \item Si $R_1$ y $R_2$ son expresiones regulares, entonces $R_1 R_2$ es una expresión regular. \pause
            \item Si $R$ es una expresión regular, entonces $R^*$ es una expresión regular.
        \end{itemize}
    \end{block}
    
\end{frame}

\begin{frame}{Operaciones}{Conceptos básicos de Expresiones Regulares}
    Las operaciones disponibles son las mismas que usábamos en autómatas, pero algunas cambian de notación: \pause

    \bigskip

    \begin{itemize}
        \item La ER $\varnothing$ representa al \textbf{lenguaje vacío} $\{\}$. \pause
        \item La \textbf{palabra vacía} $\varepsilon$ la representamos con la expresión regular $\hat{}$. \pause
        \item La \textbf{unión} se representa usualmente con la ER $+$. \pause
        \item La \textbf{cerradura o estrella de Kleene} de un lenguaje regular $R$ sigue representándose con $R^*$
    \end{itemize}
\end{frame}

\section{Ejemplos de ERs}
\label{sec:ex}

\begin{frame}{ERs sintácticamente bien formadas}{Ejemplos de ERs}
    A nivel \textit{implementación}, una ER es una secuencia de ERs que nos dicen \textbf{cómo} debería ser el lenguaje descrito por la ER completa: \pause

    \bigskip

    \onslide

    \only<2>{
        {\Large $0$} es una ER válida que describe al lenguaje $\{0\}$.
    } 
    \pause

    \only<3>{
        {\Large $01^*$} es una ER válida: \textit{La \textbf{concatenación} del lenguaje $\{0\}$ y $\{1\}^*$}---\textit{un $0$ seguido de cero o más $1$s}.
    }

    \pause

    \only<4>{
        {\Large $0 + 1$} es una ER válida: \textit{La \textbf{unión} del lenguaje $\{0\}$ y $\{1\}$}---\textit{o un $0$ o un $1$}.
    }
    
    \pause

    \only<5>{
        {\Large $(0 + 1)^*11$} es una ER válida: \textit{o $0$ o $1$ cero o más veces, seguida de dos $1$s}.
    }
\end{frame}

\begin{frame}{ERs más complejas}{Ejemplos de ERs}
    \textit{Diseñar una ER para el lenguaje de las palabras en $\{0,1\}$ que empiezan o terminan en $01$}: \pause

    \bigskip

    \begin{enumerate}
        \item Estructuramos con patrones:
        $01 \, algo \, 01$ \pause 
        \item Definimos restricciones:
        $algo$ no tiene restricción alguna, puede ser o $0$ o $1$, cero o más veces.\pause
        \item Generamos una ER para el patrón $algo$:
        $(0 + 1)^*$ \pause
        \item Finalmente, concatenamos las ERs para hacer una nueva ER.
    \end{enumerate}

    \[01(0+1)^*01\]
\end{frame}


% \section*{Referencias}
% \begin{frame}[t]{Referencias}
% \nocite{bibID01}
% \nocite{bibID02}

% \bibliographystyle{IEEE}
% \bibliography{biblio}
% \end{frame}

\end{document}