\documentclass[spanish, handout]{beamer}

\usepackage[utf8]{inputenc}
\usepackage[spanish, mexico]{babel}
\usepackage{hyperref}
\usepackage{xcolor}
\usepackage{color}
\usepackage{ragged2e}
\usepackage{tikz}
\usepackage{mathrsfs}

\usetikzlibrary{arrows, automata, positioning, fit, shapes.geometric, backgrounds}

\tikzset{
    automata/.style={
        ->, %arrow type
        >=stealth', %arrow head type (bold)
        shorten >=1pt, 
        auto,
        %semithick,
        initial text=$ $, %no start text
    }
}

% \usepackage{courier}
% \usepackage{subfigure}
% \usepackage{enumerate}
% \usepackage{algorithmic}
% \usepackage{algorithm}


% \usepackage{listings}
% \usepackage{lstlinebgrd}

\newcommand\blfootnote[1]{%
\begingroup
\renewcommand\thefootnote{}\footnote{#1}%
\addtocounter{footnote}{-1}%
\endgroup
}

\usetheme{Boadilla}
\usefonttheme[onlymath]{serif}

% Sets the templates
\definecolor{navyblue}{RGB}{0, 0, 128}

\setbeamertemplate{navigation symbols}{} 
\setbeamertemplate{headline}{}
\setbeamertemplate{footline}[frame number]
\setbeamertemplate{bibliography item}[text]
\setbeamertemplate{theorems}[numbered]

\setbeamercolor{title}{fg=navyblue, bg=white}
\setbeamercolor{frametitle}{fg=navyblue, bg=white}
\setbeamercolor{structure}{fg=navyblue}
\setbeamercolor{button}{fg=white,bg=navyblue}

\setbeamercovered{transparent}

\title{Conversión entre Expresiones Regulares y Autómatas Finitos}
\subtitle{Matemáticas Computacionales \\ (TC2020)}
\author{
\texorpdfstring{
\begin{center}
    M.C. Xavier Sánchez Díaz \\
    \href{mailto:sax@itesm.mx}{\texttt{sax@itesm.mx}}
\end{center}
}
{M.C. Xavier Sánchez Díaz}
}

\institute[Tecnológico de Monterrey]{\includegraphics[scale=0.5]{../img/logo}}
\date{}

\begin{document}

\setlength{\rightskip}{0pt}

\begin{frame}[plain]
\titlepage
\end{frame}

\begin{frame}{Tabla de contenidos}
\tableofcontents
\end{frame}

\section{Preliminares}
\label{sec:pre}

\begin{frame}{¿Por qué convertir entre ERs y AFs?}{Conversión entre ERs y AFs}
    \begin{itemize}
        \itemsep1.5em
        \item Los AFs son más \textbf{fáciles de entender} (y por tanto, de \textbf{diseñar}). \pause
        \item Los AFs son muy \textbf{aparatosos}: describen qué pasa en cada estado y con cada acción. \pause
        \item Las ERs son más \textbf{compactas}: sólo describen cómo va a ser el \textit{output} del autómata---el lenguaje de palabras que acepta. \pause
        \item Pasar de una condición de aceptación de un problema a un AF puede ser \textbf{más complicado}. \pause
        \item Explicar una condición de aceptación de un problema, a otra persona mediante una ER puede ser \textbf{muy complicado}.
    \end{itemize}
\end{frame}

\section{De AFs a ERs}

\begin{frame}{Ejemplo: BSL (4-AQA)}{De AFs a ERs}
    \begin{center}
        \begin{tikzpicture}[node distance=2.3cm, automata]

            \node[state, initial] (q0) {$q0$};
            \node[state] (q1) [above right of=q0] {$q1$};
            \node[state] (q2) [below right of=q0] {$q2$};
            \node[state] (q3) [below right of=q1] {$q3$};
            \node[state] (q4) [right of=q1] {$q4$};
            \node[state, accepting] (q5) [below right of=q4] {$q5$};

            \path
            (q0) edge node {$0$} (q2)
            (q0) edge node {$1$} (q1)
            (q1) edge node {$0$} (q4)
            (q1) edge node {$1$} (q3)
            (q2) edge node {$0$} (q3)
            (q2) edge node [below] {$1$} (q5)
            (q3) edge node {$1$} (q5)
            (q4) edge node {$0,1$} (q5)
            (q5) edge [loop below] node {$0,1$} (q5);
        \end{tikzpicture}
    \end{center} \pause

    Es un AFN ya que $q3$ no tiene transición $0$.
\end{frame}

\begin{frame}{Paso a paso}{De AFs a ERs}
    \begin{enumerate}
        \itemsep1.5em
        \item Agregamos un nuevo estado inicial y otro final por medio de transiciones vacías. \pause
        \item Empezando desde el \textbf{final} y yendo hacia atrás, reemplazamos transiciones por sus equivalencias en ERs: \pause
        \begin{enumerate}
            \itemsep1.1em
            \item Primero cambiamos sus transiciones por ERs. \pause
            \item Después cambiamos las transiciones que llegan al estado. \pause
            \item Marcamos el nodo, \textit{uniendo o concatenando} las ERs. \pause
            \item Varias ERs yendo de un estado a otro pueden reducirse a una sola \textbf{uniéndolas}. \pause
        \end{enumerate}
        \item Continuamos reemplazando hasta tener una sola ER.
    \end{enumerate}
\end{frame}

%adding s and f
\begin{frame}{Ejemplo: BSL (4-AQA)}{De AFs a ERs}
    \begin{center}
        \begin{tikzpicture}[node distance=2.3cm, automata]

            \node[state, initial] (s) {$s$};
            \node[state] (q0) [right of=s]{$q0$};
            \node[state] (q1) [above right of=q0] {$q1$};
            \node[state] (q2) [below right of=q0] {$q2$};
            \node[state] (q3) [below right of=q1] {$q3$};
            \node[state] (q4) [right of=q1] {$q4$};
            \node[state] (q5) [below right of=q4] {$q5$};
            \node[state, accepting] (f) [right of=q5] {$f$};

            \path
            (s) edge node {$\varepsilon$} (q0)
            (q0) edge node {$0$} (q2)
            (q0) edge node {$1$} (q1)
            (q1) edge node {$0$} (q4)
            (q1) edge node {$1$} (q3)
            (q2) edge node {$0$} (q3)
            (q2) edge node [below] {$1$} (q5)
            (q3) edge node {$1$} (q5)
            (q4) edge node {$0,1$} (q5)
            (q5) edge [loop below] node {$0,1$} (q5)
            (q5) edge node {$\varepsilon$} (f);
        \end{tikzpicture}
    \end{center}
\end{frame}

% changing q5 and its actions
\begin{frame}{Ejemplo: BSL (4-AQA)}{De AFs a ERs}
    \begin{center}
        \begin{tikzpicture}[node distance=2.3cm, automata]

            \node[state, initial] (s) {$s$};
            \node[state] (q0) [right of=s]{$q0$};
            \node[state] (q1) [above right of=q0] {$q1$};
            \node[state] (q2) [below right of=q0] {$q2$};
            \node[state] (q3) [below right of=q1] {$q3$};
            \node[state] (q4) [right of=q1] {$q4$};
            \node[state] (q5) [below right of=q4] {$q5$};
            \node[state, accepting] (f) [right of=q5] {$f$};

            \path
            (s) edge node {$\varepsilon$} (q0)
            (q0) edge node {$0$} (q2)
            (q0) edge node {$1$} (q1)
            (q1) edge node {$0$} (q4)
            (q1) edge node {$1$} (q3)
            (q2) edge node {$0$} (q3)
            (q2) edge node [below] {$1$} (q5)
            (q3) edge node {$1$} (q5)
            (q4) edge node {$0,1$} (q5)
            (q5) edge node {$(0+1)^*$} (f);
        \end{tikzpicture}
    \end{center}
\end{frame}

% changing actions leading to q5
\begin{frame}{Ejemplo: BSL (4-AQA)}{De AFs a ERs}
    \begin{center}
        \begin{tikzpicture}[node distance=2.3cm, automata]

            \node[state, initial] (s) {$s$};
            \node[state] (q0) [right of=s]{$q0$};
            \node[state] (q1) [above right of=q0] {$q1$};
            \node[state] (q2) [below right of=q0] {$q2$};
            \node[state] (q3) [below right of=q1] {$q3$};
            \node[state, fill=gray!50] (q4) [right of=q1] {$q4$};
            \node[state] (q5) [below right of=q4] {$q5$};
            \node[state, accepting] (f) [right of=q5] {$f$};

            \path
            (s) edge node {$\varepsilon$} (q0)
            (q0) edge node {$0$} (q2)
            (q0) edge node {$1$} (q1)
            (q1) edge node {$0$} (q4)
            (q1) edge node {$1$} (q3)
            (q2) edge node {$0$} (q3)
            (q2) edge node [below] {$1$} (q5)
            (q3) edge node {$1$} (q5)
            (q4) edge node {$0+1$} (q5)
            (q5) edge node {$(0+1)^*$} (f);
        \end{tikzpicture}
    \end{center}
\end{frame}

% removing q4 and reorienting coming actions to q5
\begin{frame}{Ejemplo: BSL (4-AQA)}{De AFs a ERs}
    \begin{center}
        \begin{tikzpicture}[node distance=2.3cm, automata]

            \node[state, initial] (s) {$s$};
            \node[state] (q0) [right of=s]{$q0$};
            \node[state] (q1) [above right of=q0] {$q1$};
            \node[state] (q2) [below right of=q0] {$q2$};
            \node[state, fill=gray!50] (q3) [below right of=q1] {$q3$};
            \node[state] (q5) [right of=q3] {$q5$};
            \node[state, accepting] (f) [right of=q5] {$f$};

            \path
            (s) edge node {$\varepsilon$} (q0)
            (q0) edge node {$0$} (q2)
            (q0) edge node {$1$} (q1)
            (q1) edge [bend left] node {$0(0+1)$} (q5)
            (q1) edge node {$1$} (q3)
            (q2) edge node {$0$} (q3)
            (q2) edge node [below] {$1$} (q5)
            (q3) edge node {$1$} (q5)
            (q5) edge node {$(0+1)^*$} (f);
        \end{tikzpicture}
    \end{center}
\end{frame}

% removing q3 and reorienting coming actions to q5
\begin{frame}{Ejemplo: BSL (4-AQA)}{De AFs a ERs}
    \begin{center}
        \begin{tikzpicture}[node distance=2.3cm, automata]

            \node[state, initial] (s) {$s$};
            \node[state] (q0) [right of=s]{$q0$};
            \node[state] (q1) [above right of=q0] {$q1$};
            \node[state] (q2) [below right of=q0] {$q2$};
            \node[state] (q5) [right of=q0, xshift=2cm] {$q5$};
            \node[state, accepting] (f) [right of=q5] {$f$};

            \path
            (s) edge node {$\varepsilon$} (q0)
            (q0) edge node {$0$} (q2)
            (q0) edge node {$1$} (q1)
            (q1) edge [bend left] node {$0(0+1)$} (q5)
            (q1) edge node [below] {$11$} (q5)
            (q2) edge [bend left] node {$01$} (q5)
            (q2) edge node [below] {$1$} (q5)
            (q5) edge node {$(0+1)^*$} (f);
        \end{tikzpicture}
    \end{center}
\end{frame}

% joining incoming actions for q5 into a single RE
\begin{frame}{Ejemplo: BSL (4-AQA)}{De AFs a ERs}
    \begin{center}
        \begin{tikzpicture}[node distance=2.3cm, automata]

            \node[state, initial] (s) {$s$};
            \node[state] (q0) [right of=s]{$q0$};
            \node[state,fill=gray!50] (q1) [above right of=q0] {$q1$};
            \node[state] (q2) [below right of=q0] {$q2$};
            \node[state] (q5) [right of=q0, xshift=2cm] {$q5$};
            \node[state, accepting] (f) [right of=q5] {$f$};

            \path
            (s) edge node {$\varepsilon$} (q0)
            (q0) edge node {$0$} (q2)
            (q0) edge node {$1$} (q1)
            (q1) edge [bend left] node {$11 + 0(0+1)$} (q5)
            (q2) edge [bend right] node {$1 + 01$} (q5)
            (q5) edge node {$(0+1)^*$} (f);
        \end{tikzpicture}
    \end{center}
\end{frame}

% removing q1 and reorienting incoming actions to q5
\begin{frame}{Ejemplo: BSL (4-AQA)}{De AFs a ERs}
    \begin{center}
        \begin{tikzpicture}[node distance=2.3cm, automata]

            \node[state, initial] (s) {$s$};
            \node[state] (q0) [right of=s]{$q0$};
            \node[state,fill=gray!50] (q2) [below right of=q0] {$q2$};
            \node[state] (q5) [right of=q0, xshift=2cm] {$q5$};
            \node[state, accepting] (f) [right of=q5] {$f$};

            \path
            (s) edge node {$\varepsilon$} (q0)
            (q0) edge node {$0$} (q2)
            (q0) edge [bend left] node {$1(11 + 0(0+1))$} (q5)
            (q2) edge [bend right] node {$1 + 01$} (q5)
            (q5) edge node {$(0+1)^*$} (f);
        \end{tikzpicture}
    \end{center}
\end{frame}

% removing q2 and reorienting incoming actions to q5
\begin{frame}{Ejemplo: BSL (4-AQA)}{De AFs a ERs}
    \begin{center}
        \begin{tikzpicture}[node distance=2.3cm, automata]

            \node[state, initial] (s) {$s$};
            \node[state] (q0) [right of=s]{$q0$};
            \node[state] (q5) [right of=q0, xshift=2cm] {$q5$};
            \node[state, accepting] (f) [right of=q5] {$f$};

            \path
            (s) edge node {$\varepsilon$} (q0)
            (q0) edge [bend left] node {$1(11 + 0(0+1))$} (q5)
            (q0) edge [bend right] node {$0(1 + 01)$} (q5)
            (q5) edge node {$(0+1)^*$} (f);
        \end{tikzpicture}
    \end{center}
\end{frame}

% joining q0 actions for q5 into a single RE (END)
\begin{frame}{Ejemplo: BSL (4-AQA)}{De AFs a ERs}
    \begin{center}
        \begin{tikzpicture}[node distance=2.3cm, automata]

            \node[state, initial] (s) {$s$};
            \node[state, fill=gray!50] (q0) [right of=s]{$q0$};
            \node[state, fill=gray!50] (q5) [right of=q0, xshift=2cm] {$q5$};
            \node[state, accepting] (f) [right of=q5] {$f$};

            \path
            (s) edge node {$\varepsilon$} (q0)
            (q0) edge node[above, yshift=0.5cm] {$1(11 + 0(0+1)) + 0(1 + 01)$} (q5)
            (q5) edge node {$(0+1)^*$} (f);
        \end{tikzpicture}
    \end{center}
\end{frame}

\begin{frame}{Ejemplo: BSL (4-AQA)}{De AFs a ERs}
    \begin{center}
        {\huge $(1(11 + 0(0+1)) + 0(1 + 01))(0+1)^*$}
    \end{center}
\end{frame}

\begin{frame}{Paso a paso}{Conversión de ERs a AFs}
    
    Para convertir de una ER a un AF hay que seguir el proceso inverso, el cual parte del mismo principio:

    \bigskip

    \begin{enumerate}
        \itemsep1.5em
        \item Empezamos con un autómata \textit{genérico} con un estado inicial y uno final, unidos por la ER. A este AF le llamaremos \alert{Gráfica de Transición}. \pause
        \item \textit{Partimos} la ER en ERs más pequeñas (vía \textbf{unión}, por ejemplo).
        \item Reemplazamos cada ER con un estado-acción.
    \end{enumerate}
\end{frame}

\begin{frame}{Generalizando}{Conversión entre ERs y AFs}
    
    \begin{columns}
        \begin{column}{0.3\textwidth}
            \begin{center}
                {\huge $R_1 R_2$}
            \end{center}
        \end{column}

        \begin{column}{0.7\textwidth}
            \begin{center}
                \begin{tikzpicture}[node distance=2cm, automata]
        
                    \node[state, initial] (s) {$s$};
                    \node[state] (k) [right of=s]{$k$};
                    \node[state, accepting] (f) [right of=k] {$f$};
        
                    \path
                    (s) edge node {$R_1$} (k)
                    (k) edge node {$R_2$} (f);
                \end{tikzpicture}
            \end{center}
        \end{column}
    \end{columns} \pause

    \bigskip

    \begin{columns}
        \begin{column}{0.3\textwidth}
            \begin{center}
                {\huge $R_1 + R_2$}    
            \end{center}
        \end{column}
        
        \begin{column}{0.7\textwidth}
            \begin{center}
                \begin{tikzpicture}[node distance=2.5cm, automata]
        
                    \node[state, initial] (s) {$s$};
                    \node[state, accepting] (f) [right of=s] {$f$};
        
                    \path
                    (s) edge [bend left] node {$R_1$} (f)
                    (s) edge [bend right] node {$R_2$} (f);
                \end{tikzpicture}
            \end{center}
        \end{column}
    \end{columns} \pause
    
    \bigskip

    \begin{columns}
        \begin{column}{0.3\textwidth}
            \begin{center}
                {\huge $R^*$}
            \end{center}
        \end{column}

        \begin{column}{0.7\textwidth}
            \begin{center}
                \begin{tikzpicture}[node distance=2cm, automata]
        
                    \node[state, initial] (s) {$s$};
                    \node[state] (k) [right of=s]{$k$};
                    \node[state, accepting] (f) [right of=k] {$f$};
        
                    \path
                    (s) edge node {$\varepsilon$} (k)
                    (k) edge [loop above] node {$R$} (k)
                    (k) edge node {$\varepsilon$} (f);
                \end{tikzpicture}
            \end{center}
        \end{column}
    \end{columns}
\end{frame}

% \section*{Referencias}
% \begin{frame}[t]{Referencias}
% \nocite{bibID01}
% \nocite{bibID02}

% \bibliographystyle{IEEE}
% \bibliography{biblio}
% \end{frame}

\end{document}