\documentclass[spanish, handout]{beamer}

\usepackage[utf8]{inputenc}
\usepackage[spanish, mexico]{babel}
\usepackage{hyperref}
\usepackage{xcolor}
\usepackage{color}
\usepackage{ragged2e}
\usepackage{tikz}
\usepackage{mathrsfs}
\usepackage{textcomp}

\usetikzlibrary{arrows, automata, positioning, fit, shapes.geometric, backgrounds}
\tikzset{%
    stylename/.style={->, >=stealth', shorten >=1pt, auto, semithick, initial text=$ $}
}

% \usepackage{courier}
% \usepackage{subfigure}
% \usepackage{enumerate}
% \usepackage{algorithmic}
% \usepackage{algorithm}


% \usepackage{listings}
% \usepackage{lstlinebgrd}

\newcommand\blfootnote[1]{%
\begingroup
\renewcommand\thefootnote{}\footnote{#1}%
\addtocounter{footnote}{-1}%
\endgroup
}

\usetheme{Boadilla}
%\usecolortheme{default}
\usefonttheme[onlymath]{serif}
\useoutertheme{infolines}

% Sets the templates
\definecolor{navyblue}{RGB}{0, 0, 128}
\definecolor{crimson}{RGB}{128, 16, 0}

\setbeamertemplate{navigation symbols}{}
\setbeamertemplate{headline}{}
\setbeamertemplate{title page}[default][colsep=-4bp,rounded=true]
\setbeamertemplate{footline}[frame number]
\setbeamertemplate{bibliography item}[text]
\setbeamertemplate{theorems}[numbered]

\setbeamercolor{title}{fg=navyblue, bg=white}
\setbeamercolor{frametitle}{fg=navyblue, bg=white}
\setbeamercolor{structure}{fg=navyblue}
\setbeamercolor{button}{fg=white,bg=navyblue}


\setbeamercovered{transparent}

\title{Gramáticas Regulares y Autómatas Finitos}
\subtitle{Matemáticas Computacionales \\ (TC2020)}
\author{
\texorpdfstring{
\begin{center}
    M.C. Xavier Sánchez Díaz \\
    \href{mailto:sax@tec.mx}{\texttt{sax@tec.mx}}
\end{center}
}
{M.C. Xavier Sánchez Díaz}
}

\institute[Tecnológico de Monterrey]{\includegraphics[scale=0.5]{../img/logo}}
\date{}

\begin{document}


\setlength{\rightskip}{0pt}

\begin{frame}[plain]
\titlepage
\end{frame}

% \begin{frame}{Tabla de contenidos}
% \tableofcontents
% \end{frame}


\begin{frame}{Equivalencias de GRs y AFs}

    Las \textbf{variables} de una GR son los \alert{estados} de un AF

    \begin{columns}
        \begin{column}{0.3\textwidth}
            \begin{enumerate}
                \itemsep1.5ex
                \item $S \to aI$
                \item $S \to bI$
                \item $I \to aP$
                \item $I \to bP$
                \item $I \to a$
                \item $P \to aI$
                \item $P \to bI$
            \end{enumerate}
        \end{column}
        \begin{column}{0.7\textwidth}
            \begin{center}
                \begin{tikzpicture}[stylename, node distance=1.5cm]

                    \node[initial,state] (S) {$S$};
                    \node[state] (I) [right of=S] {$I$};
                    \node[state] (P) [right of=I]{$P$};                    
                \end{tikzpicture}
            \end{center}
        \end{column}
    \end{columns}
\end{frame}

\begin{frame}{Equivalencias de GRs y AFs}

    Las \textbf{reglas} de una GR son las \alert{transiciones} de un AF: $X \to \sigma Y = (X,\sigma,Y)$.

    \begin{columns}
        \begin{column}{0.3\textwidth}
            \begin{enumerate}
                \itemsep1.5ex
                \item $S \to aI$
                \item $S \to bI$
                \item $I \to aP$
                \item $I \to bP$
                \item $I \to a$
                \item $P \to aI$
                \item $P \to bI$
            \end{enumerate}
        \end{column}
        \begin{column}{0.7\textwidth}
            \begin{center}
                \begin{tikzpicture}[stylename, node distance=2.5cm]

                    \node[initial,state] (S) {$S$};
                    \node[state] (I) [right of=S] {$I$};
                    \node[state] (P) [right of=I]{$P$};
                    
                    \path
                    (S) edge [bend left] node {$a$} (I)
                    (S) edge [bend right] node {$b$} (I)
                    (I) edge [bend left=65] node {$a$} (P)
                    (I) edge [bend right=65] node [below] {$b$} (P)
                    (P) edge [bend left] node {$a$} (I)
                    (P) edge [bend right] node {$b$} (I);
                \end{tikzpicture}
            \end{center}
        \end{column}
    \end{columns}
\end{frame}

\begin{frame}{Equivalencias de GRs y AFs}

    Las \textbf{reglas} de una GR son las \alert{transiciones} de un AF: $X \to \sigma = (X,\sigma,Z)$, donde $Z$ es un único estado final.

    \begin{columns}
        \begin{column}{0.3\textwidth}
            \begin{enumerate}
                \itemsep1.5ex
                \item $S \to aI$
                \item $S \to bI$
                \item $I \to aP$
                \item $I \to bP$
                \item \alert{$I \to a$}
                \item $P \to aI$
                \item $P \to bI$
            \end{enumerate}
        \end{column}
        \begin{column}{0.7\textwidth}
            \begin{center}
                \begin{tikzpicture}[stylename, node distance=2.5cm]

                    \node[initial,state] (S) {$S$};
                    \node[state] (I) [right of=S] {$I$};
                    \node[state] (P) [right of=I] {$P$};
                    \node[state, accepting] (Z) [below of=I] {$Z$};
                    
                    \path
                    (S) edge [bend left] node {$a$} (I)
                    (S) edge [bend right] node {$b$} (I)
                    (I) edge [bend left=65] node {$a$} (P)
                    (I) edge [bend right=65] node [below] {$b$} (P)
                    (P) edge [bend left] node {$a$} (I)
                    (P) edge [bend right] node {$b$} (I)
                    (I) edge node [left] {$a$} (Z);

                \end{tikzpicture}
            \end{center}
        \end{column}
    \end{columns}
\end{frame}

% \section*{Referencias}
% \begin{frame}[t]{Referencias}
% \nocite{bibID01}
% \nocite{bibID02}

% \bibliographystyle{IEEE}
% \bibliography{biblio}
% \end{frame}

\end{document}