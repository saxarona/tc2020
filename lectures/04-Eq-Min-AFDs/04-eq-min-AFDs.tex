\documentclass[spanish]{beamer}

\usepackage[utf8]{inputenc}
\usepackage[spanish, mexico]{babel}
\usepackage{hyperref}
\usepackage{xcolor}
\usepackage{color}
\usepackage{ragged2e}
\usepackage{tikz}
\usepackage{tikz-qtree}
\usepackage{mathrsfs}

\usetikzlibrary{arrows, automata, positioning, fit, shapes.geometric, backgrounds}

\tikzset{
    stylename/.style={
    ->, %arrow type
    >=stealth', %arrow head type (bold)
    shorten >=1pt, 
    auto,
    %semithick,
    initial text=$ $, %no start text
    }
}

% \usepackage{courier}
% \usepackage{subfigure}
% \usepackage{enumerate}
% \usepackage{algorithmic}
% \usepackage{algorithm}


% \usepackage{listings}
% \usepackage{lstlinebgrd}

\newcommand\blfootnote[1]{%
  \begingroup
  \renewcommand\thefootnote{}\footnote{#1}%
  \addtocounter{footnote}{-1}%
  \endgroup
}

\usetheme{Boadilla}
\usefonttheme[onlymath]{serif}

% Sets the templates
\definecolor{navyblue}{RGB}{0, 0, 128}
\definecolor{crimson}{RGB}{128, 16, 0}

\setbeamertemplate{navigation symbols}{}
\setbeamertemplate{headline}{}
\setbeamertemplate{title page}[default][colsep=-4bp,rounded=true]
\setbeamertemplate{footline}[frame number]
\setbeamertemplate{bibliography item}[text]
\setbeamertemplate{theorems}[numbered]

\setbeamercolor{title}{fg=navyblue, bg=white}
\setbeamercolor{frametitle}{fg=navyblue, bg=white}
\setbeamercolor{structure}{fg=navyblue}
\setbeamercolor{button}{fg=white,bg=navyblue}

\setbeamercovered{transparent}

\title{Equivalencia y Minimización de Autómatas Finitos Deterministas}
\subtitle{Matemáticas Computacionales \\ (TC2020)}
\author{
    \texorpdfstring{
        \begin{center}
            M.C. Xavier Sánchez Díaz \\
            \href{mailto:mail@tec.mx}{\texttt{mail@tec.mx}}
        \end{center}
    }
    {M.C. Xavier Sánchez Díaz}
}

\institute[Tecnológico de Monterrey]{\includegraphics[scale=0.5]{../img/logo}}
\date{}

\begin{document}

\setlength{\rightskip}{0pt}

\begin{frame}[plain]
    \titlepage        
\end{frame}

\begin{frame}{Tabla de contenidos}
    \tableofcontents
\end{frame}

\section{Equivalencia de AFDs}
\label{sec:equiv}

\begin{frame}{Definición de Equivalencia}{Equivalencia de AFDs}
    \begin{definition}
        Dos autómatas $M_1$ y $M_2$ son \alert{equivalentes}, $M_1 \equiv M_2$, cuando aceptan exactamente el mismo lenguaje.
    \end{definition}
\end{frame}

\begin{frame}[t]{Equivalencia}{Equivalencia de AFDs}
    ¿Son estos dos autómatas equivalentes?
    \begin{center}
        \begin{tikzpicture}[node distance=1.5cm, stylename]
            \node[initial, state, accepting] (A) {$A$};
            \node[state, right of=A] (B) {$B$};
            \path
            (A) edge [bend left] node {$a$} (B)
            (A) edge [loop below] node {$b$} (A)
            (B) edge [bend left] node {$a$} (A)
            (B) edge [loop below] node {$b$} (B);
        \end{tikzpicture}

        \bigskip
    
        \begin{tikzpicture}[node distance=1.5cm, stylename]
            \node[initial, state, accepting] (A) {$A$};
            \node[state, right of=A] (B) {$B$};
            \node[state, accepting, right of=B] (C) {$C$};
            \path
            (A) edge [bend left] node {$a$} (B)
            (A) edge [loop below] node {$b$} (A)
            (B) edge [bend left] node {$a$} (C)
            (B) edge [loop below] node {$b$} (B)
            (C) edge [loop below] node {$a$} (C)
            (C) edge [bend left] node {$b$} (B);
        \end{tikzpicture}
    \end{center}    
\end{frame}

\begin{frame}{¿Cómo podemos probarlo?}{Equivalencia de AFDs}
    \textbf{Sistemáticamente}, probando con las palabras de $\sigma^* = \{\varepsilon, a, b, aa, ab, \dots\}$ \pause
    \bigskip
    ¿Qué pasa si no son equivalentes? \pause
    Simplemente nunca acabaremos. \pause
    
    \bigskip

    Podemos probar todas las posibilidades mediante un \textbf{árbol de estados incompatibles}.
\end{frame}

\begin{frame}{Árbol de comparación}{Equivalencia de AFDs}
     \begin{columns}
        \begin{column}{0.5\textwidth}
          \begin{center}
            \begin{tikzpicture}[node distance=1.5cm, stylename]
                \node[initial, state, accepting] (A) {$A$};
                \node[state, right of=A] (B) {$B$};
                \path
                (A) edge [bend left] node {$a$} (B)
                (A) edge [loop below] node {$b$} (A)
                (B) edge [bend left] node {$a$} (A)
                (B) edge [loop below] node {$b$} (B);
            \end{tikzpicture}
    
            \bigskip
        
            \begin{tikzpicture}[node distance=1.5cm, stylename]
                \node[initial, state, accepting] (A) {$A'$};
                \node[state, right of=A] (B) {$B'$};
                \node[state, accepting, right of=B] (C) {$C'$};
                \path
                (A) edge [bend left] node {$a$} (B)
                (A) edge [loop below] node {$b$} (A)
                (B) edge [bend left] node {$a$} (C)
                (B) edge [loop below] node {$b$} (B)
                (C) edge [loop below] node {$a$} (C)
                (C) edge [bend left] node {$b$} (B);
            \end{tikzpicture}
        \end{center}
        \end{column}
        \begin{column}{0.5\textwidth}
          \begin{center}
            \begin{tikzpicture}[level distance=1.5cm, stylename]
              \Tree
              [.\node(aa){$(A,A')$};
                \edge node[midway]{$a$};
                [.\node(bb){$(B,B')$};
                  \edge node[midway]{$a$};
                  [.$(A,C')$
                  \edge node[midway, left]{$a$};
                  [.$(B,C')\alert{\times}$ ]
                  \edge node[midway, right]{$b$};
                  [.$(A,B')\alert{\times}$ ] ] ] ]

              \draw (aa) ..controls +(north east:2) and +(north:2).. node[above] {$b$} (aa);
              \draw (bb) ..controls +(north east:2) and +(east:2).. node[above] {$b$} (bb);
          \end{tikzpicture}
          \end{center}
        \end{column}
     \end{columns}
\end{frame}

\begin{frame}[t,plain]
  \begin{FlushLeft}
    \begin{tikzpicture}[node distance=1.8cm, stylename]
      \node[initial, state] (s0) {$q_0$};
      \node[state, right of=s0] (s1) {$q_1$};
      \node[state, below of=s0] (s4) {$q_4$};
      \node[state, right of=s4] (s5) {$q_5$};
      \node[state, accepting, right of=s1] (s2) {$q_2$};
      \node[state, right of=s2] (s3) {$q_3$};
      \node[state, below of=s2] (s6) {$q_6$};
      \node[state, accepting, right of=s6] (s7) {$q_7$};
  
      \path
      (s0) edge node {$a$} (s1)
      (s0) edge node [left] {$b$} (s4)
      (s1) edge [bend left] node {$a$} (s5)
      (s1) edge node {$b$} (s2)
      (s2) edge node {$a$} (s3)
      (s2) edge node {$b$} (s6)
      (s3) edge [loop above] node {$a,b$} (s3)
      (s4) edge node {$a$} (s1)
      (s4) edge [loop below] node {$b$} (s4)
      (s5) edge [bend left] node [right] {$a$} (s1)
      (s5) edge node {$b$} (s4)
      (s6) edge node {$a$} (s3)
      (s6) edge [bend left] node [below] {$b$} (s7)
      (s7) edge node [right] {$a$} (s3)
      (s7) edge [bend left] node {$b$} (s6);
  \end{tikzpicture}
  \end{FlushLeft}

  \vspace{-1.5cm}

  \begin{flushright}
    \begin{tikzpicture}[node distance=1.8cm, stylename]
      \node[initial, state] (B) {$B$};
      \node[state, right of=B] (C) {$C$};
      \node[state, accepting, right of=C] (A) {$A$};
      \node[state, right of=A] (D) {$D$};
      \node[state, below right of=A] (E) {$E$};
  
      \path
      (B) edge [bend left] node {$a$} (C)
      (B) edge [loop below] node {$b$} (B)
      (C) edge [bend left] node {$a$} (B)
      (C) edge node {$b$} (A)
      (A) edge node {$a$} (D)
      (A) edge [bend left] node [below] {$b$} (E)
      (D) edge [loop above] node {$a,b$} (D)
      (E) edge node [right] {$a$} (D)
      (E) edge [bend left] node {$b$} (A);
  \end{tikzpicture}
  \end{flushright}
  
\end{frame}

\section{Simplificación de AFDs}
\label{sec:simplify}

\begin{frame}{¿Por qué simplificar AFDs?}{Simplificación de AFDs}
  Una máquina $M$ puede tener \alert{estados redundantes}. \pause

  \begin{center}
    \begin{tikzpicture}[node distance=2.5cm, stylename]
      \node[initial, state, accepting] (q3) {$q_3$};
      \node[state, accepting, below right of=q3] (q4) {$q_4$};
      \node[state, above right of=q3] (q5) {$q_5$};
  
      \path
      (q3) edge [bend right] node [below] {$a$} (q4)
      (q3) edge [bend right] node {$b$} (q5)
      (q4) edge [bend right] node {$a$} (q3)
      (q4) edge [bend right] node {$b$} (q5)
      (q5) edge [loop above] node {$a,b$} (q5);
    \end{tikzpicture}
  \end{center}
\end{frame}

\begin{frame}{Eliminación de estados equivalentes}{Simplificación de AFDs}
  Borrar transiciones:

  \begin{center}
    \begin{tikzpicture}[node distance=2.5cm, stylename]
      \node[initial, state, accepting] (q3) {$q_3$};
      \node[state, accepting, below right of=q3] (q4) {$q_4$};
      \node[state, above right of=q3] (q5) {$q_5$};
  
      \path
      (q3) edge [bend right] node [below] {$a$} (q4)
      (q3) edge [bend right] node {$b$} (q5)
      (q5) edge [loop above] node {$a,b$} (q5);
    \end{tikzpicture}
  \end{center}
\end{frame}

\begin{frame}{Eliminación de estados equivalentes}{Simplificación de AFDs}
  Redirigir transiciones:

  \begin{center}
    \begin{tikzpicture}[node distance=2.5cm, stylename]
      \node[initial, state, accepting] (q3) {$q_3$};
      \node[state, above right of=q3] (q5) {$q_5$};
  
      \path
      (q3) edge [loop below] node {$a$} (q3)
      (q3) edge [bend right] node {$b$} (q5)
      (q5) edge [loop above] node {$a,b$} (q5);
    \end{tikzpicture}
  \end{center}
\end{frame}

\begin{frame}{Deducción de estados distinguibles}{Simplificación de AFDs}
  \begin{itemize}
    \item Dos estados son \alert{distinguibles} si son \textbf{incompatibles}: uno es final y el otro es no final. \pause
    \bigskip
    \item Si tenemos transiciones $\delta (p_0,\sigma) = p$ y $\delta (q_0, \sigma) = q$, donde $p,q$ son distinguibles, entonces también $p_0$ y $q_0$ son distinguibles.
  \end{itemize}
\end{frame}

\begin{frame}{Deducción de estados distinguibles}{Simplificación de AFDs}
  \begin{columns}
    \begin{column}{0.42\textwidth}
      \begin{center}
        \begin{tikzpicture}[node distance=1.6cm, stylename]
          \node[state, initial, accepting] (5) {$5$};
          \node[state, accepting, below right of=5] (3) {$3$};
          \node[state, accepting, above right of=5] (4) {$4$};
          \node[state, accepting, right of=4] (2) {$2$};
          \node[state, right of=3] (1) {$1$};

          \path
          (5) edge node {$a$} (4)
          (5) edge node {$b$} (3)
          (4) edge [loop above] node {$a$} (4)
          (4) edge [bend left] node {$b$} (2)
          (3) edge node {$a$} (4)
          (3) edge node {$b$} (1)
          (2) edge [bend left] node {$a$} (4)
          (2) edge node {$b$} (1)
          (1) edge [loop below] node {$a,b$} (1);
        \end{tikzpicture}
      \end{center}
    \end{column}
    \begin{column}{0.29\textwidth}
      \begin{tabular}{c|c|ccc}
      \cline{1-2}
      \multicolumn{1}{|c|}{2} &  &  &  &  \\ \cline{1-3}
      \multicolumn{1}{|c|}{3} &  & \multicolumn{1}{c|}{} &  &  \\ \cline{1-4}
      \multicolumn{1}{|c|}{4} &  & \multicolumn{1}{c|}{} & \multicolumn{1}{c|}{} &  \\ \hline
      \multicolumn{1}{|c|}{5} &  & \multicolumn{1}{c|}{} & \multicolumn{1}{c|}{} & \multicolumn{1}{c|}{} \\ \hline
        & 1 & \multicolumn{1}{c|}{2} & \multicolumn{1}{c|}{3} & \multicolumn{1}{c|}{4} \\ \cline{2-5}
      \end{tabular}

      \bigskip

      \begin{tabular}{c|c|ccc}
        \cline{1-2}
        \multicolumn{1}{|c|}{2} &  &  &  &  \\ \cline{1-3}
        \multicolumn{1}{|c|}{3} &  & \multicolumn{1}{c|}{} &  &  \\ \cline{1-4}
        \multicolumn{1}{|c|}{4} &  & \multicolumn{1}{c|}{} & \multicolumn{1}{c|}{} &  \\ \hline
        \multicolumn{1}{|c|}{5} &  & \multicolumn{1}{c|}{} & \multicolumn{1}{c|}{} & \multicolumn{1}{c|}{} \\ \hline
          & 1 & \multicolumn{1}{c|}{2} & \multicolumn{1}{c|}{3} & \multicolumn{1}{c|}{4} \\ \cline{2-5}
        \end{tabular}
    \end{column}
    \begin{column}{0.29\textwidth}
      \begin{tabular}{c|c|ccc}
        \cline{1-2}
        \multicolumn{1}{|c|}{2} &  &  &  &  \\ \cline{1-3}
        \multicolumn{1}{|c|}{3} &  & \multicolumn{1}{c|}{} &  &  \\ \cline{1-4}
        \multicolumn{1}{|c|}{4} &  & \multicolumn{1}{c|}{} & \multicolumn{1}{c|}{} &  \\ \hline
        \multicolumn{1}{|c|}{5} &  & \multicolumn{1}{c|}{} & \multicolumn{1}{c|}{} & \multicolumn{1}{c|}{} \\ \hline
          & 1 & \multicolumn{1}{c|}{2} & \multicolumn{1}{c|}{3} & \multicolumn{1}{c|}{4} \\ \cline{2-5}
        \end{tabular}

      \bigskip

      \begin{tabular}{c|c|ccc}
        \cline{1-2}
        \multicolumn{1}{|c|}{2} &  &  &  &  \\ \cline{1-3}
        \multicolumn{1}{|c|}{3} &  & \multicolumn{1}{c|}{} &  &  \\ \cline{1-4}
        \multicolumn{1}{|c|}{4} &  & \multicolumn{1}{c|}{} & \multicolumn{1}{c|}{} &  \\ \hline
        \multicolumn{1}{|c|}{5} &  & \multicolumn{1}{c|}{} & \multicolumn{1}{c|}{} & \multicolumn{1}{c|}{} \\ \hline
          & 1 & \multicolumn{1}{c|}{2} & \multicolumn{1}{c|}{3} & \multicolumn{1}{c|}{4} \\ \cline{2-5}
        \end{tabular}
    \end{column}
  \end{columns}
\end{frame}

\begin{frame}{Deducción de estados distinguibles}{Simplificación de AFDs}
  \begin{columns}
    \begin{column}{0.42\textwidth}
      \begin{center}
        \begin{tikzpicture}[node distance=1.6cm, stylename]
          \node[state, initial, accepting] (5) {$5$};
          \node[state, accepting, below right of=5] (3) {$3$};
          \node[state, accepting, above right of=5] (4) {$4$};
          \node[state, accepting, right of=4] (2) {$2$};
          \node[state, right of=3] (1) {$1$};

          \path
          (5) edge node {$a$} (4)
          (5) edge node {$b$} (3)
          (4) edge [loop above] node {$a$} (4)
          (4) edge [bend left] node {$b$} (2)
          (3) edge node {$a$} (4)
          (3) edge node {$b$} (1)
          (2) edge [bend left] node {$a$} (4)
          (2) edge node {$b$} (1)
          (1) edge [loop below] node {$a,b$} (1);
        \end{tikzpicture}
      \end{center}
    \end{column}
    \begin{column}{0.29\textwidth}
      \begin{tabular}{c|c|ccc}
      \cline{1-2}
      \multicolumn{1}{|c|}{2} &  &  &  &  \\ \cline{1-3}
      \multicolumn{1}{|c|}{3} &  & \multicolumn{1}{c|}{} &  &  \\ \cline{1-4}
      \multicolumn{1}{|c|}{4} &  & \multicolumn{1}{c|}{} & \multicolumn{1}{c|}{} &  \\ \hline
      \multicolumn{1}{|c|}{5} &  & \multicolumn{1}{c|}{} & \multicolumn{1}{c|}{} & \multicolumn{1}{c|}{} \\ \hline
        & 1 & \multicolumn{1}{c|}{2} & \multicolumn{1}{c|}{3} & \multicolumn{1}{c|}{4} \\ \cline{2-5}
      \end{tabular}

      \bigskip

      \begin{tabular}{c|c|ccc}
        \cline{1-2}
        \multicolumn{1}{|c|}{2} & $\times$ &  &  &  \\ \cline{1-3}
        \multicolumn{1}{|c|}{3} & $\times$ & \multicolumn{1}{c|}{} &  &  \\ \cline{1-4}
        \multicolumn{1}{|c|}{4} & $\times$ & \multicolumn{1}{c|}{} & \multicolumn{1}{c|}{} &  \\ \hline
        \multicolumn{1}{|c|}{5} & $\times$ & \multicolumn{1}{c|}{$\times$} & \multicolumn{1}{c|}{} & \multicolumn{1}{c|}{} \\ \hline
         & 1 & \multicolumn{1}{c|}{2} & \multicolumn{1}{c|}{3} & \multicolumn{1}{c|}{4} \\ \cline{2-5} 
        \end{tabular}
    \end{column}
    \begin{column}{0.29\textwidth}
      \begin{tabular}{c|c|ccc}
      \cline{1-2}
      \multicolumn{1}{|c|}{2} &  $\times$ &  &  &  \\ \cline{1-3}
      \multicolumn{1}{|c|}{3} &  $\times$ & \multicolumn{1}{c|}{} &  &  \\ \cline{1-4}
      \multicolumn{1}{|c|}{4} &  $\times$ & \multicolumn{1}{c|}{} & \multicolumn{1}{c|}{} &  \\ \hline
      \multicolumn{1}{|c|}{5} &  $\times$ & \multicolumn{1}{c|}{} & \multicolumn{1}{c|}{} & \multicolumn{1}{c|}{} \\ \hline
        & 1 & \multicolumn{1}{c|}{2} & \multicolumn{1}{c|}{3} & \multicolumn{1}{c|}{4} \\ \cline{2-5}
      \end{tabular}

      \bigskip

      \begin{tabular}{c|c|ccc}
        \cline{1-2}
        \multicolumn{1}{|c|}{2} & $\times$ &  &  &  \\ \cline{1-3}
        \multicolumn{1}{|c|}{3} & $\times$ & \multicolumn{1}{c|}{} &  &  \\ \cline{1-4}
        \multicolumn{1}{|c|}{4} & $\times$ & \multicolumn{1}{c|}{$\times$} & \multicolumn{1}{c|}{$\times$} &  \\ \hline
        \multicolumn{1}{|c|}{5} & $\times$ & \multicolumn{1}{c|}{$\times$} & \multicolumn{1}{c|}{$\times$} & \multicolumn{1}{c|}{} \\ \hline
         & 1 & \multicolumn{1}{c|}{2} & \multicolumn{1}{c|}{3} & \multicolumn{1}{c|}{4} \\ \cline{2-5} 
        \end{tabular}
    \end{column}
  \end{columns}
\end{frame}

\begin{frame}{Simplificación por clases de equivalencia}{Simplificación de AFDs}

  Formar clases de estados de un autómata que pudieran ser equivalentes. \pause

  \bigskip

  Al seguir examinando las clases, podremos percatarnos de si es necesario volver a \textbf{dividirlas}. \pause

  \bigskip

  Si las clases ya no pueden dividirse más, entonces hemos encontrado el autómata más pequeño.
  
\end{frame}

\begin{frame}{Simplificación por clases de equivalencia}{Simplificación de AFDs}
  \begin{center}
    \begin{tikzpicture}[node distance=2.1cm, stylename]
      \node[state, initial, accepting] (5) {$5$};
      \node[state, accepting, below right of=5] (3) {$3$};
      \node[state, accepting, above right of=5] (4) {$4$};
      \node[state, accepting, right of=4] (2) {$2$};
      \node[state, right of=3] (1) {$1$};

      \path
      (5) edge node {$a$} (4)
      (5) edge node {$b$} (3)
      (4) edge [loop above] node {$a$} (4)
      (4) edge [bend left] node {$b$} (2)
      (3) edge node {$a$} (4)
      (3) edge node {$b$} (1)
      (2) edge [bend left] node {$a$} (4)
      (2) edge node {$b$} (1)
      (1) edge [loop below] node {$a,b$} (1);
    \end{tikzpicture}
  \end{center}
\end{frame}

\begin{frame}{Simplificación por clases de equivalencia}{Simplificación de AFDs}
  \begin{center}
    \begin{tikzpicture}[node distance=2.1cm, stylename]
      \node[state, initial, accepting, draw=green] (5) {$5$};
      \node[state, accepting, below right of=5, draw=green] (3) {$3$};
      \node[state, accepting, above right of=5, draw=green] (4) {$4$};
      \node[state, accepting, right of=4, draw=green] (2) {$2$};
      \node[state, right of=3, draw=red] (1) {$1$};

      \path
      (5) edge node {$a$} (4)
      (5) edge node {$b$} (3)
      (4) edge [loop above] node {$a$} (4)
      (4) edge [bend left] node {$b$} (2)
      (3) edge node {$a$} (4)
      (3) edge node {$b$} (1)
      (2) edge [bend left] node {$a$} (4)
      (2) edge node {$b$} (1)
      (1) edge [loop below] node {$a,b$} (1);
    \end{tikzpicture}
  \end{center}
\end{frame}

\begin{frame}{Simplificación por clases de equivalencia}{Simplificación de AFDs}
  \begin{center}
    \begin{tikzpicture}[node distance=2.1cm, stylename]
      \node[state, initial, accepting, draw=green] (5) {$5$};
      \node[state, accepting, below right of=5, draw=blue] (3) {$3$};
      \node[state, accepting, above right of=5, draw=green] (4) {$4$};
      \node[state, accepting, right of=4, draw=blue] (2) {$2$};
      \node[state, right of=3, draw=red] (1) {$1$};

      \path
      (5) edge node {$a$} (4)
      (5) edge node {$b$} (3)
      (4) edge [loop above] node {$a$} (4)
      (4) edge [bend left] node {$b$} (2)
      (3) edge node {$a$} (4)
      (3) edge node {$b$} (1)
      (2) edge [bend left] node {$a$} (4)
      (2) edge node {$b$} (1)
      (1) edge [loop below] node {$a,b$} (1);
    \end{tikzpicture}
  \end{center}
\end{frame}

\begin{frame}{Simplificación por clases de equivalencia}{Simplificación de AFDs}
  \begin{center}
    \begin{tikzpicture}[node distance=2.1cm, stylename]
      \node[state, initial, accepting, draw=green] (A) {$A$};
      \node[state, accepting, draw=blue, right of=A] (C) {$C$};
      \node[state, draw=red, right of=C] (B) {$B$};

      \path
      (A) edge [loop above] node {$a$} (A)
      (A) edge [bend left] node {$b$} (C)
      (C) edge [bend left] node {$a$} (A)
      (C) edge node {$b$} (B)
      (B) edge [loop above] node {$a,b$} (B);
    \end{tikzpicture}
  \end{center}
\end{frame}

\begin{frame}{Simplificación por clases de equivalencia}{Simplificación de AFDs}
    \begin{center}
      \begin{tikzpicture}[node distance=2.5cm, stylename]
        \node[state, initial] (a) {$a$};
        \node[state, above of=a] (b) {$b$};
        \node[state, accepting, right of=a] (c) {$c$};
        \node[state, accepting, right of=b] (d) {$d$};
        \node[state, accepting, right of=c] (e) {$e$};
        \node[state, right of=d] (f) {$f$};

        \path
        (a) edge [bend left] node {$0$} (b)
        (a) edge node {$1$} (c)
        (b) edge [bend left] node {$0$} (a)
        (b) edge node {$1$} (d)
        (c) edge node {$0$} (e)
        (c) edge [bend left] node [left] {$1$} (f)
        (d) edge [bend left] node [below] {$0$} (e)
        (d) edge node {$1$} (f)
        (e) edge [loop below] node {$0$} (e)
        (e) edge node [midway, right] {$1$} (f)
        (f) edge [loop above] node {$0,1$} (f);
      \end{tikzpicture}
    \end{center}
\end{frame}

\begin{frame}{Simplificación por clases de equivalencia}{Simplificación de AFDs}
  \begin{center}
    \begin{tikzpicture}[node distance=2.5cm, stylename]
      \node[state, initial, draw=green] (a) {$a$};
      \node[state, above of=a, draw=green] (b) {$b$};
      \node[state, accepting, right of=a, draw=green] (c) {$c$};
      \node[state, accepting, right of=b, draw=green] (d) {$d$};
      \node[state, accepting, right of=c, draw=green] (e) {$e$};
      \node[state, right of=d, draw=red] (f) {$f$};

      \path
      (a) edge [bend left] node {$0$} (b)
      (a) edge node {$1$} (c)
      (b) edge [bend left] node {$0$} (a)
      (b) edge node {$1$} (d)
      (c) edge node {$0$} (e)
      (c) edge [bend left] node [left] {$1$} (f)
      (d) edge [bend left] node [below] {$0$} (e)
      (d) edge node {$1$} (f)
      (e) edge [loop below] node {$0$} (e)
      (e) edge node [midway, right] {$1$} (f)
      (f) edge [loop above] node {$0,1$} (f);
    \end{tikzpicture}
  \end{center}
\end{frame}

\begin{frame}{Simplificación por clases de equivalencia}{Simplificación de AFDs}
  \begin{center}
    \begin{tikzpicture}[node distance=2.5cm, stylename]
      \node[state, initial, draw=blue] (a) {$a$};
      \node[state, above of=a, draw=blue] (b) {$b$};
      \node[state, accepting, right of=a, draw=green] (c) {$c$};
      \node[state, accepting, right of=b, draw=green] (d) {$d$};
      \node[state, accepting, right of=c, draw=green] (e) {$e$};
      \node[state, right of=d, draw=red] (f) {$f$};

      \path
      (a) edge [bend left] node {$0$} (b)
      (a) edge node {$1$} (c)
      (b) edge [bend left] node {$0$} (a)
      (b) edge node {$1$} (d)
      (c) edge node {$0$} (e)
      (c) edge [bend left] node [left] {$1$} (f)
      (d) edge [bend left] node [below] {$0$} (e)
      (d) edge node {$1$} (f)
      (e) edge [loop below] node {$0$} (e)
      (e) edge node [midway, right] {$1$} (f)
      (f) edge [loop above] node {$0,1$} (f);
    \end{tikzpicture}
  \end{center}
\end{frame}

\begin{frame}{Simplificación por clases de equivalencia}{Simplificación de AFDs}
  \begin{center}
    \begin{tikzpicture}[node distance=2.5cm, stylename]
      \node[state, initial, draw=blue] (a) {$a$};
      \node[state, accepting, right of=a, draw=green] (b) {$b$};
      \node[state, right of=b, draw=red] (c) {$c$};

      \path
      (a) edge [loop above] node {$0$} (a)
      (a) edge node {$1$} (b)
      (b) edge [loop below] node {$0$} (b)
      (b) edge node {$1$} (c)
      (c) edge [loop above] node {$0,1$} (c);
    \end{tikzpicture}
  \end{center}
\end{frame}

% \section*{Referencias}

% \begin{frame}[t]{Referencias}
    % \nocite{bibID01}
    % \nocite{bibID02}

    % \bibliographystyle{IEEE}
    % \bibliography{biblio}
% \end{frame}

\end{document}